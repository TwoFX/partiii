\begin{enumerate}
	\item Let $k$ be a field. Then $\Spec k = (\set{0}, k)$.

		What does giving a morphism $f\colon \Spec k\to X$ a scheme mean?

		First, we need to choose a point $x \in X$, the image of $f$. Second,
		we get a local ring homomorphism
		\[ f^\#_x\colon \mathcal{O}_{X, x} \to \mathcal{O}_{\Spec k, 0} \cong k, \]
		i.e., $(f_x^\#)^{-1}(0) = m_x \subseteq \mathcal{O}_{X, x}$, the maximal
		ideal of $\mathcal{O}_{X, x}$. Thus we get a factorization
		$f^\#_x\colon \mathcal{O}_{X, x}\to \mathcal{O}_{X, x}/m_x \to k$.
		The middle quotient is a field denoted as $\kappa(x)$, the residue field
		of $X$ at $x$.

		Thus $f$ induces an inclusion $\kappa(x) \to k$.

		Conversely, given an inclusion $\iota\colon\kappa(x)\to k$ we get a morphism of
		schemes $\Spec k\to X$ by defining $f(0) = x$ and
		$f^\#\colon \mathcal{O}_X\to f_*k$ by defining $s\mapsto \iota(s(x)) \in k$,
		where $s(x)$ means taking the stalk of $s$ at $x$.

		Moral: Giving a morphism $f\colon \Spec k\to X$ is equivalent to giving
		a point $x \in X$ and an inclusion $\kappa(x)\to k$.
\end{enumerate}
