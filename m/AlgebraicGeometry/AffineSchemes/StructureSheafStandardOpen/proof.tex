Define
\begin{align*}
	\Psi\colon A_f&\to \mathcal{O}(D(f))\\
	a/f^n&\mapsto (p\mapsto a/f^n).
\end{align*}
This makes sense since if $f\notin p$, then $f^n\notin p$.
As usual, we will verify injectivity and surjectivity.

For inejctivity, assume that $\Psi(a/f^n) = 0$. Then for all $p \in D(f)$, we have
$a/f^n = 0$ in $A_p$, i.e., there is  $h \in A\setminus p$ such that $ha = 0$ in $A$.

Let $I = \set{q \in A\given q\cdot a = 0}$ (the annihilator of $a$).
So $h \in I$, but $h \notin p$, so $I\subsetneq p$.
This is true for all $p \in D(f)$, so $V(I)\cap D(f) = \varnothing$. Thus
$f \in \bigcap_{p \in V(I)}p = \sqrt{I}$, as we know from commutative algebra.
This means that $f^n \in I$ for some $n>0$. Thus $f^n\cdot a = 0$, so
$a/f = 0$ in $A_f$, so $\Psi$ is injective.

Next, we will prove surjectivity. Let $s \in \mathcal{O}(D(f))$. Cover
$D(f)$ with open sets $V_i$ on which $s$ is represented by as $a_i/g_i$ with
$a_i, g_i \in A$, $g_i \notin p$ whenevery $o \in V_i$. Thus $V_i \subseteq D(g_i)$.
By question 1 on the first example sheet, the sets of the form $D(h)$ form a
base for the Zariski topology on $\Spec A$. Thus we can assume $V_i = D(h_i)$
for some $h_i \in A$. Since $D(h_i) \subseteq D(g_i)$, we have
$V(h_i) \supseteq V(g_i)$, so $\sqrt{(h_i)} \subseteq \sqrt{(g_i)}$, since the
radical is the intersection of all the primes of $V(\cdot)$. Hence,
$h_i^n \in (g_i)$ for some $n$, say $h_i^n = c_ig_i$, so we have
$\frac{a_i}{g_i} = \frac{c_ia_i}{h_i^n}$. Now replace $h_i$ by $h_i^n$. This does
not change the open sets because in general $D(h_i) = D(h_i^n)$ and replace
$a_i$ by $c_ia_i$.

The situation so far is that we may assume that $D(f)$ is covered by sets
$D(h_i)$ such that $s$ is represented by $a_i/h_i$ on $D(h_i)$.

We now claim that $D(f)$ can be covered by a finite number of the  $D(h_i)$, i.e.,
$D(f)$ is quasicompact. Indeed, $D(f) \subseteq \bigcup_i D(h_i)$, which is equivalent
to $V(f) \supseteq \bigcap_i V(h_i) = V(\sum (h_i))$. This in turn is equivalent to
$f \in \sqrt{\sum_i (h_i)}$ (because it just says that $f$ is in every prime ideal
containing $\sum_i (h_i)$), which is equivalent to there being some $n$ such that
$f^n \in \sum_i (h_i)$. Hence, we can write $f^n = \sum_{i \in I} b_i h_i$ for some finite
set $I$.

Reversing this argument yields that $D(f) \subseteq \bigcup_{i \in I} D(h_i)$ as
required, completing the proof of the claim.

We now pass to this finite subcover $\set{D(h_i)}_{i \in I}$. On
$D(h_i)\cap D(h_j) = D(h_ih_j)$, note $a_i/h_i$ and $a_j/h_j$ both represent $s$. Since we
have already shown injectivity, this means that $a_ih_j/h_ih_j = a_jh_i/h_ih_j$ in $A_{h_ih_j}$.

Thus, for some $n$, $(h_ih_j)^n(h_ja_i - h_ia_j) = 0$ in $A$. We can pick an $n$
sufficiently large to work for all pairs $i, j$ (since there are only finitely many
such pairs).

We rewrite this equality as $h_j^{n+1}(h_i^na_i) = h_i^{n+1}(h_j^na_j) = 0$.
Now replace $h_i$ by $h_i^{n+1}$, and $a_i$ by $h_i^na_i$ (this is allowed because
$\frac{a_i}{h_i}= \frac{a_ih_i^n}{h_i^{n+1}}$. Thus we can assume that $s$ is still
represented on $D(h_i)$ by $a_i/h_i$ but also for each $i, j$ we have $h_ia_j = h_ja_i$

Note $f^n = \sum b_ih_i$ for some $b_i \in A$, since the $D(h_i)$ cover $D(f)$,
and define $a \coloneqq b_ia_i$.

Then for any $j$, we have \[ h_ja = \sum_i b_ia_ih_j = \sum_i b_ia_jh_i = f^na_j. \]
This means that $a/f^n = a_jh_j$ on $D(h_j)$. Hence $\Psi(a/f^n) = s$, completing
the proof of surjectivity.
