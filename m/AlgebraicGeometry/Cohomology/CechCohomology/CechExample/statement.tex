\begin{enumerate}
	\item Take $X = S^1$ with the usual topology and $\mathcal{F} = \underline{\mathbb{Z}}$
		the constant sheaf, i.e., the sheaf associated to the presheaf of locally constant
		functions.

		Consider the open cover consisting of two open sets, one, $U$, a slightly enlarged left
		hemisphere and the other, $V$, a slightly enlared right hemisphere. There is an image missing
		here. Then $C^0(\mathcal{U}, \mathcal{F}) = \Gamma(U, \mathcal{F}) \times \Gamma(V, \mathcal{F}) = \mathbb{Z}\times \mathbb{Z}$.
		Also, $C^1(\mathcal{U}, \mathcal{F}) = \Gamma(U\cap V, \mathcal{F}) = \mathbb{Z}^2$.
		Next, we have to check the boundary map $d\colon C^0(\mathcal{U}, \mathcal{F}) \to C^1(\mathcal{U}, \mathcal{F})$.
		It sends $(a, b)$ to $(b-a, b-a)$. Hence,

		$\check{H}^0(\mathcal{U}, \mathcal{F}) = \ker d \cong \mathbb{Z}$, and
		$\check{H}^1(\mathcal{U}, \mathcal{F}) = \coker d\cong \mathbb{Z}$.

		Note that this agrees with the singular cohomology of $S^1$. This is not
		an accident. In this case, this also agrees with the sheaf cohomology
		$H^i(S^1, \mathcal{F})$.

	\item Next, take $\mathcal{F} = \mathcal{O}_{\mathbb{P}^1}(-2)$, $\mathbb{P}^1 =\Proj k[X_0, X_1]$.
		Recall that $\mathcal{O}_{\mathbb{P}^1}(1)$ has a transition map from
		$U_0 = D_+(X_0)$ to $U_1 = D_+(X_1)$ given by $X_0/X_1$.  Thus $\mathcal{O}_{\mathbb{P}^1}(-2)$
		has transition map $X_1^2/X_0^2$. Taking $\mathcal{U} = \set{U_0, U_1}$, we get
		\[ C^0(\mathcal{U}, \mathcal{F}) = \Gamma(U_0, \mathcal{O}_{\mathbb{P}^1}(-2))\times \Gamma(U_1, \mathcal{O}_{\mathbb{P}^1}(-2)) = k[X_1/X_0] \times k[X_0/X_1]. \]
		Furthermore, $C^1(\mathcal{U}, \mathcal{F}) = \Gamma(U_0\cap U_1, \mathcal{O}_{\mathbb{P}^1}(-2)) = k[X_1/X_0]_{X_1/X_0}$,
		using the same trivialization on $U_0\cap U_1$ which we used on $U_1$. Then
		$d(f, g) = g - f\cdot \frac{X_1^2}{X_0^2}$, where we had to take into account that $f$
		uses a different trivialization, so we had to apply the transition map.

		Then $\ker d = 0$ and $\coker d$ is one-dimensional and generated by
		the monomial $X_1/X_0$. Hence $\check{H}^0(\mathcal{U}, \mathcal{O}_{\mathbb{P}^1}(-2)) = 0$ and
		$\check{H}^1(\mathcal{U}, \mathcal{O}_{\mathbb{P}^1}(-2)) = k$.
\end{enumerate}
