We will give some motivation for this construction.

Let $B$ be an algebra (think $X = \Spec B$, $Y = \Spec A$) and $M$ a $B$-module.
An $A$-derivation $d\colon B\to M$ is a map such that for all $a \in A$, $b, b' \in B$ we have
\begin{enumerate}
	\item $d(b + b') = d(b) + d(b')$,
	\item $d(bb') = bd(b') + b'd(b)$, and
	\item $d(a) = 0$.
\end{enumerate}

The module of relative differentials $\Omega_{B/A}$ is a $B$-module satisfying
a universal property: there is an $A$-derivation $d\colon B\to \Omega_{B/A}$ such
that for any other $A$-derivation $d'\colon B\to M$ we find a homomorphism
of $B$-modules $g\colon \Omega_{B/A}\to M$ making the diagram
\[\begin{tikzcd}
	B\ar[r, "d"]\ar[dr, "d'"] & \Omega_{B/A}\ar[d, "g"]\\
	& M
\end{tikzcd}\]
commute.

As an example, take $B = k[X_1, \ldots, X_n]$ and $A = k$. We may describe
$\Omega_{B/A}$ as a direct sum
\[ \Omega_{B/A} = \bigoplus_{i=1}^n BdX_i, \]
where $dX_i$ is a generator of the summand. We set $d(X_i) = dX_i$, $d(f) = \sum_{i=1}^n \frac{\partial f}{\partial X_i} dX_i$.
Given $d'\colon B\to M$ define $g\colon \Omega_{B/A} \to B$ by $g(dX_i) = d'(X_i)$.

More generally, we can construct $\Omega_{B/A}$ as follows. We have a homomorphism
$\varphi\colon B\tensor_A B\to B$ sending $b\tensor b'\mapsto bb'$. Let $I\coloneqq \ker\varphi$.
Then $I/I^2$ is a $B$-module and we may define $d\colon B\to I/I^2$ via
$d(b) = 1\tensor b-b\tensor 1$ and we can check that $\Omega_{B/A} = I/I^2$ has
the universal property (cf. Example Sheet 4).

Notive that if $X = \Spec B$, $Y = \Spec A$, then $\varphi$ induces the diagonal
morphism $\Delta\colon X\to X\times_Y X$ and $\tilde{I} = \mathcal{I}_{X/X\times_Y X}$.

Then $\Delta^*\mathcal{I}_{X/X\times_Y X}/\mathcal{I}_{X/X\times_Y X}^2$
coincides with the sheafification of $I/I^2$ viewing $I/I^2$ as a $B$-module.

In this way, the sheaf of differentials is the geometric version of the module
of relative differentials.

For example if $Y = \Spec k$ and $X$ is a nonsingular connected variety, then
so is $X\times_k X$ and $\codim(\Delta(X), X\times_k X) = \dim X$.
So $\Omega_{X/\Spec k} = \Omega_X$ is a locally free sheaf of rank $\dim X$.

As an example within the example, if $X = \mathbb{A}^n_k$, then $\Omega_X = \bigoplus_{i=1}^n \mathcal{O}_X dX_i$.
One can think of $\Omega_X$ of as the cotangent bundle. Then
$\mathcal{T}_X = \SheafHom_{\mathcal{O}_X}(\Omega_X, \mathcal{O}_X)$ is the tangent
bundle.
