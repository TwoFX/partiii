We will calculate this using the comparison theorem and the standard affine
cover $\mathcal{U} = \set{U_i = D_+(X_i)\given 0\leq i\leq r}$. Furthermore,
we will be calculating the cohomology of $\mathcal{F}\coloneqq \bigoplus_{n \in \mathbb{Z}} \mathcal{O}_X(n)$
(this is not coherent, but it is quasicoherent)
and then we will be done because cohomology commutes with coproducts (this must
be checked).

The key point is the following: recall that the transition map for $\mathcal{O}_X(1)$
from $U_i$ to $U_j$ is $X_i/X_j$, and so the transition maps for $\mathcal{O}_X(m)$
are $X_i^m/X_j^m$. For $I \subseteq \set{0, \ldots r}$, we have $U_I = \cap_{i \in I} D_+(X_i) = D_+(X_I)$,
where $X_i\coloneqq \prod_{i \in I} X_i$.

Thus $\Gamma(U_I, \mathcal{O}_{\mathbb{P}^r}\cong S_{(X_I)}$. We will identify
$\Gamma(U_I, \mathcal{O}_X(m))$ with the $k$-subspace of $S_{X_I}$ spanned by
Laurent monomials of degree $m$, i.e., monomials of the form
$X_0^{a_0}\cdots X_r^{a_r}$ with $\sum a_i = n$ and $a_i < 0\implies i \in I$.

Given such a monomial $M$, using the trivialization on $U_i$, we will
identify the section of $\mathcal{O}_X(m)$ defined by $M$ with
$M/X_i^m \in \Gamma(U_I, \mathcal{O}_{\mathbb{P}^r}$ (with $i \in I$).

If $i, j \in I$, then note
\[ \frac{M}{X_i^m}\cdot \frac{X_i^m}{X_j^m} = \frac{M}{X_j^m}. \]
Thus we have a canonical identification of $\Gamma(U_I, \mathcal{O}_{\mathbb{P}^r}(m))$
with the space spanned by Laurent monomials of degree $m$.

Thus $\Gamma(U_I, \mathcal{F})$ can be identified with $S_{X_I}$.

Now we have a Cech complex $C^\bullet(\mathcal{U}, \mathcal{F})$ which looks like
\[\begin{tikzcd}
	\prod_{0\leq i_0\leq r}S_{X_{i_0}} \ar[r, "d_0"] & \prod_{0\leq i_0\leq i_1\leq r} S_{X_{i_0}X_{i_1}}\ar[r] & \cdots\ar[r, "d_{r-1}"] & S_{X_0\cdots X_r}.
\end{tikzcd}\]

Note that $H^0(X, \mathcal{F}) = \ker d_0$. Note also that all modules in the Cech
complex are sub-$S$-modules of $S_{X_0\cdots X_r}$.
We have
\[ d_0((f_i)_{i \in\set{0, \ldots, r}}) = (f_j - f_i)_{0\leq i<j\leq r}. \]
Thus if $(f_i)_{0\leq i\leq r} \in \ker d_0$ we actually have $f_i = f_j$ for all $i$.

Thus $f_i, f_j \in S$ since otherwise $f_i$ involves a negative power of
$X_i$, which can't occur in $f_j$, or vice versa. Thus $_i = f$ for all $i$
with $f \in S$, and hence $\ker d_0\cong S$. Thus $H^0(X, \mathcal{F}) = S$
preserving degrees, so $H^0(X, Ocal_X(m)) = S_m$.

Now consider
\[ d_{r-1}\colon \prod_{0\leq k\leq r} S_{X_0\cdots \widehat{X_k}\cdots X_r} \to S_{X_0\cdots X_r}. \]
Note $S_{X_0\cdots X_r}$ is the $k$-vector space with basis
$\prod_{i = 0}^r X_i^{a_i}$, where the $a_i \in \mathbb{Z}$ and the image of
$d_{r-1}$ is spanned by monomials of the form
$\prod_{i = 0}^r X_i^{a_i}$ with at least one $a_i\geq 0$. Thus a basis
for $\coker d_{r-1}$ is given by
\[ \set{\prod_{i =0}^r X_ia_i\given a_i\leq -1 \forall i}. \]
In particular, $H^r(X, \mathcal{O}_X(-r-1))$ is generated by
$X_0^{-1}\cdots X_r^{-1}$, and thus $H^r(X, \mathcal{O}_X(r-1))\cong k$.

For the perfect pairing, note that $H^0(\mathbb{P}^r, \mathcal{O}_{\mathbb{P}^r}(n)) = 0$
for $n < 0$, as $S_n = 0$ for $n < 0$, and $H^r(\mathbb{P}^r, \mathcal{O}_{\mathbb{P}^r}(-n-r-1)) = 0$
for $n < 0$ as there are no monomials with only negative exponents of degree
greater than $-r-1$, and so there is nothing to check in this case.

If $n\geq 0$, we have a basis
\[ \set{\prod_i X_i^{m_i}\given\sum m_i = n, m_i\geq 0} \]
for $H^r(X, \mathcal{O}_X(n))$ and a basis
\[ \set{\prod_i X_i^{\ell_i}\given \sum \ell_i = -n-r-1, \ell_i\leq -1} \]
for $H^r(X, \mathcal{O}_X(-n-r-1))$. The perfect pairing is given by
\[ (X_0^{m_0}\cdots X_r^{m_r})\cdots(X_0^{\ell_0}\cdots X_r^{\ell_r}) = X_0^{m_0+\ell_0}\cdots X_r^{m_r+\ell_r}, \]
where we interpret the result as $0$ if $m_i+\ell_i \leq 0$ for any $i$. Indeed,
this gives a pairing
\[ H^0(X, \mathcal{O}_X(n))\times H^r(X, \mathcal{O}_X(-n-r-1))\to H^r(X, \mathcal{O}_X(-r-1)) = k(X_0\cdots X_r)^{-1} \]
and it is easy to check that it is perfect.

It remains to show that $H^i(X, \mathcal{O}_X(m)) = 0$ for $0 < i < r$. The proof
will be by induction on $r$.

In the base case $r=1$ the statement is vacuous. For the induction step, if
we localize $C^\bullet(\mathcal{U}, \mathcal{F})$ at $X_r$ as graded $S$-modules,
we get a Cech complex which calculates the cohomology groups $H^i(\mathcal{U}_r, \mathcal{F}|_{\mathcal{U}_r})$
(calculate using the Cech cover $\mathcal{U}' = \set{U_i\cap U_r\given 0\leq i\leq r}$).
But $U_r\cong \mathbb{A}^r_k$, and Cech cohomology can also be calulated via
the cover $\set{U_r}$, so $H^i(\mathcal{U}_r, \mathcal{F}|_{\mathcal{U}_r}) = 0$ for
all $i$\footnote{Observe that this implies that if $\mathcal{F}$ is any quasicoherent
sheaf on an affine scheme $X$, then $H^i(X, \mathcal{F}) = 0$ for all $i > 0$. But this
is not really a proof, because the proof of the comparison theorem uses this fact. Actually,
vanishing cohomology for any quasicoherent sheaf in positive degree characterises affine
schemes.}

Now localizing at $X_r$ is an exact functor, so
\[ H^i(C^\bullet(\mathcal{U}, \mathcal{F})_{X_r}) = H^i(C^\bullet(\mathcal{U}, \mathcal{F}))_{X_r}, \]
so thus, $H^i(X, \mathcal{F}_{X_r} = H^i(\mathcal{U}_r, \mathcal{F}|_{\mathcal{U}_r}) = 0$ for $i > 0$.
For this to be the case, every element of $H^i(X, \mathcal{F})$ must be annihilated by some
power of $X_r$.
