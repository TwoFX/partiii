It will suffice to show that $D$ is principal if and only if $\mathcal{O}_X(D)\cong \mathcal{O}_X$.

If $D$ is principal, then $D$ is represented by a rational function $(X, f)$ with
$f \in \Gamma(X, \mathcal{K}_X^\times)$. So $\mathcal{O}_X(D)= \mathcal{O}_X\cdot f^{-1}\cong \mathcal{O}_X$,
where the middle term is interpreted as a subsheaf of $\mathcal{K}_X$.

Conversely, if $\mathcal{O}_X(D)\cong \mathcal{O}_X$ let \[ 1 \in \Gamma(X, \mathcal{O})\mapsto f \in \Gamma(X, \mathcal{O}_X(D)) \subseteq \Gamma(X, \mathcal{K}_X) \]
and note that we in fact have $f \in \Gamma(X, \mathcal{K}_X^\times)$. Then
$(X, f^{-1})$ representes $D = \set{(U_i, g_i)}$ as $f^{-1} $ and $g_i$ only
differ by a factor of an invertible function on $U_i$. Thus $D$ is principal.
