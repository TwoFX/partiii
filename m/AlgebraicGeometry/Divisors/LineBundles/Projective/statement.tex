Observe that $\mathbb{P}_k^n$ satisfies all of the hypotheses of this theorem.
We have isomorphisms $\mathbb{Z} \cong \Cl \mathbb{P}^n \cong \CaCl \mathbb{P}^n \cong \Pic \mathbb{P}^n$,
using that $\mathbb{\mathbb{P}^n_k}$ is nonsingular, i.e., all local rings are
regular. The generator of $\Cl \mathbb{P}^n$ is $H$, a hyperplane, and it is not hard to see
that $\mathcal{O}_{\mathbb{P}^n}(H) = \mathcal{O}_{\mathbb{P}^n}(1)$ constructed previously
(checking this is an very important exercise). Hence, $\Pic \mathbb{P}^n$ is generated by
$\mathcal{O}_{\mathbb{P}^n}(1)$. Thus, it makes sense to define $\mathcal{O}_{\mathbb{P}^n}(d)$
as $\mathcal{O}_{\mathbb{P}^n}(1)^{\tensor d}$ for $d > 0$ and $\mathcal{O}_{\mathbb{P}^n}(-d)^{\vee}$
for $d < 0$. We have $\mathcal{O}_{\mathbb{P}^n}(d) \cong \mathcal{O}_{\mathbb{P}^n}(dH)$.

We will see that $\Gamma(\mathbb{P}^n, \mathcal{O}_{\mathbb{P}^n}(d))\cong S_d$, where
$S = k[X_0, \ldots, X_n] = \bigoplus_d S_d$, where $S_d$ is the component of degree $d$.

As an exercise, show that if $f \in S_d$ is a homogeneous polynomial of degree $d$
and $f = \prod_{i=1}^n f_i^{d_i}$ is its prime factorization, then
$(f)_0 = \sum d_i V(f_i)$.
