Want to construct an inverse $\beta$ to $\alpha$. Define for
$\varphi\colon A\to \Gamma(X, \mathcal{O}_X)$. Define $\beta(\varphi)$ as follows.
Cover $X$ with affine schemes $\set{U_i}$,  $U_i = \Spec B_i$. We have
restriction maps $\Gamma(X, \mathcal{O}_X) \to \Gamma(U_i, \mathcal{O}_X) = \Gamma(U_i, \mathcal{O}_{\Spec B_i}) = B_i$.

This igves by composition with $\varphi$ maps $\varphi_i\colon A\to B_i$. This
induces a morphism $f_i\colon U_i = \Spec B_i \to \Spec A$. We want
to show that we can glue the $f_i$, by first showing that they agree on $U_i\cap U_j$.

Using Exercises 1 and 11, we may cover $U_i\cap U_j$ with affine schemes $\set{U_{ijk}}$,
where $U_{ijk} = \Spec B_{ijk}$. Then we have a commutative diagram
\[\begin{tikzcd}
	\Gamma(X, \mathcal{O}_X)\ar[r]\ar[d] & B_j\ar[d]\\
	B_i\ar[r] & B_{ijk}
\end{tikzcd}\]
of restriction maps for $\mathcal{O}_X$.

Thus the compositions
\[\begin{tikzcd}
	U_{ijk}\ar[r, hook] & U_i\ar[r, "f_i"] & \Spec A\\
	U_{ijk}\ar[r, hook] & U_j\ar[r, "f_j"] & \Spec A
\end{tikzcd}\]
agree. Thus $f_i|_{U_i\cap U_j} = f_j|_{U_i\cap U_j}$.
We can now glue these morphisms to get a morphism $f\colon X\to \Spec A$.
\begin{itemize}
	\item Obtaining $f$ as a continuous map is no problem.
	\item We need to construct $f^\#$.
\end{itemize}
Given $V \subseteq \Spec A$, we need a map $f^\#_V\colon \Gamma(V, \mathcal{O}_{\Spec A}\to \Gamma(f^{-1}(V), \mathcal{O}_X)$.

Note $f^{-1}(V)$ is covered by the sets $f_i^{-1}(V) = f^{-1}(V) \cap U_i$. So we have for
$s \in \Gamma(V, \mathcal{O}_{\Spec A})$, $f_i^\#(s) \in \Gamma(f^{-1}_i(v), \mathcal{O}_X)$
and since $f_i|_{U_i\cap U_j} = f_j|_{U_i\cap U_j}$ we have
\[f_i^\#(s)|_{f_i^{-1}(V)\cap f_j^{-1}(V)} = f_j^\#(s)|_{f_i^{-1}(V)\cap f_j^{-1}(V)}. \]
By the sheaf gluing axiom, we obtain $f^\#(s) \in \Gamma(f^{-1}(V), \mathcal{O}_X)$.

Note: This is a general fact: given $\set{U_i}$ a cover of $X$ and $f_i\colon U_i\to Y$ morphisms
such that $f_i|_{U_i\cap U_j} = f_j|_{U_i\cap U_j}$, then we obtain a glued morphism.

This gives $\beta\colon \Hom_{\text{Ring}}(A, \Gamma(X, \mathcal{O})X))\to \Hom_{\text{Sch}}(X, \Spec A)$.
Need to check:
\begin{itemize}
	\item $\alpha \circ \beta$ is the identity: given $\varphi$, $f = \beta(\varphi)$ is contructed
		such that the composition
		\[\begin{tikzcd}
			A\ar[r, "f^\#"] & \Gamma(X, \mathcal{O}_X)\ar[r] & \Gamma(U_i, \mathcal{O}_X)
		\end{tikzcd}\]
		coincides with
		\[\begin{tikzcd}
			A\ar[r, "\varphi"] & \Gamma(X, \mathcal{O}_X)\ar[r] & \Gamma(U_i, \mathcal{O}_X),
		\end{tikzcd}\]
		so by the first sheaf axiom we must have $f^\# = \varphi = \alpha(f)$.
	\item $\beta \circ \alpha$ is the identity: given a morphism
		$f\colon X\to \Spec A$, this induces by restriction to $U_i$ a morphism
		$U_i\to \Spec A$; necessarily induced by the composition
		\[\begin{tikzcd}
			A\ar[r, "f^\#"] & \Gamma(X, \mathcal{O}_X)\ar[r] & \Gamma(U_i, \mathcal{O}_X) = B_i.
		\end{tikzcd}\]
		SO $f$ is induced by $\alpha(f)$ on open sets $U_i$. So $f|_{U_i} = \beta(\alpha(f))|_{U_i}$.
		Thus $ = \beta(\alpha(f))$.
\end{itemize}

Note: not every details has been checked here, for example that $f^\#$ constructed above
is indeed a morphism of locally ringed spaces (the reason is because locally, we
already started with a morphism). Checking details is essential to understanding what
is important and what isn't. During marking, not checking unimportant/easy things is
usually not a problem, forgetting to check important things is.

