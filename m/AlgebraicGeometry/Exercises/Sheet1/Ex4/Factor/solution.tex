Let $U$ be an open and let $s \in \mathcal{F}^+U$.
Cover $U$ with the $V_p$ from the definition of $\mathcal{F}^+$
and obtain the associated $s_{V_p} \in \mathcal{F}V_p$. Define
$t_{V_p} \coloneqq \varphi_{V_p}(s_{V_p}) in \mathcal{G}V_p$. We can calulate that
for $q \in V_p$ we have
\[ (t_{V_p})_q = (\varphi_{V_p}(s_{V_p}))_q = \varphi_q((s_{V_p})_q) = \varphi_q(s(q)). \]
Therefore, Lemma 2 gives us a unique $t_U \in \mathcal{G}_U$ such that
\begin{equation}\tag{$\star$}
	\forall q \in U\colon (t_U)_q = \varphi_q(s(q)).
\end{equation}
We define $\varphi^+_U(s) = t_U$.

This is indeed a morphism of sheaves: if $V \subseteq U$ and $s \in \mathcal{F}^+U$,
then
\[ \varphi^+(s|_V) = \varphi^+(s)|_V \]
follows from the fact that, using $(\star)$, the germ of both sides at $p \in V$
is just $\varphi_p(s(p))$. By Lemma 1, the two sides are equal.

Similarly, if $s \in \mathcal{F}U$ and $p \in U$, then
\[ (\varphi^+_U\theta_U(s))_p \stackrel{(\star)}{=} \varphi_q(\theta(s)(q)) = \varphi_q(s_q) = (\varphi_U(s))_q, \]
so $\varphi^+_U \circ \theta_U = \varphi_U$ by Lemma 1, so
$\varphi^+ \circ \theta = \varphi$.

Finally, to see uniqueness, assume that $\varphi^\#$ satisfies
 $\varphi^\# \circ \theta = \varphi$. Let $s \in \mathcal{F}^+U$ and $p \in U$.
By definition of $\mathcal{F}^+$ there is $p \in V_p \subseteq U$, $s_{V_p} \in \mathcal{F}V_p$
such that $\forall q \in V_p\colon (s_{V_p})_q = s(q)$. The condition can be
reprased as $s|_{V_p} = \theta(s_{V_p})$ and we calculate
\begin{align*}
	(\varphi^\#_U(s))_p &= (\varphi_U^\#(s)|_{V_p})_p
= (\varphi_{V_p}^\#(s|_{V_p}))_p
= (\varphi_{V_p}^\#(\theta(s_{V_p})))_p\\
	&= (\varphi_{V_p}^+(\theta(s_{V_p})))_p = \ldots = (\varphi^+_U(s))_p,
\end{align*}
so by Lemma 1, we have $\varphi^+_U = \varphi^\#_U$, so $\varphi^+ = \varphi^\#$,
completing the proof of uniqueness.
