Assume that $f\colon \mathcal{F}\to \mathcal{G}$ and $g\colon \mathcal{G}\to \mathcal{H}$
are morphisms of sheaves such that $g \circ f = 0$. Consider the diagram
\[\begin{tikzcd}[row sep = 1cm, column sep=1cm]
	\mathcal{F}\ar[r, "f"]\ar[d] & \mathcal{G}\ar[r, "g"] & \mathcal{H}\\
	\im' f\ar[ur]\ar[r, "\theta"]\ar[rr, bend right, "\iota"] &\im f\ar[r, "\varphi"]\ar[u] & \ker g,\ar[ul]
\end{tikzcd}\]
where the map $\iota$ is an inclusion of subpresheaves of $\mathcal{G}$ and $\varphi$ is
induced by $\iota$. We say that $\im f = \ker g$ if $\varphi$ is an isomorphism. By
a result of the lecture, this is the case if and only if forall $p \in X$, the
induced map $\varphi_p\colon (\im f)_p \to (\ker g)_p$ is an isomorphism. Since
$\theta$ induces isomorphisms on stalks and the bottom triangle commutes, this is
the case if and only if $\iota_p\colon (\im' f)_p\to (\ker g)_p$ is an isomorphism
for every $p \in X$. Now consider the diagram
\[\begin{tikzcd}
	(\im' f)_p \ar[r, "\iota_p"]\ar[d, "\cong"] & (\ker g)_p\ar[d, "\cong"]\\
	\im f_p\ar[r, "i"] & \ker g_p,
\end{tikzcd}\]
where the left and right maps are the isomorphisms defined in the proof of a result
from the lecture and the bottom map is just the inclusion (this makes sense since
$g \circ f = 0 \iff \forall p \in X\colon g_p \circ f_p = 0$ as stalks characterize morphisms). The
diagram commutes since none of the maps actually does anything. Since the left and
right maps are isomorphisms, we have that the top map is an isomorphism if and only
if the bottom map is an isomophism.

But the bottom map is an isomorphism if and only if the sequence
\[\begin{tikzcd}
	\mathcal{F}_p\ar[r, "f_p"] & \mathcal{G}_p\ar[r, "g_p"] & \mathcal{H}_p
\end{tikzcd}\]
is exact, so putting everything together, we find that $(f, g)$ is exact if and only if
$(f_p, g_p)$ is exact for every $p \in X$.
