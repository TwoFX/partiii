The natural map $e\colon \mathcal{F}\to \mathcal{F}/\mathcal{F}'$ is
given as the composite $\theta \circ \hat{e}$, where $\hat{e}$ is
the map $\mathcal{F}\to \coker' i$, where $i\colon \mathcal{F}'\to \mathcal{F}$ is
the inclusion and $\coker' i$ is the presheaf cokernel of $i$, and
$\theta$ is the natural map into the sheafification.

For every $p \in X$, $\theta_p$ is surjective because $\theta$ induces
isomorphisms on stalks, and $\hat{e}_p$ is surjective, because
$\hat{e}$ is surjective on open sets, which in particular implies
surjectivity on stalks. Hence $e_p$ is surjective as the composite
of two open maps. By a result from the lecture, this implies that $e$
is surjective.

Since for any open set $U$ and $s \in \mathcal{F}'U$ we have
$e_U(s) = \theta_U(\hat{e}_U(s)) = \theta_U(0) = 0$, we obtain a map
$\varphi\colon\mathcal{F}' \to \ker e$. Let $p \in X$.
\[\begin{tikzcd}
	\mathcal{F}_p'\ar[d, "\varphi_p"]\ar[r, "i_p"] & \mathcal{F}_p\ar[r, "e_p"]\ar[dr, "\hat{e}_p"] & (\mathcal{F}/\mathcal{F}')_p\\
	(\ker e)_p\ar[ur]\ar[r, "\cong"] & \ker e_p\ar[u] & (\coker' i)_p\ar[u, "\theta_p"]
\end{tikzcd}\]
The map $\varphi_p$ is injective because $i_p$ is, and it is surjective, because
$\ker e_p = \ker \hat{e}_p = i_p$. Hence $\ker e = \mathcal{F}'$ as subsheaves
of $\mathcal{F}$.

Now we have $\im i = \mathcal{F}'$ as subsheaves of $\mathcal{F}$, since
the map $\theta\colon \mathcal{F}' = \im' i \to \im i$ induces isomorphisms on
stalks, but since the domain already is a sheaf this forces $\theta$ to be an
isomorphism. Hence $\im i = \ker e$ as subsheaves of $\mathcal{F}$, so the sequence
is exact at $\mathcal{F}$.  Exactness at $\mathcal{F}'$ and
$\mathcal{F}/\mathcal{F}'$ is trivially checked on stalks using Exercise 6.
