We have a commutative diagram
\[\begin{tikzcd}[row sep=1cm, column sep=1cm]
	\mathcal{F'}\ar[d, "\hat{\imath}"]\ar[r, "i"] & \mathcal{F}\\
	\im' i\ar[r, "\theta"]\ar[ur] & \im i,\ar[u, "\iota"]
\end{tikzcd}\]
where the diagonal arrow is the inclusion, the bottom arrow is the natural map
into the associated sheaf, and the right arrow is induced by the diagonal arrow.
By Exercise 5, $\im i$ can be regarded as a subsheaf of $\mathcal{F}$.
Since $i$ is injective, for every $p \in X$, $i_p$ is injective, so by commutativity,
$\hat{\imath}_p$ is injective. Furthermore, for every $p \in X$, $\hat{\imath}$
is surjective, because it is surjective on open sets. Hence, the composite
$\theta \circ \hat{\imath}$ is an isomorphism on stalks. Since it is a map between
sheaves, this means that is is an isomorphism. Hence, $\mathcal{F}$ is isomorphic
to the subsheaf $\im i$.

Next, consider the diagram
\[\begin{tikzcd}[row sep=1cm, column sep=1cm]
	&\coker'\iota\ar[r, "\eta"]\ar[dr, "\hat{p}"]& \coker\iota\ar[d, "\hat{p}^+"]\\
	\im i\ar[r, "\iota"] &\mathcal{F}\ar[r, "p"]\ar[u, "\pi"] &\mathcal{F}'',
\end{tikzcd}\]
where the map $\hat{p}$ is defined on open sets using the fact that
$p \circ \iota = 0$, hence $p_U \circ \iota_U = 0$, hence $(\im i)(U) \subseteq (\ker p)(U)$.
The map, $\eta$ is the natural map into the associated sheaf and
$\hat{p}^+$ is obtained from the universal property. Since $p$ is surjective,
it is surjective on stalks, hence by commutatity $\hat{p}^+$ must also be surjective
on stalks.

TODO: show that $\hat{p}^+$ is injective on stalks.
