We have a commutative diagram
\[\begin{tikzcd}[row sep=1cm, column sep=1cm]
	\mathcal{F}\ar[d, "\hat{\imath}"]\ar[r, "i"] & \mathcal{F}'\\
	\im' i\ar[r, "\theta"]\ar[ur] & \im i,\ar[u]
\end{tikzcd}\]
where the diagonal arrow is the inclusion, the bottom arrow is the natural map
into the associated sheaf, and the right arrow is induced by the diagonal arrow.
By Exercise 5, $\im i$ can be regarded as a subsheaf of $\mathcal{F}$.
Since $i$ is injective, for every $p \in X$, $i_p$ is injective, so by commutativity,
$\hat{\imath}_p$ is injective. Furthermore, for every $p \in X$, $\hat{\imath}$
is surjective, because it is surjective on open sets. Hence, the composite
$\theta \circ \hat{\imath}$ is an isomorphism on stalks. Since it is a map between
sheaves, this means that is is an isomorphism. Hence, $\mathcal{F}$ is isomorphic
to the subsheaf $\im i$.
