Consider a scheme $X$ over $\Spec \mathbb{Z}$, for example
$X = \Proj \mathbb{Z}[X, Y, Z]/(X^n + Y^n - Z^n)\to \Spec \mathbb{Z}$.

We may consider the base change
\[ \Spec \mathbb{F}_p \to \Spec \mathbb{Z} \]
induced by $\mathbb{Z}\to \mathbb{Z}/p\mathbb{Z} \cong \mathbb{F}_p$. Here
we have
\[ X\times_{\Spec \mathbb{Z}} \Spec \mathbb{F}_p \cong \Proj \mathbb{F}_p[X, Y, Z]/(X^n + Y^n - Z^n). \]

Another possible base change is $\Spec \mathbb{Q}\to \Spec\mathbb{Z}$ induced by
$\mathbb{Z}\to \mathbb{Q}$. In this case we have
\[ X\times_{\Spec \mathbb{Z}}\Spec \mathbb{Q} \cong \Proj \mathbb{Q}[X, Y, Z]/(X^n, Y^n - Z^n). \]

Yet another example is given by $\Spec \mathbb{C}\to \Spec \mathbb{Z}$ induced by
$\mathbb{Z}\to \mathbb{C}$. This gives
\[ X\times_{\Spec \mathbb{Z}} \Spec \mathbb{C}\cong \Proj \mathbb{C}[X, Y, Z]/(X^n + Y^n - Z^n) \subseteq \mathbb{P}_{\mathbb{C}}^2. \]

This illustrates the power of scheme-theoretic language to translate questions
between different areas (e.g., number theory or geometry)
