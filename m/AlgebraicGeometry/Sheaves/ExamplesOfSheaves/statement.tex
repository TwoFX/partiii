\begin{enumerate}
	\item Let  $X$ be any topological space, $\mathcal{F}U$ the continuous functions $U\to \mathbb{R}$.

		This is a sheaf: $\rho_{UV}\colon \mathcal{F}U\to \mathcal{F}V$ is just the restriction.

		The first sheaf axiom says that a contiunous function is zero if it is zero on every open set
		of cover.

		The second sheaf axiom says that continuous functions can be glued.

	\item Let $X = \mathbb{C}$ with the Euclidean topology.

		Define $\mathcal{F}U$ to be the set of bounded analytic functions $f\colon U\to \mathbb{C}$.

		This is a presheaf, since the restriction of bounded analytic functions is
		bounded analytic. It also satisfies the first sheaf axiom. However, it does not
		satisfy the second sheaf axiom.

		For example, consider the cover $\set{U_i}_{i \in \mathbb{N}}$ of $\mathbb{C}$ given
		by $U_i = \set{z \in \mathbb{C}\given \abs{z}<i}$. Define $s_i\colon U_i\to \mathbb{C}$
		by $z\mapsto z$. Note that if $i < j$, then $U_i\cap U_j = U_i$ and
		$s_i|_{U_i\cap U_j} = s_j|_{U_i\cap U_j}$. However, gluing yields the identity
		function on $\mathbb{C}$, which is not bounded (note that complex analysis tells
		us that $\mathcal{F}\mathbb{C} = \mathbb{C}$.

		The underlying problem is that sheafs can only track properties that can be tested
		locally.

	\item Let $G$ be a group and set $\mathcal{F}U \coloneqq G$ for any open set $U$.
		This is called the constant presheaf. This is in general not a sheaf (unless $G$ is trivial).

		Take $U$ to be an disjoint union of open sets $U_1 \cup U_2$. If $\mathcal{F}U_1 = G$ and
		$\mathcal{F}U_2 = G$, then we need $\mathcal{F}(U_1\cap U_2) = 0$.

		If the second sheaf axiom was to be satisfied, we would want $s_1 \in \mathcal{F}U_1$ and
		$s_2 \in \mathcal{F}U_2$ to glue, so we should have $\mathcal{F}U = G\times G$.

		Now give $G$ the discrete topology, and define instead $\mathcal{F}U$ to be the set
		of continuous maps $f\colon U\to G$. By our choice of topology, this means that $f$
		is locally constant, i.e., for every $x \in U$ we have a neighborhood $V\subseteq U$ of $x$
		such that  $f|_V$ is constant.

		This is called the constant sheaf and if $U$ is nonempty and connected then
		$\mathcal{F}U = G$.

	\item If $X$ is an algebraic variety, $U \subseteq X$ a Zariski open subset, then
		define $\mathcal{O}_X(U)$ to be the regular functions $f\colon U\to k$.

		Roughly, $f$ regular means that every point of $U$ has an open neighborhood
		on which $f$ is expressed as a ratio of polynomials $g/h$ with $h$
		nonvanishing on the neighborhood.

		$\mathcal{O}_X$ is a sheaf, called the structure sheaf of $X$.
\end{enumerate}
