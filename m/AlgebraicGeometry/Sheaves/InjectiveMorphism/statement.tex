We say that $f$ is injective if $\ker f = 0$. We say that $f$ is surjective if $\im f = \mathcal{G}$.

Note that surjectivity does not imply that $f_U$ is surjective for every $U$.

We say that a sequence of morphisms of sheaves
\[\begin{tikzcd}
	\ldots \arrow[r] & \mathcal{F}^{i-1} \arrow[r, "f^i"] & \mathcal{F}^i \arrow[r, "f^{i+1}"] & \mathcal{F}^{i+1} \arrow[r] & \ldots
\end{tikzcd}\]

If $\mathcal{F}' \subseteq \mathcal{F}$ is a subsheaf, then we write $\mathcal{F}/\mathcal{F}'$ for the
sheaf associated to the presheaf $U\mapsto \mathcal{F}U/\mathcal{F}'U$, so $\mathcal{F}/\mathcal{F}'$ is
the cokernel of the inclusion $\mathcal{F}' \to \mathcal{F}$.
