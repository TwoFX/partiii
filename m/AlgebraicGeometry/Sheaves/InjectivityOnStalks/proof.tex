$f_p$ is injective for every $p$ if and only if $\ker f_p = 0$ for every $p$
if and only if $(\ker f)_p = 0$ for every $p$.

In the exercises, we show that for any sheaf $\mathcal{F}$, the map
\[ \mathcal{F}U \to \prod_{p \in U} \mathcal{F}_p \]
is injective. Now if all of the $\mathcal{F}_p$ are trivial, then so is
$\mathcal{F}U$.

Therefore $(\ker f)_p = 0$ for every $p$ if and only if $\ker f = 0$.

Similarly, $f_p$ is surjective for every $p$ iff $\im f_p = \mathcal{G}_p$ for every $p$.
Now consider the diagram
\[\begin{tikzcd}
	\im f_p\ar[r]\ar[d, "\cong"] & \mathcal{G}_p\\
	(\im' f)_p\ar[ur]\ar[r, "\cong"] & (\im f)_p,\ar[u]
\end{tikzcd}\]
where
\begin{itemize}
	\item the top arrow is the inclusion,
	\item the left arrow is the isomorphism defined in Lemma 1.9,
	\item the bottom arrow is the isomorphism on stalks induced by the inclusion into
		the associated sheaf,
	\item the diagonal arrow is the morphism on stalks induced by the inclusion of
		the presheaf image, and
	\item the right arrow is induced by the arrow making the sheaf image into a
		subsheaf.
\end{itemize}

The upper triangle commutes trivially, and the lower triangle commutes because
by construction the right arrow is induced by the unique arrow making the non-stalk
version of the triangle commute.
Thus, since the bottom and left arrows are isomorphisms and the diagram commutes,
we have that $\im f_p \to \mathcal{G}_p$ is an isomorphism (which just means that
$\im f_p = \mathcal{G}_p$ if and only if
$(\im f)_p \to \mathcal{G}_p$ is an isomorphism.

Now, the arrow $(\im f)_ \to \mathcal{G}_p$ is an isomorphism for every $p$ if
and only if $\im f \to \mathcal{G}$ is an isomorphism (Proposition 1.7), and this
is the definition of surjectivity.
