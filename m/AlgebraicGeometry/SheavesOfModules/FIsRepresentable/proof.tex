If the statement holds, then there is a \enquote{universal object}, i.e., an lement
of $F(\mathbb{P}^n_K)$ corresponding to the identity $1_{\mathbb{P}^n_k} \in h_{\mathbb{P}^n_k}(\mathbb{P}^n_k)$,
i.e., a surjective map $\varphi\colon\mathcal{O}_{\mathbb{P}^n}^{\oplus(n+1)}\to \mathcal{L}$. Further, following the proof of
Yoneda, given $f\colon X\to \mathbb{P}^n_k$ and $T\colon h_{\mathbb{P}^n} \to F$ the natural transformation
giving the natural isomorphism of functors, we get a commutative diagram
\[\begin{tikzcd}
	h_{\mathbb{P}^n}(\mathbb{P}^n)\ar[d, "h_{\mathbb{P}^n}(f)"]\ar[r, "T(\mathbb{P}^n)"] & F(\mathbb{P}^n)\ar[d, "F(f)"]\\
	h_{\mathbb{P}^n}(X)\ar[r, "T(X)"] & F(X)
\end{tikzcd}\]
By commutativity of the diagram, we find that $f^*\varphi\colon \mathcal{O}_X^{\oplus(n+1)}\to f^*\mathcal{L}$
is the same as $T(X)(f)$.

So the representing scheme $\mathbb{P}^n$ comes with the universal object
(surjective morphism) $\mathcal{O}_{\mathbb{P}^n}^{\otimes(n+1)}\to \mathcal{L}$.

Hence, our proof strategy will be to construct this universal object. The
line bundle we construct has a name: $\mathcal{O}_{\mathbb{P}^n}(1)$.

Recall that $\mathbb{P}^n$ has an open cover $\set{D_+(X_i)\given i\leq i\leq n}$,
where $S = k[X_0, \ldots, X_n]$, $\mathbb{P}_k^n = \Proj S$,
$D_+(X_i) = \set{\mathfrak{p} \in\Proj S\given X_i \in \mathfrak{p}}$.

We will take $U$ to be the trivializing cover for $\mathcal{O}_{\mathbb{P}^n}(1)$
with transition map $g_{ij} \in \mathcal{O}_{\mathbb{P}^n}^\times(D_+(X_i)\cap D_+(X_j)) = \mathcal{O}_{\mathbb{P}^n}^\times(D_+(X_iX_j)) \cong S_{(X_iX_j)}$
given by $g_{ij} = X_i/X_j = X_i^2/X_iX_j$. Note that $g_{ji} = X_j/X_i$, so $g_{ij}$
is invertible and the compatibility condition $g_{ij}\cdot g_{jk} = g_{ik}$ is
satisfied.

We have a morphism $\mathcal{O}_{\mathbb{P}^n}^{\oplus(n+1)}\to \mathcal{O}_{\mathbb{P}^n}(1)$
defined on $D_+(X_i)$ by $e_j\mapsto \frac{X_j}{X_i}$ using the trivialization of
$\mathcal{O}_{\mathbb{P}^n}(1)$ on $D_+(X_i)$, i.e., we have an isomorphism
$\mathcal{O}_{\mathbb{P}^n}(1)|_{U_i}\cong \mathcal{O}_{U_i}$.

We need to check that this is well-defined globally. We have maps
\begin{align*}
	\mathcal{O}_{\mathbb{P}^n}^{\oplus(n+1)}|_{D_+(X_iX_k)}&\to \mathcal{O}_{D_+(X_i)}|_{D_+(X_jX_k)},\\
	e_j&\mapsto X_j/X_i,\\
	\mathcal{O}_{\mathbb{P}^n}^{\oplus(n+1)}|_{D_+(X_iX_k)}&\to \mathcal{O}_{D_+(X_k)}|_{D_+(X_jX_k)},\\
	e_j &\mapsto X_j/X_k,\\
	\mathcal{O}_{D_+(X_i)}|_{D_+(X_jX_k)} &\stackrel{\cdot g_{ik}}{\longrightarrow} \mathcal{O}_{D_+(X_k)}|_{D_+(X_iX_k)}
\end{align*}

And the triangle they form commutes:
\[ g_{ik}\cdot \frac{X_j}{X_i} = \frac{X_i}{X_k}\cdot \frac{X_j}{X_i} = \frac{X_i}{X_j}. \]

Note in particular, that each $e_j$ maps to a global section of $\mathcal{O}_{\mathbb{P}^n}(1)$.
(Remark: if instead we have chosen $g_{ij} = X_j/X_i$ we would have obtained the line bundle
$\mathcal{O}_{\mathbb{P}^n}(1)^\vee \eqqcolon \mathcal{O}_{\mathbb{P}^n}(-1)$, and
$\Gamma(\mathbb{P}^n, \mathcal{O}_{\mathbb{P}^n}(-1)) = 0$).

We now have a morphism $\mathcal{O}_{\mathbb{P}^n}^{\oplus(n+1)} \to \mathcal{O}_{\mathbb{P}^n}(1)$,
and we need to check that it is surjective.  On $D_+(X_i)$, we have
$e_i\mapsto X_i/X_i = 1 \in\Gamma(D_+(X_i), \mathcal{O}_{\mathbb{P}^n})$, so in
particular, looking at sections over $D_+(X_i)$, we get a homomorphism of
$S_{(X_i)}$-modules $S_{(X_i)}^{\oplus(n+1)}\to S_{(X_i)}$ sending
$e_i\mapsto 1$, so it is clearly a surjective map of modules.

Thus, $(\psi\colon \mathcal{O}_{\mathbb{P}^n}^{\oplus(n+1)}\to \mathcal{O}_{\mathbb{P}^n}(1)) \in F(\mathbb{P}^n)$.

Ite remains to show that given $(X, \varphi\colon \mathcal{O}_X^{\oplus(n+1)}\to \mathcal{L})$,
there exists a unique morphism $f\colon X\to \mathbb{P}^n$ such that $\varphi\cong (f^*\psi\colon \mathcal{O}_X^{\oplus(n+1)}\to f^*\mathcal{O}_{\mathbb{P}^n}(1))$.
Indeed, this will give the natural transformation $F\to h_{\mathbb{P}^n}$ and the
inverse natural transformation $h_{\mathbb{P}^n}\to F$ is given by pullback,
$X\to \mathbb{P}^n$ gives $f^*\psi\colon \mathcal{O}_X^{\oplus(n+1)}\to f^*\mathcal{O}_{\mathbb{P}^n}(1)$.

To prove the existence of the unique morphism, let $\varphi(e_i) = s_i \in\Gamma(X, \mathcal{L})$.
Define $Z_i \coloneqq \set{x \in X\given (s_i)_x \in \mathfrak{m}_x \mathcal{L}_X}$,
where $(s_i)_x$ is the germ of $s_i$ at $x$ and $\mathfrak{m}_x \subseteq \mathcal{O}_{X, x}$
is the maximal ideal. We claim that $Z_i$ is a closed set.

Observe that being a closed set can be checked on an open cover $\set{U_i}$, since
$Z \subseteq X$ if and only if $Z\cap U_i$ is closed on $U_i$ for all $i$.

Thus we may use a trivializing affine cover $\set{U_i}$ of $X$, so we reduce to the
case that $X = \Spec A$, $\mathcal{L} \cong \mathcal{O}_{\Spec A}$, so $\Gamma(X, \mathcal{L})\cong A$,
so by abuse of notation we have $s_i \in A$. This induces $(s_i)_{\mathfrak{p}} = \frac{s_i}{1} \in A_{\mathfrak{p}}$.

Now $\frac{s_i}{1} \in \mathfrak{p}A_{\mathfrak{p}}$ if and only if $s_i$ lies in the
inverse image of $\mathfrak{m}_{\mathfrak{p}} A_{\mathfrak{p}}$ under the localization
map $A\to A_{\mathfrak{p}}$. Hence, $Z_i = V(s_i)$ is closed.

Let $U_i\coloneqq X|_{Z_i}$ Then there is an isomorphism $\mathcal{O}_{U_i} \to \mathcal{L}|_{U_i}$
sencind $1\mapsto s_i$, and the inverse sends $s\mapsto s/s_j$. Injectivity is obvious,
surjectivity is by definition of $Z_i$.

(Remark: If we were working in the world of varieties, locally the section $s_i$ is
viewed as a function and $Z_i$ is the locus where $s_i$ vanishes. On $U_i$, we will
define a morphism to projective space $U_i\to D_+(X_i) \subseteq \mathbb{P}^n$
via $p\mapsto (\frac{s_o(p)}{s_i(p)}, \ldots, \frac{s_n(p)}{s_i(p)})$. Equivalently,
on $X$, we can view this function as $X\to \mathbb{P}^n$, $p\mapsto (s_0(p), \ldots, s_n(p))$.)
