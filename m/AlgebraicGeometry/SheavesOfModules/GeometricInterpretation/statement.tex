One may define the notion of a vector bundle over a scheme $X$ as another
scheme $E$ with a morphism $\pi\colon E\to X$ whose fibres are $\mathbb{A}^r$,
and there is an open cover $\set{U_i}$ such that $\pi^{-1}(U_i)\cong U_i\times \mathbb{A}^r$
(and some other conditions).

We get a sheaf $\mathcal{E}(U) = \set{s\colon U\to \pi^{-1}(U)\given \pi \circ s = \id_U}$.
This gives a locally free sheaf on $X$. For details, see the exercises of Hartshorne, II.5.

As an example let $E = X\times \mathbb{A}^1$, then $\mathcal{E}(U) = \mathcal{O}_X(U)$.
Giving a morphism $s\colon U\to U\times_{\Spec k} \mathbb{A}^1_k$ whose composition with
$p_1\colon U\times \mathbb{A}^1_k\to U$ is the identity is the same as giving a
morphism $U\to \mathbb{A}^1_k$:
\[\begin{tikzcd}
	U\ar[drr, "f", bend left]\ar[ddr, "\id_U", bend right]\ar[dr, densely dotted, "s"]\\
	& U\times_{\Spec k} \mathbb{A}^1_k\ar[r]\ar[d] & \mathbb{A}^1_k\ar[d]\\
	& U\ar[r] &\Spec k
\end{tikzcd}\]
Hence giving a morphism $f\colon U\to \mathbb{A}_k^1$ is the same thing as giving
a homomorphism of $k$-algebras $k[X]\to \mathcal{O}_X(U)$ mapping $X\mapsto \varphi$
for some $\varphi \in \mathcal{O}_X(U)$. Hence the set of such homomorphisms is
completely determined by the elements of $\mathcal{O}_X(U)$.
