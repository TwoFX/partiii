The pushforward is not as well-behaved, e.g., $f_*\mathcal{L}'$ for $\mathcal{L}'$ a line
bundle on $Y$ need not be a line bundle. In fact, it will always be quasi-coherent, but not
necessarily coherent.

If $\mathcal{L}_1, \mathcal{L}_2$ are line bundles on $X$, with a common trivializing
cover $\set{U_i}$ (this is always possible) and with transition functions
$g_{ij}, h_{ij}$ respectively, then:
\begin{enumerate}
	\item The transition functions of $\mathcal{L}_1\tensor_{\mathcal{O}_X} \mathcal{L}_2$
		are $g_{ij}h_{ij}$. Note that if $(\cdot g)\colon A\to A$ and  $(\cot h)\colon A\to A$
		are given, then these two homomorphisms induce the homomorphism
		$(\cdot g)\tensor (\cdot h)\colon A\tensor_A A\to A\tensor_A A$, but
		$A\tensor_A A\cong A$, and applying this isomorphism on both sides yields
		the map $A\to A$ given by $(\cdot (g\cdot h))$.
	\item Set $\mathcal{L}_1^\vee\cong \SheafHom_{\mathcal{O}_X}(\mathcal{L}_1, \mathcal{O}_X)$.
		This is also a line bundle, because on $U_i$, $\mathcal{L}_1|_{U_i}\cong \mathcal{O}_{U_i}$,
		and $\SheafHom_{\mathcal{O}_{U_i}}(\mathcal{O}_{U_i}, \mathcal{O}_{U_i}) = \mathcal{O}_{U_i}$,
		noting that $\Hom_A(A, A)\cong A$.

		The transition maps are given by $g_{ij}^{-1}$:
		\[\begin{tikzcd}
			\mathcal{O}_{U_i}|_{U_i\cap U_j}\ar[r, "\cdot g_{ij}"] & \mathcal{O}_{U_j}|_{U_i\cap U_j},
		\end{tikzcd}\]
		and we need to take the transpose inverse of this map represented by a
		$1\times 1$ matrix.

		Note that $\mathcal{L}_1^\vee\tensor_{\mathcal{O}_X}\mathcal{L}_1$ has transition
		maps given by $g_{ij}^{-1}g_{ij} = 1$. Thus $\mathcal{L}_1^\vee \tensor_{\mathcal{O}_X} \mathcal{L}_1\cong \mathcal{O}_X$.
\end{enumerate}
