Let $(X, \mathcal{O}_X)$ be a ringed space. A sheaf of  $\mathcal{O}_X$-modules
is a sheaf of abelian groups on $X$ such that for each $U \subseteq X$, $\mathcal{F}(U)$
has the structure of an $\mathcal{O}_X(U)$-module, compatible with restriction,
i.e., if $s \in \mathcal{O}_X(U)$, $m \in F(U)$ and $V \subseteq U$, then
$s|_V\cdot m|_V = (s\cdot m)|_V$.

A morphism of sheaves of $\mathcal{O}_X$-modules $\varphi\colon \mathcal{F}\to \mathcal{G}$
is a morphism of sheaves of abelian groups such that for all $U \subseteq X$
the map $\varphi_U\colon \mathcal{F}(U)\to \mathcal{G}(U)$ is $\mathcal{O}_X(U)$-linear.

Kernels, cokernals and images of morphisms of sheaves of $\mathcal{O}_X$-modules
are sheaves of $\mathcal{O}_X$-modules (TODO: show that sheafification retains the
$\mathcal{O}_X$-module structure).

We write $\Hom_{\mathcal{O}_X}(\mathcal{F}, \mathcal{G})$ for the group of
$\mathcal{O}_X$-module homomorphisms from $\mathcal{F}$ to $\mathcal{G}$.
This is an $\mathcal{O}_X(X)$-module.

Then $U\mapsto \Hom_{\mathcal{O}_U}(\mathcal{F}|_U, \mathcal{G}|_U)$ is a sheaf
of $\mathcal{O}_X$-modules, written as $\SheafHom(\mathcal{F}, \mathcal{G})$
and pronounced \enquote{sheaf hom}.

If $\mathcal{F}$, $\mathcal{G}$ are $\mathcal{O}_X$-modules, we denote by
$\mathcal{F}\tensor_{\mathcal{O}_X} \mathcal{G}$ with the trivial Lie bracket.

the sheaf associated to the
presheaf $U\mapsto \mathcal{F}(U) \tensor_{\mathcal{O}_X(U)} \mathcal{G}(U)$.

Pushforwards and pullbacks: We will start with some motivation.
Consider a map of rings $\varphi A\to B$ and let $M$ be a
$B$-module and $N$ and $A$-module. Then $M$ is also an $A$-module via restriction
of scalars. Also, $B\tensor_A N$ is a $B$-module via extension of scalars. We
want to find analogues of this for sheaves of $\mathcal{O}_X$-modules.

Given $f\colon X\to Y$ a morphism of ringed spaces (recall that this includes
a map $f^\#\colon \mathcal{O}_Y \to f_*\mathcal{O}_X$), a sheaf $\mathcal{F}$ of
$\mathcal{O}_X$ modules and a sheaf $\mathcal{G}$ of $\mathcal{O}_Y$-modules,
then
\begin{enumerate}
	\item $f_*\mathcal{F}$ is naturally a sheaf of $f_*\mathcal{O}_X$-modules since
		$(f_*\mathcal{F})(U)  = \mathcal{F}(f^{-1}(U))$, which is a
		 $\mathcal{O}_X(f^{-1}(U)) = (f_*\mathcal{O}_X)(U)$-module and hence
		 $f_*\mathcal{F}$ becomes a $\mathcal{O}_Y$-module via restriction of
		 scalars.
	\item $f^{-1}\mathcal{G}$ is naturally an $f^{-1}\mathcal{O}_Y$-module. But
		$f^\#$ induces the adjoint map $f^\#\colon f^{-1}\mathcal{O}_Y\to \mathcal{O}_X$
		by Question 10 on the first example sheet.
		Define $f^*\mathcal{G}\coloneqq f^{-1}\mathcal{G}\tensor_{f^{-1}\mathcal{O}_Y}\mathcal{O}_X$.
		This is a sheaf of $\mathcal{O}_X$-modules.
\end{enumerate}
