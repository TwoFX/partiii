\documentclass[a4paper]{amsbook}

\usepackage[utf8]{inputenc}
\usepackage[T1]{fontenc}

\usepackage{lmodern}
\usepackage{microtype}
\usepackage{enumitem}

\usepackage[english]{babel}

\usepackage{mathtools}
\usepackage{amsfonts, amssymb}
\usepackage{amsthm}
\usepackage{tikz-cd}

\usepackage{csquotes}
\usepackage[hidelinks]{hyperref}

\title{Algebraic Geometry}
\author{Mark Gross}

\theoremstyle{definition}
\newtheorem{definition}{Definition}[chapter]
\newtheorem{theorem}[definition]{Theorem}
\newtheorem{lemma}[definition]{Lemma}
\newtheorem{remark}[definition]{Remark}
\newtheorem{example}[definition]{Example}
\newtheorem{corollary}[definition]{Corollary}
\newtheorem{exercise}[definition]{Exercise}
\newtheorem{proposition}[definition]{Proposition}
\newtheorem{notation}[definition]{Notation}
\newtheorem*{definition*}{Definition}
\newtheorem*{theorem*}{Theorem}
\newtheorem*{lemma*}{Lemma}
\newtheorem*{remark*}{Remark}
\newtheorem*{example*}{Example}
\newtheorem*{corollary*}{Corollary}
\newtheorem*{exercise*}{Exercise}
\newtheorem*{proposition*}{Proposition}
\newtheorem*{notation*}{Notation}

\DeclarePairedDelimiterX\set[1]\lbrace\rbrace{\def\given{\;\delimsize\vert\;}#1}
\DeclareMathOperator\Spec{Spec}
\DeclareMathOperator\Proj{Proj}
\DeclareMathOperator\id{id}
\DeclareMathOperator\coker{coker}
\DeclareMathOperator\im{im}
\DeclarePairedDelimiter\abs\lvert\rvert


\begin{document}
\maketitle

These notes, taken by Markus Himmel, will at times differ significantly from
what was lectured. In particular, all errors are almost certainly my own.

\tableofcontents

\chapter*{Introduction}
\label{Introduction}
\begin{definition}
\label{Spectrum}
Let $A $ be a ring. Then $\Spec A\coloneqq \set{p \subseteq A\given p\text{ a prime ideal}}$.

For $I \subseteq A$ an ideal, define
\[ V(I)\coloneqq \set{p \subseteq A\given p\text{ prime}, p\supseteq I}. \]
\end{definition}

\begin{proposition}
\label{ZariskiTopology}
The sets $V(I)$ form the closed sets of a topology on $\Spec A$, called the
Zariski topology.
\end{proposition}
\begin{proof}[Proof]
\begin{enumerate}
	\item $V(A) = \varnothing$
	\item $V(0) = \Spec A$
	\item If $\set{I_i}_{i \in J}$ is a collection of ideals, then
		$V(\sum_{i \in J} I_j) = \bigcap V(I_i)$.
	\item We claim: $V(I_1 \cap I_2) = V(I_1) \cup V(I_2)$.

		\enquote{$\supseteq$} is obvious.

		\enquote{$\subseteq$}: Follows from the fact that $p\supseteq I_1\cap I_2$ is prime, then
		$p\supseteq I_1$ or $p\supseteq I_2$.
\end{enumerate}
\end{proof}

\begin{example}
\label{PolynomialRing}
Let $A = k[X_1, \ldots, X_n]$ with $k$ algebraically closed. Let $I \subseteq A$
be an ideal. Then the maximal ideals $m$ of  $A$ containing $I$ are in one-to-one
correspondence with $V(I)$ in $\mathbb{A}^n(k)$: by Nulstellensatz, every
maximal ideal is of the form $(X_1 - a_1, \ldots, X_n - a_n)$, which corresponds
to $(a_1, \ldots, a_n)$ in the old $V(I)$.

The new $V(I)$ now extends this notion of zero set by including other prime
ideals.
\end{example}

\begin{example}
\label{Field}
If $k$ is a field, then $\Spec k = \set{0}$, so the topological space cannot
see the field. We fix this by also thinking about what functions are on
these spaces.
\end{example}


\chapter{Sheaves}
\label{Sheaves}
\begin{remark*}
\label{FixSpace}
Fix a topological space $X$.
\end{remark*}

\begin{definition}
\label{Presheaf}
A presheaf $\mathcal{F}$ on $X$ consists of
\begin{enumerate}
	\item For every open set $U \subseteq X$ an abelian group $\mathcal{F}U$,
	\item for every inclusion $V \subseteq U \subseteq X$ a restriction map
		$\rho_{UV}\colon \mathcal{F}U \to \mathcal{F}V$ such that $\rho_{UU} = \id_{\mathcal{F}U}$ and
		$p_{UW} = \rho_{VW} \circ \rho_{UV}$.
\end{enumerate}
\end{definition}

\begin{remark}
\label{PresheafCategorically}
A presheaf is just a contravariant functor from the poset category of open sets
of $X$ to the catgegory of abelian groups.

We can generalize this to any contravariant functor $X^{\text{op}}\to \mathcal{C}$ for
some category $\mathcal{C}$.
\end{remark}

\begin{definition}
\label{MorphismOfPresheaves}
A morphism of presheaves $f\colon \mathcal{F}\to \mathcal{G}$ on $X$ is a collection of morphisms
$f_U\colon \mathcal{F}U\to \mathcal{G}U$ such that for all $V \subseteq U$ the diagram
\[
	\begin{tikzcd}
		\mathcal{F}U\ar[r, "f_U"]\ar[d, "\rho_{UV}"] & \mathcal{G}U\ar[d, "\rho_{UV}"]\\
		\mathcal{F}V\ar[r, "f_V"] & \mathcal{G}V
	\end{tikzcd}
	\]
commutes.
\end{definition}

\begin{definition}
\label{Sheaf}
A presheaf $\mathcal{F}$ is called a sheaf if it satisfies additional axioms:
\begin{enumerate}[label=(S\arabic*)]
	\item If  $U \subseteq X$ is covered by an open cover $\set{U_i}$ and $s \in \mathcal{F}U$ satsfies
		$s|_{U_i}\coloneqq\rho_{UU_i}(s) = 0$ for all $i$, then $s = 0$
	\item If $U$, and $U_i$ are as before, and if $s_i \in \mathcal{F}U_i$ such that
		for all $i$ and $j$ we have $s_i|_{U_i\cap U_j} = s_j|_{U_i \cap U_j}$, then
		there is some $s \in \mathcal{F}U$ such that $s|_{U_i} = s_i$ for all $i$.
\end{enumerate}
\end{definition}

\begin{remark}
\label{SheafRemark}
\begin{enumerate}
	\item If $\mathcal{F}$ is a sheaf, then $\varnothing \subseteq X$ is covered
		by the empty covering; hence $\mathcal{F}(\varnothing) = 0$.
	\item The two sheaf axioms can be described as saying that given $U, \set{U_i}$,
		\[
			\begin{tikzcd}
				0\ar[r] & \mathcal{F}U\ar[r, "\alpha"] & \prod_i \mathcal{F}U_i \ar[r, shift left, "\beta_1"]\ar[r, shift right, "\beta_2" below] & \prod_{i, j} \mathcal{F}(U_i\cap U_j)
			\end{tikzcd}
			\]
			is exact, where $\alpha(s) = (s|_{U_i})_{i \in I}$, $\beta_1((s_i)_{i \in I}) = (s_i|_{U_i \cap U_j})$,
			$\beta_2((s_i)_{i \in I}) = (s_j|_{U_i\cap U_j})_{i, j}$.

			Exactness means that $\alpha$ is injective, $\beta_1\circ\alpha = \beta_2\circ\alpha$,
			and for any $(s_i) \in \prod_{i \in I}\mathcal{F}U_i$, with $\beta_1((s_i)) = \beta_2((s_i))$,
			there exists $s \in \mathcal{F}U$ with $\alpha(s) = (s_i)$.

			This is all subsumed by saying that $\alpha$ is the equalizer of $\beta_1$ and $\beta_2$.
\end{enumerate}
\end{remark}

\begin{example*}
\label{ExamplesOfSheaves}
\begin{enumerate}
	\item Let  $X$ be any topological space, $\mathcal{F}U$ the continuous functions $U\to \mathbb{R}$.

		This is a sheaf: $\rho_{UV}\colon \mathcal{F}U\to \mathcal{F}V$ is just the restriction.

		The first sheaf axiom says that a contiunous function is zero if it is zero on every open set
		of cover.

		The second sheaf axiom says that continuous functions can be glued.

	\item Let $X = \mathbb{C}$ with the Euclidean topology.

		Define $\mathcal{F}U$ to be the set of bounded analytic functions $f\colon U\to \mathbb{C}$.

		This is a presheaf, since the restriction of bounded analytic functions is
		bounded analytic. It also satisfies the first sheaf axiom. However, it does not
		satisfy the second sheaf axiom.

		For example, consider the cover $\set{U_i}_{i \in \mathbb{N}}$ of $\mathbb{C}$ given
		by $U_i = \set{z \in \mathbb{C}\given \abs{z}<i}$. Define $s_i\colon U_i\to \mathbb{C}$
		by $z\mapsto z$. Note that if $i < j$, then $U_i\cap U_j = U_i$ and
		$s_i|_{U_i\cap U_j} = s_j|_{U_i\cap U_j}$. However, gluing yields the identity
		function on $\mathbb{C}$, which is not bounded (note that complex analysis tells
		us that $\mathcal{F}\mathbb{C} = \mathbb{C}$.

		The underlying problem is that sheafs can only track properties that can be tested
		locally.

	\item Let $G$ be a group and set $\mathcal{F}U \coloneqq G$ for any open set $U$.
		This is called the constant presheaf. This is in general not a sheaf (unless $G$ is trivial).

		Take $U$ to be an disjoint union of open sets $U_1 \cup U_2$. If $\mathcal{F}U_1 = G$ and
		$\mathcal{F}U_2 = G$, then we need $\mathcal{F}(U_1\cap U_2) = 0$.

		If the second sheaf axiom was to be satisfied, we would want $s_1 \in \mathcal{F}U_1$ and
		$s_2 \in \mathcal{F}U_2$ to glue, so we should have $\mathcal{F}U = G\times G$.

		Now give $G$ the discrete topology, and define instead $\mathcal{F}U$ to be the set
		of continuous maps $f\colon U\to G$. By our choice of topology, this means that $f$
		is locally constant, i.e., for every $x \in U$ we have a neighborhood $V\subseteq U$ of $x$
		such that  $f|_V$ is constant.

		This is called the constant sheaf and if $U$ is nonempty and connected then
		$\mathcal{F}U = G$.

	\item If $X$ is an algebraic variety, $U \subseteq X$ a Zariski open subset, then
		define $\mathcal{O}_X(U)$ to be the regular functions $f\colon U\to k$.

		Roughly, $f$ regular means that every point of $U$ has an open neighborhood
		on which $f$ is expressed as a ratio of polynomials $g/h$ with $h$
		nonvanishing on the neighborhood.

		$\mathcal{O}_X$ is a sheaf, called the structure sheaf of $X$.
\end{enumerate}
\end{example*}

\begin{definition}
\label{Stalk}
Let $\mathcal{F}$ be a presheaf on $X$ and  let $x \in X$. Then the stalk of
$\mathcal{F}$ at $x$ is $\mathcal{F}_x\coloneqq\set{(U, s)\given U \subseteq X~\text{open neighborhood at $x$}, s \in \mathcal{F}U}/\sim$,
w here $(U, s)\sim (V, s')$ if there is a neighborhood $W \subseteq U\cap V$ of $x$ such that
$s|_W = s'|_W$. An equivalence class of a pair $(U, s)$ is called a germ.
\end{definition}

\begin{remark*}
\label{SheafCategorically}
$\mathcal{F}_x$ is just the colimit of $\mathcal{F}U$ where $U$ ranges over the
open neighborhoods of $x$.

Note that a morphism $f\colon \mathcal{F}\to \mathcal{G}$ of presheaves induces a morphism
$f_p\colon \mathcal{F}_p\to \mathcal{G}_p$ via $f_p(U, s)\coloneqq (U, f_U(s))$.
\end{remark*}

\begin{proposition}
\label{SheafIso}
Let $f\colon \mathcal{F}\to \mathcal{G}$ be a morphism of sheaves. Then
$f$ is an isomorphism if and only if $f_p$ is an isomorphism for every $p \in X$.
\end{proposition}
\begin{proof}[Proof]
\enquote{$\Longrightarrow$} is obvious.

\enquote{$\Longleftarrow$}: Assume that $f_p$ is an isomorphism for all
$p \in X$. Need to show that $f_U\colon \mathcal{F}U \to \mathcal{G}U$ is an
isomorphism for all $U \subseteq X$, as then we can define $(f^{-1})_U = (f_U)^{-1}$.
This defines a morphism of sheaves, as
\begin{align*}
	\rho_{UV}^{\mathcal{F}}\circ f_U^{-1} &= f_V^{-1}\circ f_V \circ \rho_{UV}^{\mathcal{F}} \circ f_U^{-1}\\
	&= f_V^{-1} \circ \rho_{UV}^{\mathcal{G}}\circ f_U \circ f_U^{-1}\\
	&= f_V^{-1} \circ \rho_{UV}^{\mathcal{G}}.
\end{align*}

We will first check that $f_U$ is injective. Suppose $s \in \mathcal{F}U$ and
$f_U(s) = 0$. Then for all $p \in U$, we have $f_p(U, s) = (U, f_U(s)) = (U, 0) = 0 \in \mathcal{G}_p$.
Since $f_p$ is injective, this means that $(U, s) = 0$ in $\mathcal{F}_p$. This
means that there is an open neighborhood $V_p$ of $p$ in $U$ such that
$s|_{V_p} = 0$. Since the sets $\set{V_p}_{p \in U}$ cover $U$, we see by
sheaf axiom 1 that we have $s = 0$.

Next, we will show that $f_U$ is surjective. Let $t \in \mathcal{G}U$ and write
$t_p\coloneqq (U, t) \in \mathcal{G}_p$. Since $f_p$ is surjective, we find $s_p \in \mathcal{F}_p$
with $f_p(s_p) = t_p$. This means that we find an open neighborhood $V_p \subseteq U$ of $p$
and a germ $(V_p, s_p)$ such that $(V_p, f_{V_p}(s_p)) \sim (U, t)$. By
shrinking $V_p$ if necessary wen can assume that $t|_{V_p} = f_{V_p}(s_p)$.

Now on $V_p\cap V_q$, $f_{V_p\cap V_q}(s_p|_{V_p\cap V_q} - s_q|_{V_p\cap V_q}) = t_{V_p\cap V_q} - t|_{V_p\cap V_q} = 0$
and hence by injectivity of $f_{V_p\cap V_q}$ already proved, we have $s_p|_{V_p\cap V_q} = s_q|_{V_p\cap V_q}$.
By the second sheaf axiom, the $s_p$ glue to give an element $s \in \mathcal{F}U$ with
$s|_{V_p} = s_p$ for every $p \in U$.

Now $f_U(s)|_{V_p} = f_{V_p}(s|_{V_p}) = f_{V_p}(s_p) = t|_{V_p}$.
By the first sheaf axiom applied to $f_U(s) - t$ we get $f_U(s) = t$. This shows
surjectivity of $f_U$, completing the proof.
\end{proof}

\begin{theorem}
\label{Sheafification}
Given a presheaf $\mathcal{F}$ there is a sheaf $\mathcal{F}^+$ and a morphism
$\theta\colon \mathcal{F}\to \mathcal{F}^+$ satisfying the following
universal propert:

For any sheaf $\mathcal{G}$ and morphism $\varphi\colon \mathcal{F}\to \mathcal{G}$
there is a unique morphism $\varphi^+\colon \mathcal{F}^+\to \mathcal{G}$ such
that $\varphi^+ \circ\theta = \varphi$.

The pair $(\mathcal{F}^+, \theta)$ is unique up to unique isomorphism and is called
the sheafification of $\mathcal{F}$.
\end{theorem}
\begin{proof}[Proof]
See exercises.
\end{proof}

\begin{definition*}
\label{Kernel}
Let $f\colon \mathcal{F} \to \mathcal{G}$ be a morphism of presheaves on a space
$X$. We define
\begin{enumerate}
	\item The presheaf kernel of $f$, $\ker f$, is the presheaf given by
		\[ (\ker f)(U)\coloneqq \ker f_U. \]
		One should check that this is a presheaf.

	\item The presheaf cokernel of $f$, $\coker f$, is the presheaf given by
		\[ (\coker f)(U)\coloneqq \coker f_U. \]

	\item The presheaf image $\im f$ is the presheaf given by
		\[ (\im f)(U) = \im f_U. \]
\end{enumerate}
\end{definition*}

\begin{remark*}
\label{KernelForSheaf}
If $f\colon \mathcal{F}\to \mathcal{G}$ is a morphism of sheaves, then $\ker f$ is
also a sheaf. The identity axiom is certainly satisfied: If $s \in (\ker f)(U) \subseteq \mathcal{F}U$
satisfies $s|_{U_i} = 0$ for all $U_i$ in a cover of $U$, then we use the identity axiom
for $\mathcal{F}$ to find that $s = 0$.

Given $s_i \in (\ker f)(U_i)$ with $\set{U_i}$ an open cover of $U$, and with
$s_i|_{U_i\cap U_j} = s_j|_{U_i\cap U_j}$, then we find $s \in \mathcal{F}U$ with
$s|_{U_i} = s_i$. But $f_U(s) = 0$ since
\[ f_U(s)|_{U_i} = f_{U_i}(s|_{U_i}) = f_{U_i}(s_i) = 0, \]
and we can use the identity axiom to conclude that $f_U(s) = 0$.
\end{remark*}

\begin{example*}
\label{ProjectiveExample}
Let $X = \mathbb{P}^1$ (or think of the Riemann sphere). Let $P, Q \in X$ be distinct
points. Let $\mathcal{G}$ be the sheaf of regular functions on $X$ (alternatively, think
of holomorphic functions on the Riemann sphere). Next, let $\mathcal{F}$ be the sheaf
of regular functions which vanish on $P$ and $Q$. Notice that
$\mathcal{F}U = \mathcal{G}U$ if $U\cap \set{P, Q} = \varnothing$.

Let $U \coloneqq \mathbb{P}^1\setminus\set{P}$, $V = \mathbb{P}^1\setminus\set{Q}$.

Note that $\mathcal{F}(\mathbb{P}^1) = 0$, $\mathcal{G}(\mathbb{P}^1) = k$,
because regular functions on $\mathbb{P}^1$ are constants. Let
$f\colon \mathcal{F}\to \mathcal{G}$ be the inclusion.

Then $(\coker f)(\mathbb{P}^1) \cong k$, $(\coker f)(U) = \mathcal{G}U/\mathcal{F}U = k[X]/(X) \cong k$a,
$(\coker f)(V) \cong k$. However, $(\coker f)(U\cap V) = \mathcal{G}(U\cap V)/\mathcal{F}(U\cap V) \cong 0$.

Therefore, if the gluing axiom held, then we could need to have
\[ (\coker f)(\mathbb{P}^1) \cong k\oplus k. \]

Note that this failure to be a sheaf is not a bug, but a feature!
\end{example*}

\begin{definition*}
\label{SheafKernel}
Let $f\colon \mathcal{F}\to \mathcal{G}$ be a morphism of sheaves. The sheaf kernel of $f$
is just the presheaf kernel.

The sheaf cokernel is the sheaf associated to the presheaf cokernel of $f$.

The sheaf image is the sheaf associated to the presheaf image of $f$.

We can check that these notions give kernels, cokernels and images in the category
of sheaves.
\end{definition*}

\begin{exercise*}
\label{SheafImageExercise}
The sheaf image $\im f$ is a subsheaf of $\mathcal{G}$, where $\mathcal{F}$ is called a
subsheaf of $\mathcal{G}$ if we have a morphism $f\colon \mathcal{F}\to \mathcal{G}$
such that $f_U$ is a monomorphism for every open set $U$.
\end{exercise*}
\begin{proof}[Solution]
See exercises.
\end{proof}

\begin{definition*}
\label{InjectiveMorphism}
We say that $f$ is injective if $\ker f = 0$. We say that $f$ is surjective if $\im f = \mathcal{G}$.

Note that surjectivity does not imply that $f_U$ is surjective for every $U$.

We say that a sequence of morphisms of sheaves
\[\begin{tikzcd}
	\ldots \arrow[r] & \mathcal{F}^{i-1} \arrow[r, "f^i"] & \mathcal{F}^i \arrow[r, "f^{i+1}"] & \mathcal{F}^{i+1} \arrow[r] & \ldots
\end{tikzcd}\]
is exact if $\ker f^{i+1} = \im f^i$ for all $i$.

If $\mathcal{F}' \subseteq \mathcal{F}$ is a subsheaf, then we write $\mathcal{F}/\mathcal{F}'$ for the
sheaf associated to the presheaf $U\mapsto \mathcal{F}U/\mathcal{F}'U$, so $\mathcal{F}/\mathcal{F}'$ is
the cokernel of the inclusion $\mathcal{F}' \to \mathcal{F}$.
\end{definition*}

\begin{lemma}
\label{KernelsCokernelsStalks}
Let $f\colon \mathcal{F}\to \mathcal{G}$ be a morphism of sheaves. Then
for all $p \in X$ we have
\begin{align*}
	(\ker f)_p &= \ker(f_p\colon \mathcal{F}_p\to \mathcal{G}_p)\\
	(\im f)_p &= \im f_p
\end{align*}
\end{lemma}
\begin{proof}[Proof]
We first define a map $(\ker f)_p \to \ker f_p$. If $(U, s) \in (\ker f)_p$, then
$(U, s) \in \mathcal{F}_p$ and
\[ f_p(U, s) = (U, f_U(s)) = (U, 0) = 0 \in \mathcal{G}_p. \]
Therefore, $(U, s) \in \ker f_p$.

We will check injectivity and surjectivity of this map.

For injectivity, assume that $(U, s) = 0$ in $\mathcal{F}_p$, then there is
$V \subseteq U$ of $p$ such that $s|_V = 0$. Then we also have the equality
\[ (U, s) = (V, s|_V) = (V, 0) = 0 \]
in  $(\ker f)_p$.

For surjectivity, assume that $(U, s) \in \ker f_p$. This means that
 $(U, f_U(s)) = 0$ in $\mathcal{G}_p$, so there is $p \in V \subseteq U$ such
 that $0 = f_U(s)|_V = f_V(s|_V)$. Thus, $s|_V \in (\ker f)(V)$, and
 $(V, s|_V) \in (\ker f)_p$, and $(V, s|_V)$ maps to the element in $\ker f_p$
 represented by $(U, s)$.

For images: Let $\im' f$ be the presheaf image.

From the exercises we know that $\theta_p\colon \mathcal{F}_p \to \mathcal{F}^+_p$ is
an isomorphism for every $p$.

Therefore $(\im f)_p \cong (\im' f)_p$, so we need to show that
$(\im' f)_p \cong \im f_p$. Define a map
$(\im' f)_p \to \im f_p$ by
\[ (U, s)  \in (\im' f)_p \mapsto (U, s) \in \im f_p. \]

Once again, we will check that this is injective and surjective.

For injectivity: if $(U, s) = 0$ in $\mathcal{G}_p$ then there is a neighborhood
$V \subseteq U$ of $p$ such that $s|_V = 0$. Then $(U, s) = (V, 0)$ in $(\im' f)_p$.

For surjectivity: if $(U, s) \in \im f_p$, then there is
$(V, t) \in \mathcal{F}_p$ with $(V, f_V(t)) = f_p(V, t) = (U, s)$, so after
shrinking $U$ and $V$ if necessary, then we can take $U = V$ and
$f_U(t) = s$. Then $(U, s) \in (\im' f)_p$.
\end{proof}

\begin{proposition*}
\label{InjectivityOnStalks}
Let $f\colon \mathcal{F}\to \mathcal{G}$ be a morphism of sheaves. Then
$f$ is injective if and only if for every $p \in X$ the map $f_p\colon \mathcal{F}_p \to \mathcal{G}_p$ is injective
and $f$ is surjective if and only if for every $p \in X$ the map $f_p\colon \mathcal{F}_p\to \mathcal{G}_p$ is surjective.
\end{proposition*}
\begin{proof}[Proof]
$f_p$ is injective for every $p$ if and only if $\ker f_p = 0$ for every $p$
if and only if $(\ker f)_p = 0$ for every $p$.

In the exercises, we show that for any sheaf $\mathcal{F}$, the map
\[ \mathcal{F}U \to \prod_{p \in U} \mathcal{F}_p \]
is injective. Now if all of the $\mathcal{F}_p$ are trivial, then so is
$\mathcal{F}U$.

Therefore $(\ker f)_p = 0$ for every $p$ if and only if $\ker f = 0$.

Similarly, $f_p$ is surjective for every $p$ iff $\im f_p = \mathcal{G}_p$ for every $p$.
Now consider the diagram
\[\begin{tikzcd}
	\im f_p\ar[r]\ar[d, "\cong"] & \mathcal{G}_p\\
	(\im' f)_p\ar[ur]\ar[r, "\cong"] & (\im f)_p,\ar[u]
\end{tikzcd}\]
where
\begin{itemize}
	\item the top arrow is the inclusion,
	\item the left arrow is the isomorphism defined in Lemma 1.9,
	\item the bottom arrow is the isomorphism on stalks induced by the inclusion into
		the associated sheaf,
	\item the diagonal arrow is the morphism on stalks induced by the inclusion of
		the presheaf image, and
	\item the right arrow is induced by the arrow making the sheaf image into a
		subsheaf.
\end{itemize}

The upper triangle commutes trivially, and the lower triangle commutes because
by construction the right arrow is induced by the unique arrow making the non-stalk
version of the triangle commute.
Thus, since the bottom and left arrows are isomorphisms and the diagram commutes,
we have that $\im f_p \to \mathcal{G}_p$ is an isomorphism (which just means that
$\im f_p = \mathcal{G}_p$) if and only if
$(\im f)_p \to \mathcal{G}_p$ is an isomorphism.

Now, the arrow $(\im f)_p \to \mathcal{G}_p$ is an isomorphism for every $p$ if
and only if $\im f \to \mathcal{G}$ is an isomorphism (Proposition 1.7), and this
is the definition of surjectivity.
\end{proof}

\begin{exercise*}
\label{ImCoker}
Given $f\colon \mathcal{F}\to \mathcal{G}$, then we have $\mathcal{G}/\im f\cong \coker f$.
\end{exercise*}

\section{Passing between spaces}
\label{PassingBetweenSpaces}
\begin{remark*}
\label{Intro}
Let $f\colon X\to G$ be a continuous map between topological spaces, $\mathcal{F}$ a sheaf on
$X$, $\mathcal{G}$ a sheaf on $Y$.
\end{remark*}

\begin{definition*}
\label{PushforwardSheaf}
Define $f_*\mathcal{F}$ by setting
\[ (f_*\mathcal{F})\coloneqq \mathcal{F}(f^{-1}(U)) \]
for $U \subseteq Y$ open.
\end{definition*}

\begin{exercise*}
\label{PushforwardSheafVer}
$f_*\mathcal{F}$ is a sheaf on $Y$.
\end{exercise*}
\begin{proof}[Solution]
TODO
\end{proof}

\begin{definition*}
\label{PullbackSheaf}
Define  $f^{-1}\mathcal{G}$ to be the sheaf associated to the presheaf
\[ U\mapsto \set{(V, s)\given f(U) \subseteq V, s \in \mathcal{G}V}/\sim \]
where $(V, s)\sim (v', s')$ if there is some $W \subseteq V\cap V'$ such that
$f(U) \subseteq W$ and $s|_W = s'|_W$.
\end{definition*}

\begin{example*}
\label{Examples}
If $f\colon \set{p} \to X$ is an inclusion of a point, then $f^{-1}\mathcal{G}$ is the
sheaf on the one-point space given by $\mathcal{G}_p$.

More generally, if $i\colon Z\to X$ is the inclusion of a subspace, we often write
$\mathcal{F}|_Z\coloneqq i^{-1}\mathcal{F}$. If $Z$ is open in $X$, then we have
\[ F|_Z(U) \cong \mathcal{F}U. \]
\end{example*}

\begin{notation}
\label{Sections}
If $s \in \mathcal{F}U$, then we say that $s$ is a section of $\mathcal{F}$ over $U$.

We often write $\mathcal{F}U = \Gamma(U, \mathcal{F})$. This allows us to think
of $\Gamma(U, \cdot)$ as a functor from the category of presheaves on $X$ to the
category of abelian groups.
\end{notation}



\chapter{Affine schemes}
\label{AffineSchemes}
\begin{remark*}
\label{Goal}
Our goal is to construct a sheaf $\mathcal{O}$ on $\Spec A$, analogous to the sheaf of
regular functions on a variety.

$\mathcal{O}$ will be a sheaf of rings, i.e., $\mathcal{O}U$ will be a ring for each
open set $U$ and restriction maps will be ring homomorphisms.
\end{remark*}

\begin{remark*}
\label{Localization}
Let $A$ be a ring and let $S \subseteq A$ be a multiplicative subset (i.e., $1 \in S$ and
$S$ is closed under multiplication). We define a ring
\[ S^{-1}A = \set{(a, s)\given a \in A, s \in S}/\sim, \]
where $(a, s) \sim (a', s')$ iff there is $s'' \in S$ such that
$s''(as'-a's) = 0$.

We write $a/s$ for the equivalence class of $(a, s)$.

Observe that the usual equivalence relation on fractions suggests that we should
have $a/s = a'/s' \iff as' = a's$. We need the extra possibility of killing
$as'-a's$ with $s''$ if $A$ is not an integral domain.

The ring $S^{-1}A$ is called the localization of $A$ at $S$.
\end{remark*}

\begin{example*}
\label{LocalizationExample}
\begin{enumerate}
	\item Take $f \in A$ and $S\coloneqq \set{f^n\given n \in \mathbb{N}_0}$. Then we write
		$A_f\coloneqq S^{-1}A$.

		This example will correspond to open subsets.
	\item Let $p \subseteq A$ be a prime ideal of $A$. Then $S\coloneqq A\setminus p$
		satisfies $1 \in S$ and is closed under multiplication since  $p$ is
		prime. We define $A_p\coloneqq S^{-1}A$. This is the localization of $A$
		at (or rather, away from?) $p$.

		This example will correspond to taking stalks.
\end{enumerate}
\end{example*}

\begin{definition*}
\label{StructureSheaf}
$\mathcal{O}$ should satisfy $\mathcal{O}_p = A_p$.

Define
\[ \mathcal{O}U \coloneqq\set{s\colon U\to\coprod_{p \in U} A_p\given (\star)}, \]
where $(\star)$ means that
\begin{enumerate}
	\item $\forall p \in U\colon s(p) \in A_p$,
	\item for each $p \in U$ there is some $p \in V \subseteq U$ with $V$ open and
		$a, f \in A$ such that for all $q \in V\colon f\notin q\wedge s(q) = a/f$.
\end{enumerate}
\end{definition*}

\begin{lemma*}
\label{StructureSheafStalk}
For any $p \in \Spec A$, we have $\mathcal{O}_p\cong A_p$.
\end{lemma*}
\begin{proof}[Proof]
We define a map
\begin{align*}
	\mathcal{O}_p&\to A_p\\
	(U, s)&\mapsto s(p)
\end{align*}
and will show that it is injective and surjective.

For surjectivity, notice that every element of $A_p$ can be written as
$a/f$ for some $a \in A$, $f \notin p$. Then
\[ D(f)\coloneqq \Spec A\setminus V(f) = \set{p \in \Spec A\given f\notin p} \]
is an open set (in fact it is called a standard open). Now $a/f$ defines an
element of $s \in \mathcal{O}(D(f))$ given by $q\mapsto a/f \in A_q$.
In particular, $s(p) = a/f \in A_p$.

For injectivity, let $p \in U \subseteq \Spec A$, $s \in \mathcal{O}U$ with $s(p) = 0$ in
$A_p$. We need to show that $(U, s) = 0$ in $\mathcal{O}_p$. By shrinking $U$
we can assume that $s$ is given by $a, f \in A$ with $s(q) = a/f$ for all $q \in U$.
In particular $f\notin q$ for every $q \in U$.

Thus, $a/f = 0/1$ in $A_p$. By definition of localization, this means that there
is $h \in A\setminus p$ such that  $h\cdot(a\cdot 1 + f\cdot 0) = 0$ in $A$,
so we have $ah = 0$.

Now let $V = D(f)\cap D(h)$. Then $(V, s|_V) = 0$ in $\mathcal{O}_p$, since
for $q \in V$, $s|_v(q) = s(q) = a/f \in A_q$ and $ha = 0$, $h \notin A\setminus q$,
so $ha = 0$ implies $a/f = 0/1$ in $A_q$. Thus $(U, s) = 0$ in $\mathcal{O}_p$.
\end{proof}

\begin{lemma*}
\label{StructureSheafStandardOpen}
For any $f \in A$, we have $\mathcal{O}(D(f)) \cong A_f$.

In particular, since $\Spec A = D(1)$, we have $\mathcal{O}(\Spec A) \cong A_1 \cong A$.
\end{lemma*}
\begin{proof}[Proof]
Define
\begin{align*}
	\Psi\colon A_f&\to \mathcal{O}(D(f))\\
	a/f^n&\mapsto (p\mapsto a/f^n).
\end{align*}
This makes sense since if $f\notin p$, then $f^n\notin p$.
As usual, we will verify injectivity and surjectivity.

For injectivity, assume that $\Psi(a/f^n) = 0$. Then for all $p \in D(f)$, we have
$a/f^n = 0$ in $A_p$, i.e., there is  $h \in A\setminus p$ such that $ha = 0$ in $A$.

Let $I = \set{q \in A\given q\cdot a = 0}$ (the annihilator of $a$).
So $h \in I$, but $h \notin p$, so $I\nsubseteq p$.
This is true for all $p \in D(f)$, so $V(I)\cap D(f) = \varnothing$. Thus
$f \in \bigcap_{p \in V(I)}p = \sqrt{I}$, as we know from commutative algebra.
This means that $f^n \in I$ for some $n>0$. Thus $f^n\cdot a = 0$, so
$a/f^n = 0$ in $A_f$, so $\Psi$ is injective.

Next, we will prove surjectivity. Let $s \in \mathcal{O}(D(f))$. Cover
$D(f)$ with open sets $V_i$ on which $s$ is represented by as $a_i/g_i$ with
$a_i, g_i \in A$, $g_i \notin p$ whenever $p \in V_i$. Thus $V_i \subseteq D(g_i)$.
By question 1 on the first example sheet, the sets of the form $D(h)$ form a
base for the Zariski topology on $\Spec A$. Thus we can assume $V_i = D(h_i)$
for some $h_i \in A$. Since $D(h_i) \subseteq D(g_i)$, we have
$V(h_i) \supseteq V(g_i)$, so $\sqrt{(h_i)} \subseteq \sqrt{(g_i)}$, since the
radical is the intersection of all the primes of $V(\cdot)$. Hence,
$h_i^n \in (g_i)$ for some $n$, say $h_i^n = c_ig_i$, so we have
$\frac{a_i}{g_i} = \frac{c_ia_i}{h_i^n}$. Now replace $h_i$ by $h_i^n$. This does
not change the open sets because in general $D(h_i) = D(h_i^n)$ and replace
$a_i$ by $c_ia_i$.

The situation so far is that we may assume that $D(f)$ is covered by sets
$D(h_i)$ such that $s$ is represented by $a_i/h_i$ on $D(h_i)$.

We now claim that $D(f)$ can be covered by a finite number of the  $D(h_i)$, i.e.,
$D(f)$ is quasicompact. Indeed, $D(f) \subseteq \bigcup_i D(h_i)$, which is equivalent
to $V(f) \supseteq \bigcap_i V(h_i) = V(\sum (h_i))$. This in turn is equivalent to
$f \in \sqrt{\sum_i (h_i)}$ (because it just says that $f$ is in every prime ideal
containing $\sum_i (h_i)$), which is equivalent to there being some $n$ such that
$f^n \in \sum_i (h_i)$. Hence, we can write $f^n = \sum_{i \in I} b_i h_i$ for some finite
set $I$.

Reversing this argument yields that $D(f) \subseteq \bigcup_{i \in I} D(h_i)$ as
required, completing the proof of the claim.

We now pass to this finite subcover $\set{D(h_i)}_{i \in I}$. On
$D(h_i)\cap D(h_j) = D(h_ih_j)$, note $a_i/h_i$ and $a_j/h_j$ both represent $s$. Since we
have already shown injectivity, this means that $a_ih_j/h_ih_j = a_jh_i/h_ih_j$ in $A_{h_ih_j}$.

Thus, for some $n$, $(h_ih_j)^n(h_ja_i - h_ia_j) = 0$ in $A$. We can pick an $n$
sufficiently large to work for all pairs $i, j$ (since there are only finitely many
such pairs).

We rewrite this equality as $h_j^{n+1}(h_i^na_i) - h_i^{n+1}(h_j^na_j) = 0$.
Now replace $h_i$ by $h_i^{n+1}$, and $a_i$ by $h_i^na_i$ (this is allowed because
$\frac{a_i}{h_i}= \frac{a_ih_i^n}{h_i^{n+1}}$. Thus we can assume that $s$ is still
represented on $D(h_i)$ by $a_i/h_i$ but also for each $i, j$ we have $h_ia_j = h_ja_i$.

Since $D(f) \subseteq \bigcup_{i \in I} D(h_i)$, we have
$V(\sum (h_i)) = \bigcap_{i \in I} V(h_i) \subseteq V(f)$, hence
$f^n = \sum b_ih_i$ for some $h_i$. Define $a\coloneqq b_i a_i$.

Then for any $j$, we have \[ h_ja = \sum_i b_ia_ih_j = \sum_i b_ia_jh_i = f^na_j. \]
This means that $a/f^n = a_j/h_j$ on $D(h_j)$. Hence $\Psi(a/f^n) = s$, completing
the proof of surjectivity.
\end{proof}

\begin{remark*}
\label{WhereAreWe}
We now have a topological space $\Spec A$ equipped with a sheaf of rings
$\mathcal{O}$.
\end{remark*}

\begin{definition*}
\label{RingedSpace}
A ringed space is a pair $(X, \mathcal{O}_X)$ where $X$ is a topological space
and $\mathcal{O}_X$ is a sheaf of rings on $X$.

A morphism of ringed spaces $f\colon (X, \mathcal{O}_X) \to (Y, \mathcal{O}_Y)$
consists of a continuous map $X\to Y$ and a morphism of shaves of rings
$f^\#\colon \mathcal{O}_Y\to f_*\mathcal{O}_X$, i.e., for every open $O \subseteq Y$,
a homomorphism of rings $f^\#_U\colon \mathcal{O}_Y(U)\to \mathcal{O}_X(f^{-1}(U))$.
\end{definition*}

\begin{example*}
\label{RingedSpaceExamples}
\begin{enumerate}
	\item Let $X, Y$ be topological spaces and $\mathcal{O}_X$ and $\mathcal{O}_Y$
		the sheaf of continuous $\mathbb{R}$-valued functions. Given $f\colon X\to Y$,
		we get $f^\#\colon \mathcal{O}_Y\to f_*\mathcal{O}_X$ defined by
		$f_U^\#(\varphi) = \varphi \circ f$.
	\item Let $X$ be a variety and $\mathcal{O}_X$ the sheaf of regular functions
		on $X$. A morphism of varieties $f\colon X\to Y$ is a continuous map
		inducing
		\begin{align*}
			\mathcal{O}_Y(U) &\to \mathcal{O}_X(f^{-1}(U)),\\
			\varphi&\mapsto \varphi \circ f.
		\end{align*}
\end{enumerate}
\end{example*}

\begin{definition*}
\label{LocallyRingedSpace}
A locally ringed space $(X, \mathcal{O}_X)$ is a ringed space such that
$\mathcal{O}_{X, p}$ is a local ring (i.e., has a unique maximal
ideal) for every $p \in X$.

A morphism $f\colon (X, \mathcal{O}_X)\to (Y, \mathcal{O}_Y)$ of locally ringed spaces is a morphism of ringed spaces such that
such that the induced map $f_p^\#\colon \mathcal{O}_{Y, f(p)} \to \mathcal{O}_{X, p}$ is a local
homomorphism. Here,
\begin{itemize}
	\item the map $f_p^\#$ is defined by $(U, s)\mapsto (f^{-1}(U), f^\#_U(s))$ for
		a section $s \in \mathcal{O}_Y(U)$, and
	\item a local homomorphism $\varphi\colon (A, m_A)\to (B, m_B)$ is a ring homomorphism between local rings such that
		$\varphi^{-1}(m_B) = m_A$. Note that $\varphi(A\setminus m_A) = \varphi(A^\times) \subseteq B^\times = B\setminus m_B$.
		Hence, $\varphi^{-1}(m_B) \subseteq m_A$ is always true, and the opposite
		inclusion is what makes a ring homomorphism local.
\end{itemize}
\end{definition*}

\begin{remark*}
\label{LocallyRingedSpaceIntuition}
In the case of varieties, $\mathcal{O}_{X, p}$ has a unique maximal ideal
$\set{(U, f) \in \mathcal{O}_X(U)\given f(p) = 0}/\sim$, i.e., if $f(p)\neq 0$,
then $f$ is nowhere vanishing on some neighborhood of $p$, so after shrinking
$U$, we can invert $f$.

The local homomorphism condition just follows from the pullback of a function
$\varphi$ vanishing at $f(p)$ vanishes at $p$.
\end{remark*}

\begin{example*}
\label{SpecIsLRS}
$(\Spec A, \mathcal{O})$ is a locally ringed space; which we call an affine scheme.
\end{example*}

\begin{theorem*}
\label{CategoryOfAffineSchemes}
The category of affine schemes with locally ringed morphisms is equivalent
to the opposite of the category of rings.
\end{theorem*}
\begin{proof}[Proof]
We need to show the following things.
\begin{enumerate}
	\item If $\varphi\colon A\to B$ is a ring homomorphism, we obtain an
		induced morphism
		\[ (f, f^\#)\colon (\Spec B, \mathcal{O}_B)\to (\Spec A, \mathcal{O}_A). \]
	\item Any morphism of affine schemes as locally ringed spaces arises in
		this way.
\end{enumerate}

For the first part, let $\varphi\colon A\to B$ be a ring homomorphism and define
\begin{align*}
	f\colon \Spec B&\to \Spec A\\
	p&\mapsto \varphi^{-1}(p),
\end{align*}
where we use that $\varphi^{-1}(p)$ is prime: if $ab \in \varphi^{-1}(p)$, then
$\varphi(ab) = \varphi(a)\varphi(b) \in p$. Hence $\varphi(a) \in p$ or
$\varphi(b) \in p$, hence $a \in \varphi^{-1}(p)$ or $b \in \varphi^{-1}(p)$.

We also need to show that $f$ is continuous. Any closed set is of the form
$V(I)$. We calculate
\begin{align*}
	f^{-1}(V(I)) &= f^{-1}(\set{p \in \Spec A\given p\supseteq I})\\
	&= \set{q \in \Spec B\given f(q) \supseteq I}\\
	&= \set{q \in \Spec B\given \varphi^{-1}(q)\supseteq I}\\
	&= \set{q \in \Spec B\given q\supseteq \varphi(I)}\\
	&= V(\varphi(I)).
\end{align*}
Hence the preimage of a closed set is closed, so $f$ is continuous.

We need to construct a morphism of sheaves
\[ f_\#\colon \mathcal{O}_{\Spec A} \to f_*\mathcal{O}_{\Spec B}. \]
For $p \in \Spec B$, we obtain a natural homomorphism
\begin{align*}
	\varphi_p\colon A_{\varphi^{-1}(p)}&\to B_p\\
	\frac{a}{s}&\mapsto \frac{\varphi(a)}{\varphi(s)},
\end{align*}
where $a \in A$, $s\notin \varphi^{-1}(p)$. This makes sense since $\varphi(a) \in B$
and $\varphi(s)\notin p$.

The maximal ideal $pB_p$ of $B_p$ is generated by the image of $p$ under the map
$B\to B_p$. The maximal ideal $\varphi^{-1}(p)A_{\varphi^{-1}(p)}$ of $A_{\varphi^{-1}(p)}$ is generated by the image
of $\varphi^{-1}(p)$ under the map $A\to A_p$.

We have a commutative diagram
\[\begin{tikzcd}
	A\ar[r, "\varphi"]\ar[d] & B\ar[d]\\
	A_{\varphi^{-1}(p)}\ar[r, "\varphi_p"] & B_p.
\end{tikzcd}\]
Thus $\varphi_p^{-1}(pB_p) = \varphi^{-1}(p)A_{\varphi^{-1}(p)}$.
Given $V \subseteq \Spec A$ open, we may define
\begin{align*}
	f^\#_V\colon \mathcal{O}_{\Spec A}(V)&\to \mathcal{O}_{\Spec B}(f^{-1}(V))\\
	s&\mapsto (q\mapsto \varphi_q(s(f(q)))).
\end{align*}
We now have to check the local coherence condition of $\mathcal{O}$, i.e.,
if $s$ is locally given by $a/h$, then $f^\#_V(s)$ is locally given by
$\frac{\varphi(a)}{\varphi(h)}$ (this is obvious, but should be checked carefully).

This gives the desired map $f^\#\colon \mathcal{O}_{\Spec A}\to f_*\mathcal{O}_{\Spec B}$
and the induced map on stalks $f^\#_p\colon O_{\Spec A, f(p)}\to \mathcal{O}_{\Spec B, p}$
agrees with $\varphi_p\colon A_{\varphi^{-1}(p)}\to B_p$ by construction (this should
be checked carefully). Hence,
the pair $(f, f^\#)$ is a morphism of locally ringed spaces.

Now suppose given a morphism $(f, f^\#)\colon \Spec B\to \Spec A$ of locally
ringed spaces. We have
\[ f^\#_{\Spec A}\colon \Gamma(\Spec A, \mathcal{O}_{\Spec A}) \to \Gamma(\Spec B, \mathcal{O}_{\Spec B}), \]
but since the glocal sections of $\Spec R$ are just $R$, we get $\varphi\colon A\to B$.

We need to show that $\varphi$ gives rise to $(f, f^\#)$. We have a local homomorphism
\[ f_p^\#\colon A_{f(p)} \cong \mathcal{O}_{\Spec A, f(p)} \to \mathcal{O}_{\Spec B, p}\cong B_p. \]
This is compatible with the corresponding map on glocal sections in the sense that
\[\begin{tikzcd}
	\Gamma(\Spec A, \mathcal{O}_{\Spec A})\ar[d]\ar[r, "f^\#_{\Spec A}"] & \Gamma(\Spec B, \mathcal{O}_{\Spec B})\ar[d]\\
	\mathcal{O}_{\Spec A, f(p)}\ar[r, "f_p^\#"] & \mathcal{O}_{\Spec B, p}
\end{tikzcd}\]
is a commutative diagram. By applying our calculations, this yields a diagram
\[\begin{tikzcd}
	A\ar[r, "\varphi"]\ar[d] & B\ar[d]\\
	A_{f(p)}\ar[r, "f_p^\#"] & B_p.
\end{tikzcd}\]
Recall that $f_p^\#$ is a local homomorphism. Thus $(f_p^\#)^{-1}(pB_p) = f(p)A_{f(p)}	$.
Along the lower left path, the maximal ideal $pB_p$ is pulled back to $f(p)A_{f(p)}$ and
then to $f(p)$. Along the upper right path, it gets pulled back to $p$ and then to $\varphi^{-1}(p)$.
By commutativity, we conclude that $f(p) = \varphi^{-1}(p)$.

Thus $f$ is induced by $\varphi$ and by commutativity, $f_p^\# = \varphi_p$.
Then $f^\#$ is as constructed previously (this needs to be checked).
\end{proof}

\begin{remark*}
\label{Antiequivalence}
Note that demanding that $(f, f^\#)$ is a morphism of locally ringed spaces rather
than merely ringed spaces was crucial to make the proof work.
\end{remark*}

\begin{definition}
\label{Scheme}
An affine scheme is a locally ringed space that is isomorphic as a locally
ringed space to $(\Spec A, \mathcal{O}_{\Spec A})$ for some ring $A$.

A scheme is a locally ringed space $(X, \mathcal{O}_X)$ with an open cover
$\set{(U_i, \mathcal{O}_X|_{U_i})}$ such that each $(U_i, \mathcal{O}_X|_{U_i})$
is an affine scheme.  Recall that we have $\mathcal{O}_X|_{U_i}(V) = \mathcal{O}_X(V)$
for $V \subseteq U$ open.
\end{definition}

\begin{example*}
\label{ExamplesOfSchemes}
\begin{enumerate}
	\item Let $k$ be a field. Then $\Spec k = (\set{0}, k)$.

		What does giving a morphism $f\colon \Spec k\to X$ a scheme mean?

		First, we need to choose a point $x \in X$, the image of $f$. Second,
		we get a local ring homomorphism
		\[ f^\#_0\colon \mathcal{O}_{X, x} \to \mathcal{O}_{\Spec k, 0} \cong k, \]
		i.e., $(f_0^\#)^{-1}(0) = \mathfrak{m}_x \subseteq \mathcal{O}_{X, x}$, the maximal
		ideal of $\mathcal{O}_{X, x}$. Thus we get a factorization
		$f^\#_0\colon \mathcal{O}_{X, x}\to \mathcal{O}_{X, x}/\mathfrak{m}_x \to k$.
		The middle quotient is a field denoted as $\kappa(x)$, the residue field
		of $X$ at $x$.

		Thus $f$ induces an inclusion $\kappa(x) \to k$.

		Conversely, given an inclusion $\iota\colon\kappa(x)\to k$ we get a morphism of
		schemes $\Spec k\to X$ by defining $f(0) = x$ and
		$f^\#\colon \mathcal{O}_X\to f_*k$ by defining $s\mapsto \iota(s(x)) \in k$,
		where $s(x)$ means taking the stalk of $s$ at $x$.

		Moral: Giving a morphism $f\colon \Spec k\to X$ is equivalent to giving
		a point $x \in X$ and an inclusion $\kappa(x)\to k$.

		Note: If $X = \Spec A$, then giving a morphism $\Spec k \to \Spec A$ is
		equivalent to giving a homomorphism $A\to k$, which we viewed at the
		beginning of the course as a \enquote{$k$-valued point} on $\Spec A$.
	\item What does giving a morphism $X\to \Spec k$ mean? The continuous map
		$X \to \Spec k$ does not carry any information, since $\Spec k$ is
		a singleton space. We also have a map
		\[ f^\#\colon k \cong \mathcal{O}_{\Spec k}\to f_*\mathcal{O}_X, \]
		i.e., a map $k\to \Gamma(\Spec k, f_*\mathcal{O}_X) = \Gamma(X, \mathcal{O}_X)$,
		i.e., $\Gamma(X, \mathcal{O}_X)$ carries a $k$-algebra structure.

		Notice that this induces $k$-algebra structures on $\mathcal{O}_X(U)$
		for all open sets $U$ via the composite
		\[ k \longrightarrow \mathcal{O}_X(X) \stackrel{\rho}\longrightarrow \mathcal{O}_X(U), \]
		and similarly all stalks $\mathcal{O}_{X, p}$ are also $k$-algebras.

		In this situation we say that $X$ is a scheme (defined) over $k$.
	\item Consider $A = k[X_1, \ldots, X_n]/I$ with $I = \sqrt{I}$. Then
		$\Spec A$ is a replacement for $V(I) \subseteq \mathbb{A}^n_k$, viewing
		$\Spec A$ as a scheme over $k$.

		If $k \subseteq k'$ is a field extension, a $k'$-valued point of $X/k$ is
		a commutative diagram
		\[\begin{tikzcd}
			\Spec k' \ar[r]\ar[dr] & X\ar[d]\\
			& \Spec k,
		\end{tikzcd}\]
		which has the dual
		\[\begin{tikzcd}
			k' & A\ar[l]\\
			& k\ar[u]\ar[ul],
		\end{tikzcd}\]
		so the top arrow is a homomorphism of  $k$-algebras.

		We write $X(k')$ for the set of such morphisms.
\end{enumerate}
\end{example*}

\begin{remark*}
\label{RelativePointOfView}
It is rare in algebraic geometry to work with schemes alone. Rather, we always work
over a base scheme.

Fix a base scheme $S$. Define the category of schemes over $S$ to be the
category whose objects are morphisms $T\to S$ and morphisms are commutative
triangles. This is just the normal comma construction.

We will frequently work with schemes over $\Spec k$, which we will also refer to
as schemes over $k$.

Given two schemes over $S$, $T\to S$ and $X\to S$, we define a $T$-valued point
of $X\to S$ as a morphism $T\to X$ over $S$. We write $X(T)$ for the set of
$T$-valued points.

By Yoneda, the collection of $X(T)$ for every $T$ determines $X$ up to isomorphism.
\end{remark*}

\begin{example*}
\label{AnotherExample}
Fix a field $k$, and let $D = \Spec k[t]/(t^2) = (\set{(t)}, k[t]/(t^2))$, where $(t)$ is
the unique prime ideal.
$t$ doesn't make sense as a $k$-valued function any more, as $t^2 = 0$.

Let $X$ be any scheme over $k$. What is $X(D)$? Given a morphism $f\colon D\to X$
of schemes over $k$, we get a point $x \in X$ as the image of $f$ and a local
homomorphism
\[ f^\#_x\colon \mathcal{O}_{X, x}\to k[t]/(t^2), \]
such that $(f^\#_x)^{-1}((t)) = \mathfrak{m}_x$. Note thate $\mathfrak{m}_x^2$ maps
to $0$, hence we get a $k$-linear map
\[ \mathfrak{m}_x/\mathfrak{m}_x^2 \to (t)\cong k, \]
where the isomorphism is as a $k$-vector space. We also have a composed $k$-algebra
homomorphism
\[ \mathcal{O}_{X, x} \to k[t]/(t^2)\to k[t]/(t) \cong k \]
with kernel $\mathfrak{m}_x$, and hence we have $\kappa(x) = \mathcal{O}_{X, x}/\mathfrak{m}_x \cong k$.
To see this, we must use that this is a homomorphism of $k$-algebras, so the $k$ sitting inside
$\mathcal{O}_{X, x}$ maps to $k$ on the right, i.e., the composite is surjective.

Se we get:
\begin{enumerate}
	\item a $k$-valued point with residue field $k$,
	\item a morphism of $k$-vector spaces $\mathfrak{m}_x/\mathfrak{m}_x^2\to k$,
		i.e., an element of $(\mathfrak{m}_x/\mathfrak{m}_x^2)^*$, the dual vector space.
\end{enumerate}

The space $(\mathfrak{m}_x/\mathfrak{m}_x^2)^*$ is called the Zariski tangent space
to $X$ at $x$. It can be thought of as a kind of \enquote{differentiation rule}.

Think of $D$ as a point plus an arrow: mapping $D$ into a scheme $X$ carries as
data a point of $X$ and a tangent vector\footnote{This is just a vague intuition.}.
\end{example*}

\begin{example*}[Glued Schemes]
\label{GluedSchemes}
This is a special case of a question of Example Sheet 1.

Suppose we are given to schemes $X_1$, $X_2$ and open subsets $U_i \subseteq X_i$.

Recall $U_i$ is also a locally ringed space $(U_i, \mathcal{O}_{X_i}|_{U_i})$ and
in fact $U_i$ is then a scheme (this is not obvious and will be discussed later).

Given an isomorphism $f\colon U_1\to U_2$, we can glue $X_1$ and $X_2$ along
$U_1$ and $U_2$ to get a scheme $X$ with an open cover $\set{X_1, X_2}$.

As a topological space, $X$ is just the topological gluing of $X_1$ and $X_2$.
Refer to the example sheet for the construction of $\mathcal{O}_X$.

Now take $\mathbb{A}^n_k \coloneqq \Spec k[X_1, \ldots, X_n]$. Hence,
$\mathbb{A}^1_k = \Spec k[X]$. Take $X_1 = X_2 = \mathbb{A}^1_k$. Glue
$U_1 \coloneqq \mathbb{A}^1\setminus\set{0} = D(X) \subseteq A_k^1 = X$, where $0$ is the point
corresponding to the prime ideal $(X)$ and
$U_2\coloneqq \mathbb{A}^1\setminus\set{0} = D(X) \subseteq X_2$ via the
identity map. As a topological space, $X$ is just the line with two origins.
The resulting scheme is called the affine line with doubled origin. It is not
a variety.

Note that $U_i = \Spec k[X]_X$ (localization). Hence, we could also glue
$U_1$ and $U_2$ via the map given by $X\mapsto X^{-1}$.

When we glue this way, we get the projective line over $k$, $\mathbb{P}^1_k$.
\end{example*}


\chapter{Projective schemes}
\label{ProjectiveSchemes}
\begin{remark*}
\label{Conventions}
Let $S$ be a graded ring, i.e., $S = \bigoplus_{d\geq 0} S_d$ with $S_d$ an abelian
group, and we have the product law $S_d\cdot S_{d'} \subseteq S_{d+d'}$.

For example, if $S = k[X_0, \ldots, X_n]$, then we get a grading such that $S_d$
is the space of homogenous polynomials of degree $d$.

We write $S_+ \coloneqq \bigoplus_{d\geq 1}$, which we call the irrelevant ideal.
\end{remark*}

\begin{definition*}
\label{HomogeneousIdeal}
$I \subseteq S$ is called a homogeneous ideal if $I$ is generated by its
homogeneous elements, i.e., elements in $S_d$ for various $d$.
\end{definition*}

\begin{definition*}
\label{Proj}
$\Proj S\coloneqq \set{p \in \Spec S\given p\text{ homogeneous}, p\nsupseteq S_+}$.

For $I \subseteq S$ a homogeneous ideal, set $V(I)\coloneqq \set{p \in \Proj S\given p\supseteq I}$.
\end{definition*}

\begin{exercise*}
\label{ProjectiveZariski}
Check the $V(I)$ form the closed sets of a topology on $\Proj S$.
\end{exercise*}
\begin{proof}[Solution]
TODO.
\end{proof}

\begin{remark}
\label{HomogenousNotation}
For $p \in \Proj S$, let
\[ T = \set{f \in S\setminus p\given \text{$f$ is homogeneous}}. \]
Then $T$ is a multiplicateively closed subset of $S$ and let $S_{(p)} \subseteq T^{-1}S$
be the subring of elements of degree $0$, i.e., written in the form $s/s'$ with
$s \in S$ homogeneous, $s' \in T$ with $\deg s = \deg s'$.

For $f \in S$ homogeneous, we write $S_(f) \subseteq S_f$ for the subset of
elements of degree $0$.
\end{remark}

\begin{definition*}
\label{ProjectiveStructureSheaf}
For $U \subseteq \Proj S$ open, set
\[ \mathcal{O}(U)\coloneqq \set{s\colon U\to \coprod_{p \in U} S_(p)\given (\star)}, \]
where ($\star$) means that
\begin{enumerate}
	\item for all  $p \in U$, $s(p) \in S_(p)$, and
	\item for all $p \in U$ there is $p \in V \subseteq U$ and $a, f \in S$ homogeneous
		of the same degree such that for all $q \in V$ we have $f \notin q$
		and $\forall q \in V\colon s(q) = a/f$.
\end{enumerate}

As before, we can calculate $\mathcal{O}_p \cong S_{(p)}$.

Important question: is the locally ringed space $(\Proj S, \mathcal{O})$ a
scheme?

If $f \in S$ is homogeneous, then we write $D_+(f) = \set{p \in \Proj S\given f\notin p}$.
This is the open set, as we have $D_+(f) = \Proj S\setminus V(f)$.
\end{definition*}

\begin{proposition*}
\label{ProjectiveIsScheme}
We have $(D_+(f), \mathcal{O}|_{D_+(f)}) \cong \Spec S_{(f)}$ as locally ringed
spaces. Further, the open sets $D_+(f)$ for $f \in S_+$ cover $\Proj S$. Hence
$(\Proj S, \mathcal{O})$ is a scheme.
\end{proposition*}
\begin{proof}[Proof]
This appears on the second example sheet.
\end{proof}

\begin{definition}
\label{ProjectiveSpace}
If $A$ is a ring, define
\[ \mathbb{P}^n_A\coloneqq \Proj A[X_0, \ldots, X_n]. \]
\end{definition}

\begin{example*}
\label{TwoDimensionalProjective}
Let $k$ be an algebraically closed field, consider $\mathbb{P}_k^1 = \Proj k[X_0, X_1]$.

The closed points, i.e., points $p$ such that $\set{p}$ is closed, correspond to
maximal elements of $\Proj S$ (TODO: exercise!). These maximal elements are ideals
of the form $(aX_0 - bX_1)$:

Note that the only maximal homogeneous ideal of $k[X_0, X_1]$ is $(X_0, X_1) = S_+$,
which is the irrelevant ideal, hence not part of $\Proj S$ (TODO), since
any maximal ideal is of the form $(X_0 - a_0, X_1 - a_1)$ by the Nullstellensatz.

The other prime ideals of $k[X_0, X_1]$ are principal, i.e. of the form
$(f)$ with $f$ irreducible or zero.

For $(f)$ to be homogeneous, $f$ must be homogeneous. Any such polynomial splits
into linear factors, all homogeneous, so in order for $f$ to be irreducible,
it must be linear.

Note that we hve a bijective correspondence between the collection of ideals
$(aX_0 - bX_1)$ with $a, b \in k$, $a, b$ not both zero and
$(k^2\setminus\set{0, 0})/k^\times$ given by $(aX_0 - bX_1)\mapsto (b : a)$.

Conclusion: The closed points of $\mathbb{P}^1_k = \Proj k[X_0, X_1]$ are in one-to-one correspondence
with points of $(k^2\setminus\set{0})/k^\times$.

More generally, the closed points of $P^n_k$ are in one-to-one correspondence with
points of $(k^{n+1}\setminus\set{0})/k^\times$. This is harder (but a good exercise), but it can be seen
by using the open cover $(D_+(X_i))$ (note that if $p \nsubseteq D_+(X_i)$ for any $i$,
then $X_i \in p$ for any $i$, hence $S_+ \subseteq p$, so $p\notin \Proj S$).
\end{example*}

\begin{example*}
\label{WeirdGrading}
Let $S = k[X_0, \ldots, X_n]$, but grade by $\deg X_i = w_i$, where
$w_0, \ldots, w_n$ are positive integers. Define weighted projective
space via $W\mathbb{P}^n(w_0, \ldots, w_n) = \Proj S$.

For example, consider $W\mathbb{P}^2(1, 1, 2)$. This has an open cover
$\set{D_+(X_i)}$. We have $D_+(X_2) \cong \Spec S_{(X_2)}$.
Note \[ S_{(X_2)} = k[u\coloneqq\frac{X_0^2}{X_2}, v\coloneqq\frac{X_0X_1}{X_2}, w\coloneqq\frac{X_1^2}{X^2}] \subseteq S_{X_2} \cong k[u, v, w]/(uw-v^2)\]
$\Spec S_{(X_2)}$ then is a quadric cone (an image is missing here).

$D_+(X_0)$ and $D_+(X_1)$ are both isomorphic to $\mathbb{A}_k^2$.
\end{example*}

\begin{example*}
\label{LatticeExample}
Let $M = \mathbb{Z}^n$, $M_\mathbb{R} = M\otimes_{\mathbb{Z}} \mathbb{R}=\mathbb{R}^n$.
 Let $\Delta \subseteq M_\mathbb{R}$ be a compact convex lattice polytope,
 i.e., there is some finite set $V \subseteq M$ such that $\Delta$ is the convex
 hull of $V$, i.e., the smallest convex set containing $V$.

(there is a picture missing here)

Let $C(\Delta) = \set{(m, r) \in M_\mathbb{R}\oplus \mathbb{R}\given m \in r\Delta, r\geq 0}$.
Here $r\Delta = \set{r\cdot m\given m \in \Delta}$. This is the cone over $\Delta$.

Let \[S = k[C(\Delta)\cap (M\oplus \mathbb{Z})] = \bigoplus_{p \in C(\Delta)\cap(M\oplus \mathbb{Z})}k\cdot z^p .\]
The thing in the square bracket is a monoid (use convexity to prove this).
We have a multiplication given by $z^p \cdot z^{p'} = z^{p + p'}$ making
$S$ into a ring, and it is graded by $\deg z^{m, r} = r$.

Define $\mathbb{P}_{\Delta} = \Proj S$. This is called a projective
toric variety.

Examples: If $\Delta$ is a standard $n$-simplex, i.e., the convex hull of
$\set{0, e_1, \ldots, e_n}$, then it is possible to check that
$S\cong k[X_0, \ldots, X_n]$ with the standard grading such that
$X_0 \leftrightarrow z^{(0, 1)}$, $X_0 \leftrightarrow z^{(e_i, i)}$. Hence
$\mathbb{P}_\Delta = P^n_k$.

Let $n = 2$ and $\Delta$ be the convex hull of $\set{(0, 0), (1, 0), (0, 1), (1, 1)}$,
i.e., the unit square. In $S$, the degree $d$ monomials are $\set{z^{(a, b, d)}\given 0\leq a, b\leq d}$.
Any of these monomials can be written as a product of monomials of degree $1$, i.e.,
$x\coloneqq z^{(0, 0, 1)}, y\coloneqq z^{(1, 0, 1)}, w\coloneqq z^{(0, 1, 1)}, t\coloneqq z^{(1, 1, 1)}$.
Thus $S = k[x, y, w, t]/(xt-yw)$ (it is possible but nontrivial to verify that this is
the only relation).

Hence, $\Proj S$ can be thought of as a quadric surface in $\mathbb{P}^3_k$.
\end{example*}


\chapter*{Exercises}
\label{Exercises}
\section*{Example Sheet 1}
\label{Sheet1}
\subsection*{Exercise 1}
\label{Ex1}
\begin{exercise*}
\label{FirstPart}
Let $A$ be a ring. Show that the sets
$D(f)\coloneqq \set{\mathfrak{p} \in \Spec A\given f\notin \mathfrak{p}}$
with $f$ ranging over elements of $A$ form a basis of the topology on $\Spec A$.
\end{exercise*}
\begin{proof}[Solution]
We have $\Spec A = D(1)$ and for  $f, g \in A$ we have
\begin{align*}
	D(f)\cap D(g) &= \set{\mathfrak{p} \in \Spec A\given f\notin \mathfrak{p} \wedge g\notin \mathfrak{p}}\\
	&= \set{\mathfrak{p} \in \Spec A\given fg\notin \mathfrak{p}}\\
	&= D(fg),
\end{align*}
so the collection $\set{D(f)}$ forms the basis of \emph{a} topology, and it remains
to show that the topology generated by the $D(f)$ is the Zariski topology. Firstly,
for any $f \in A$ we have
\begin{align*}
	\Spec A\setminus D(f) &= \set{\mathfrak{p} \in \Spec A\given f \in p}\\
	&= \set{\mathfrak{p} \in \Spec A\given (f) \subseteq \mathfrak{p}}\\
	&= V((f)),
\end{align*}
so each $D(f)$ is open. It remains to show that every open set is the union
of sets of the form $D(f)$. Indeed, if $I$ is any ideal of $A$, then
\begin{align*}
	\Spec A\setminus V(I) &= \set{\mathfrak{p} \in \Spec A\given I\nsubseteq \mathfrak{p}}\\
	&= \set{\mathfrak{p} \in \Spec A\given \exists f\in I\colon f\notin \mathfrak{p}}\\
	&=\bigcup_{f \in I}D(f)
\end{align*}
as required.
\end{proof}

\begin{exercise*}
\label{SecondPart}
An element $f \in A$ is nilpotent if and only if $D(f) = \varnothing$.
\end{exercise*}
\begin{proof}[Solution]
If $f$ is nilpotent, say $f^n = 0$, and $\mathfrak{p}$ is a prime ideal, then we have
$f^n = 0 \in \mathfrak{p}$, so $f \in \mathfrak{p}$. Hence, $D(f) = \varnothing$.

If  $f$ is not nilpotent, then define $\mathcal{S}$ to be the collection of all
ideals $I$ such that $f^n\notin I$ for every $n > 0$. Since $f$ is not nilpotent,
$(0) \in \mathcal{S}$. The set $\mathcal{S}$ is partially ordered by inclusion
and admits upper bounds, since the increasing union of ideals disjoint from $\set{f^n}$ is
still an ideal disjoint from $\set{f^n}$. Hence $\mathcal{S}$ admits a maximal member
$I$. We will show that $I$ is prime.

Let $x, y \in A$ such that $xy \in I$ and suppose that $x\notin I$, $y\notin I$.
Then $I + Ax$ and $I + Ay$ are not disjoint from $\set{f^n}$ so we find
$n, m \in \mathbb{N}$, $i, j \in I$ and $a, b \in A$ such that
$f^n = i + ax$,  $f^m = j + by$. But then $f^{n+m} = ij + iby + jax + abxy \in I$,
a contradiction, so $x \in I$ or $y \in I$ and $I$ is prime. Hence, $I \in D(f)$,
so $D(f) \neq \varnothing$.
\end{proof}


\subsection*{Exercise 4}
\label{Ex4}
\begin{notation*}
\label{StalkNotation}
For $s \in \mathcal{F}U$ and $p \in U$ we will write
$s_p \coloneqq (U, s) \in \mathcal{F}_p$.
\end{notation*}

\begin{definition*}
\label{AssociatedSheaf}
Let $\mathcal{F}$ be a presheaf and $U \subseteq X$ an open set.
Define

\[ \mathcal{F}^+U\coloneqq \set{s\colon U\to \coprod_{p \in U}\mathcal{F}_p\given \forall p \in U\colon s(p) \in \mathcal{F}_p, (\star)}, \]

where $(\star)$ is the following statement: for every $p \in U$ there is an open
$p \in V_p \subseteq U$ and a section $s_{V_p} \in \mathcal{F}U$ such that
for ever $q \in V_p$ we have $(s_{V_p})_q = s(q)$.
\end{definition*}

\begin{exercise*}
\label{AssociatedSheafIsSheaf}
$\mathcal{F}^+$ together with the obvious restriction maps forms a sheaf.
\end{exercise*}
\begin{proof}[Solution]
$\mathcal{F}^+U$ is an abelian group with pointwise addition, as the sum
of $s, t \in F^+U$ still satisfies $(\star)$ by taking the intersection of the
$V_p$ obtained from $s$ and $t$.

It is obvious that $\mathcal{F}^+$ is a presheaf.

Next, let $s \in \mathcal{F}^+U$ and $\set{U_i}$ an open cover such that
 $\forall i, s|_{U_i} = 0$. Let $p \in U$.  Then $p \in U_i$ for some $i$ and
 we have $s(p) = (s|_{U_i})(p) = 0$, so $s = 0$, so the identity axiom is satisfied.

Next, let $\set{U_i}_{i \in I}$ be a cover, $s_i \in \mathcal{F}^+U_i$ such that
$\forall i, j\colon s_i|_{U_i\cap U_j} = s_j|_{U_i\cap U_j}$. Given $p \in U$,
define $s(p)\coloneqq s_i(p)$ for $p \in U_i$. This is well-defined because of
the compatibility condition. We need to show that $s \in \mathcal{F}^+U$. Indeed,
let $p \in U$. Then $s(p) = s_i(p)$ for some $i$, and since $s_i \in \mathcal{F}^+U_i$
and taking stalks is compatible with restrictions, we get a neighborhood that
satisfies the required condition.
It remains to show that for all $i$, $s|_{U_i} = s_i$, but that is true by
definition.
\end{proof}

\begin{definition*}
\label{Inclusion}
For a presheaf $\mathcal{F}$ and an open set $U$, define
\[ \theta_U\colon \mathcal{F}U\to \mathcal{F}^+U;\qquad s\mapsto (p\mapsto s_p). \]
This is obviously a homomorphism of groups. It also defines a morphism of shaves,
because for $s \in \mathcal{F}U$, $V \subseteq U$ and $p \in V$ we have
\[ \theta_U(s)|_V(p) = \theta_U(s)(p) = s_p = (s|_V)_p =  \theta_V(s|V)(p). \]
\end{definition*}

\begin{lemma*}
\label{Lemma1}
Let $\mathcal{F}$ be a sheaf and $U$ an open set. Then the natural map
\[ \mathcal{F}U \to \prod_{p \in U} \mathcal{F}_p \]
is injective.
\end{lemma*}
\begin{proof}[Proof]
Let $s, t \in \mathcal{F}U$ such that $s_p = t_p$ for every $p$. Let $p \in U$.
By definition of a stalk, $s_p = t_p$ means that there is an open $p \in V_p \subseteq U$
such that $s|_{V_p} = t|_{V_p}$. These $V_p$ cover $U$ so by the identity axiom
we have $s = t$.
\end{proof}

\begin{lemma*}
\label{Lemma2}
Let $\mathcal{F}$ be a sheaf. Let $U$ be an open set.
Let $s\colon U\to\coprod_{p \in U} \mathcal{F}_p$ such that
for every $p \in U$ we have $s(p) \in \mathcal{F}_p$ and there is an open
$p \in V_p \subseteq U$ together with $s_{V_p} \in \mathcal{F}V_p$ such that
for every $q \in V_p$ we have $(s_{V_p})_q = s(q)$. Then there is a unique
$t \in \mathcal{F}U$ such that $t_q = s(q)$ for every $q \in U$.
\end{lemma*}
\begin{proof}[Proof]
Uniqueness follows from the previous lemma. For existence, notice that the
$V_p$ cover $U$. Let $p, q \in U$. The $s_{V_p}$ are gluable because their stalks
agree on the intersection, so the conditions of the gluing axiom are satisfied
by the previous lemma. Since talking stalks is compatible with restrictions, the
glued section has the correct stalks.
\end{proof}

\begin{exercise*}
\label{Factor}
Let $\mathcal{F}$ be a presheaf, $\mathcal{G}$ a sheaf and $\varphi\colon \mathcal{F}\to \mathcal{G}$
a morphism of presheaves. Then there is a unique morphism of sheaves
$\varphi^+\colon \mathcal{F}^+\to \mathcal{G}$ such that $\varphi = \varphi^+ \circ \theta$.
\end{exercise*}
\begin{proof}[Solution]
Let $U$ be an open and let $s \in \mathcal{F}^+U$.
Cover $U$ with the $V_p$ from the definition of $\mathcal{F}^+$
and obtain the associated $s_{V_p} \in \mathcal{F}V_p$. Define
$t_{V_p} \coloneqq \varphi_{V_p}(s_{V_p}) in \mathcal{G}V_p$. We can calulate that
for $q \in V_p$ we have
\[ (t_{V_p})_q = (\varphi_{V_p}(s_{V_p}))_q = \varphi_q((s_{V_p})_q) = \varphi_q(s(q)). \]
Therefore, Lemma 2 gives us a unique $t_U \in \mathcal{G}_U$ such that
\begin{equation}\tag{$\star$}
	\forall q \in U\colon (t_U)_q = \varphi_q(s(q)).
\end{equation}
We define $\varphi^+_U(s) = t_U$.

This is indeed a morphism of sheaves: if $V \subseteq U$ and $s \in \mathcal{F}^+U$,
then
\[ \varphi^+(s|_V) = \varphi^+(s)|_V \]
follows from the fact that, using $(\star)$, the germ of both sides at $p \in V$
is just $\varphi_p(s(p))$. By Lemma 1, the two sides are equal.

Similarly, if $s \in \mathcal{F}U$ and $p \in U$, then
\[ (\varphi^+_U\theta_U(s))_p \stackrel{(\star)}{=} \varphi_q(\theta(s)(q)) = \varphi_q(s_q) = (\varphi_U(s))_q, \]
so $\varphi^+_U \circ \theta_U = \varphi_U$ by Lemma 1, so
$\varphi^+ \circ \theta = \varphi$.

Finally, to see uniqueness, assume that $\varphi^\#$ satisfies
 $\varphi^\# \circ \theta = \varphi$. Let $s \in \mathcal{F}^+U$ and $p \in U$.
By definition of $\mathcal{F}^+$ there is $p \in V_p \subseteq U$, $s_{V_p} \in \mathcal{F}V_p$
such that $\forall q \in V_p\colon (s_{V_p})_q = s(q)$. The condition can be
reprased as $s|_{V_p} = \theta(s_{V_p})$ and we calculate
\begin{align*}
	(\varphi^\#_U(s))_p &= (\varphi_U^\#(s)|_{V_p})_p
= (\varphi_{V_p}^\#(s|_{V_p}))_p
= (\varphi_{V_p}^\#(\theta(s_{V_p})))_p\\
	&= (\varphi_{V_p}^+(\theta(s_{V_p})))_p = \ldots = (\varphi^+_U(s))_p,
\end{align*}
so by Lemma 1, we have $\varphi^+_U = \varphi^\#_U$, so $\varphi^+ = \varphi^\#$,
completing the proof of uniqueness.
\end{proof}

\begin{exercise*}
\label{Stalks}
We have $(\mathcal{F}^+)_p = \mathcal{F}_p$ for $p \in X$. Show that if $f\colon \mathcal{F}\to \mathcal{G}$ is
a morphism of presheaves, then there is an induced morphism $f^+\colon \mathcal{F}^+ \to \mathcal{G}^+$
with $(f^+)_p = f_p$.
\end{exercise*}
\begin{proof}[Solution]
Let $p \in X$. Of course, $(\mathcal{F}^+)_p$ and $\mathcal{F}_p$
cannot be literally equal. Instead, we show the following more precise
statement: The map $\theta_p\colon \mathcal{F}_p \to \mathcal{F}^+_p$ is an
isomorphism.

Indeed, we define $g_p\colon \mathcal{F}^+_p \to \mathcal{F}_p$ as follows:
for an open $U$ and $s \in \mathcal{F}^+U$ we define $g_p(s_p)\coloneqq s(p)$.
This is well-defined because sections $s \in \mathcal{F}^+U$,
$t \in \mathcal{F}^+V$
that have the same germ at $p$ must satisfy $s|_W = t|_W$ for some $W$ that contains
$p$, so  $s(p) = s|_W(p) = t|_W(p) = t(p)$.

Next, let $U$ be an open and $s \in \mathcal{F}_p^+$. By definition of $\mathcal{F}^+$,
there is some $p \in V_p \subseteq U$ open, $s_{V_p} \in \mathcal{F}V_p$ such that
for all $q \in V_p$ we have $(s_{V_p})_q = s(q)$. This is equivalent to saying
that $s|_{V_p} = \theta_{V_p}(s_{V_p})$, so in particular, in $\mathcal{F}_p^+$, we have
$s_p = (\theta_{V_p}(s_{V_p}))_p$. This lets us calculate
\[ \theta_p(g_p(s_p)) = \theta_p(s(p)) = \theta_p((s_{V_p})_p) = (\theta_{V_p}(s_{V_p}))_p = s_p, \]
so we have $\theta_p \circ g_p = \id_{\mathcal{F}^+_p}$.

Next, let $U$ be an open and $s \in \mathcal{F}U$. Then we have
\[ g_p(\theta_p(s_p)) = g_p(\theta_U(s)_p) = g_p((q\mapsto s_q)_p) = (q\mapsto s_q)(p) = s_p, \]
so $g_p \circ \theta_p = \id_{\mathcal{F}_p}$, and $\theta_p$ is an isomorphism as required.

Next, let $\mathcal{F}$ and $\mathcal{G}$ be presheaves and let $\theta\colon \mathcal{F}\to \mathcal{F}^+$
and $\iota\colon \mathcal{G}\to \mathcal{G}^+$ denote the natural maps to
the associated sheaf. If $f\colon \mathcal{F}\to \mathcal{G}$ is a map of presheaves,
we can invoke the universal property of $\mathcal{F}^+$ on the composite
$\iota \circ f$ and find a morphism
$f^+\colon \mathcal{F}^+ \to \mathcal{G}^+$ making the diagram
\[\begin{tikzcd}[row sep=1.5cm, column sep=1.5cm]
	\mathcal{F}\ar[r, "\theta"]\ar[d, "f"] & \mathcal{F}^+\ar[d, densely dotted, "f^+"]\\
	\mathcal{G}\ar[r, "\iota"] & \mathcal{G}^+
\end{tikzcd}\]
commute.

On stalks, we have
\[ f_p^+ \circ \theta_p = (f^+ \circ \theta)_p = (\iota \circ f)_p = \iota_p \circ f_p, \]
and since $\theta_p$ is an isomorphism, we have
\[ f_p^+ = \iota_p \circ f_p \circ \theta_p^{-1}, \]
which is how we should interpret the \enquote{equality} $(f^+)_p = f_p$ under the
natural identifications $\theta_p$ and $\iota_p$.
\end{proof}


\subsection*{Exercise 5}
\label{Ex5}
\begin{exercise*}
\label{Exercise5}
Show that if $f\colon \mathcal{F}\to \mathcal{G}$ is a morphism between sheaves,
then the sheaf image $\im f$ can be naturally identified with a subsheaf of
$\mathcal{G}$.
\end{exercise*}
\begin{proof}[Solution]
We will prove the following more general statement: if $\mathcal{F}$ is a presheaf
satisfying the identity axiom, $\mathcal{G}$ is a sheaf and $f\colon \mathcal{F}\to \mathcal{G}$
is a morphism of presheaves such that $f_U$ is injective for every $U$, then the induced
morphism $f^+_U\colon \mathcal{F}^+U\to \mathcal{G}U$ is injective for every $U$.

Indeed, the inclusion of the presheaf image into $\mathcal{G}$ satisfies these conditions.
It satisfies sheaf axiom 1 for the same reason that the presheaf kernel does.

We will now prove the claim. Let $U$ be an open set, $s \in \mathcal{F}^+U$ such that
$f^+_U(s) = 0$. From the construction of the associated sheaf we see that
$f^+_U(s) = f_U(t)$ where $t$ is the unique element of $\mathcal{F}U$ such that
$\forall q \in U\colon t_q = f_q(s(q))$.

So we have $0 = f^+_U(s) = f_U(t)$, so since $f_U$ is injective we have $t = 0$.
Let $q \in U$. Then $f_q(s(q)) = t_q = 0_q = 0$. The element $s(q)$ of $\mathcal{F}_q$
is represented by some open set $V$ and a section $u \in \mathcal{F}V$. Thus
$0 = f_q(s(q)) = f_q(V, u) = (V, f_V(u))$. Thus, there is some open $W \subseteq V$
such that $0 = f_V(u)|_W = f_W(u|_W)$. Since $f_W$ is injective, we conclude
$u|_W = 0$, and $u|_W$ represents the same element in $\mathcal{F}_q$ as $u$,
but that element is just $s(q)$, so $s(q) = 0$. Since $q$ was arbitrary, we
conclude $s = 0$.
\end{proof}


\subsection*{Exercise 6}
\label{Ex6}
\begin{exercise*}
\label{Exercise6}
A sequence of sheaves is exact if and only if for every $p \in X$ the corresponding
sequence of maps of abelian groups is exact.
\end{exercise*}
\begin{proof}[Solution]
Assume that $f\colon \mathcal{F}\to \mathcal{G}$ and $g\colon \mathcal{G}\to \mathcal{H}$
are morphisms of sheaves such that $g \circ f = 0$. Consider the diagram
\[\begin{tikzcd}[row sep = 1cm, column sep=1cm]
	\mathcal{F}\ar[r, "f"]\ar[d] & \mathcal{G}\ar[r, "g"] & \mathcal{H}\\
	\im' f\ar[ur]\ar[r, "\theta"]\ar[rr, bend right, "\iota"] &\im f\ar[r, "\varphi"]\ar[u] & \ker g,\ar[ul]
\end{tikzcd}\]
where the map $\iota$ is an inclusion of subsheaves of $\mathcal{G}$ and $\varphi$ is
induced by $\iota$. We say that $\im f = \ker g$ if $\varphi$ is an isomorphism. By
a result of the lecture, this is the case if and only if forall $p \in X$, the
induced map $\varphi_p\colon (\im f)_p \to (\ker g)_p$ is an isomorphism. Since
$\theta$ induces isomorphisms on stalks and the bottom triangle commutes, this is
the case if and only if $\iota_p\colon (\im' f)_p\to (\ker g)_p$ is an isomorphism
for every $p \in X$. Now consider the diagram
\[\begin{tikzcd}
	(\im' f)_p \ar[r, "\iota_p"]\ar[d, "\cong"] & (\ker g)_p\ar[d, "\cong"]\\
	\im f_p\ar[r, "i"] & \ker g_p,
\end{tikzcd}\]
where the left and right maps are the isomorphisms defined in the proof of a result
from the lecture and the bottom map is just the inclusion (this makes sense since
$g \circ f = 0 \iff \forall p \in X\colon g_p \circ f_p = 0$ as stalks characterize morphisms). The
diagram commutes since none of the maps actually does anything. Since the left and
right maps are isomorphisms, we have that the top map is an isomorphism if and only
if the bottom map is an isomophism.

But the bottom map is an isomorphism if and only if the sequence
\[\begin{tikzcd}
	\mathcal{F}_p\ar[r, "f_p"] & \mathcal{G}_p\ar[r, "g_p"] & \mathcal{H}_p
\end{tikzcd}\]
is exact, so putting everything together, we find that $(f, g)$ is exact if and only if
$(f_p, g_p)$ is exact for every $p \in X$.
\end{proof}


\subsection*{Exercise 7}
\label{Ex7}
\begin{exercise*}
\label{TheExercise}
Show that a morphism of sheaves is an isomorphism if and only if it is injective
and surjective.
\end{exercise*}
\begin{proof}[Solution]
Let $f\colon \mathcal{F}\to \mathcal{G}$ be a morphism of sheaves.
By a result from the lecture, $f$ is an isomorphism if and only if
$f_p\colon \mathcal{F}_p\to \mathcal{G}_p$ is an isomorphism for every $p$.
Since $f_p$ is a morphism of abelian groups, this is the case if and only if $f_p$
is injective and surjective for every $p$. By another result from the lecture,
this is the case if and only if $f$ is injective and surjective.
\end{proof}


\subsection*{Exercise 8}
\label{Ex8}
\begin{exercise*}
\label{First}
Let $\mathcal{F}'$ be a subsheaf of a sheaf $\mathcal{F}$. Then the
natural map $\mathcal{F}\to \mathcal{F}/\mathcal{F}'$ is surjective and has
kernel $\mathcal{F}'$ so that there is an exact sequence
\[\begin{tikzcd}
0 \arrow[r] & \mathcal{F}' \arrow[r] & \mathcal{F} \arrow[r] & \mathcal{F}/\mathcal{F}' \arrow[r] & 0.
\end{tikzcd}\]
\end{exercise*}
\begin{proof}[Solution]
The natural map $e\colon \mathcal{F}\to \mathcal{F}/\mathcal{F}'$ is
given as the composite $\theta \circ \hat{e}$, where $\hat{e}$ is
the map $\mathcal{F}\to \coker' i$, where $i\colon \mathcal{F}'\to \mathcal{F}$ is
the inclusion and $\coker' i$ is the presheaf cokernel of $i$, and
$\theta$ is the natural map into the sheafification.

For every $p \in X$, $\theta_p$ is surjective because $\theta$ induces
isomorphisms on stalks, and $\hat{e}_p$ is surjective, because
$\hat{e}$ is surjective on open sets, which in particular implies
surjectivity on stalks. Hence $e_p$ is surjective as the composite
of two open maps. By a result from the lecture, this implies that $e$
is surjective.

Since for any open set $U$ and $s \in \mathcal{F}'U$ we have
$e_U(s) = \theta_U(\hat{e}_U(s)) = \theta_U(0) = 0$, we obtain a map
$\varphi\colon\mathcal{F}' \to \ker e$. Let $p \in X$.
\[\begin{tikzcd}
	\mathcal{F}_p'\ar[d, "\varphi_p"]\ar[r, "i_p"] & \mathcal{F}_p\ar[r, "e_p"]\ar[dr, "\hat{e}_p"] & (\mathcal{F}/\mathcal{F}')_p\\
	(\ker e)_p\ar[ur]\ar[r, "\cong"] & \ker e_p\ar[u] & (\coker' i)_p\ar[u, "\theta_p"]
\end{tikzcd}\]
The map $\varphi_p$ is injective because $i_p$ is, and it is surjective, because
$\ker e_p = \ker \hat{e}_p = i_p$. Hence $\ker e = \mathcal{F}'$ as subsheaves
of $\mathcal{F}$.

Now we have $\im i = \mathcal{F}'$ as subsheaves of $\mathcal{F}$, since
the map $\theta\colon \mathcal{F}' = \im' i \to \im i$ induces isomorphisms on
stalks, but since the domain already is a sheaf this forces $\theta$ to be an
isomorphism. Hence $\im i = \ker e$ as subsheaves of $\mathcal{F}$, so the sequence
is exact at $\mathcal{F}$.  Exactness at $\mathcal{F}'$ and
$\mathcal{F}/\mathcal{F}'$ is trivially checked on stalks using Exercise 6.
\end{proof}

\begin{exercise*}
\label{second}
If
\[\begin{tikzcd}
	0 \arrow[r] & \mathcal{F}' \arrow[r] & \mathcal{F}  \arrow[r] & \mathcal{F}'' \arrow[r] & 0
\end{tikzcd}\]
is an exact sequence, then $\mathcal{F}'$ is isomorphic to a subsheaf of $\mathcal{F}$ and
$\mathcal{F}''$ is isomorphic to the quotient of $\mathcal{F}$ by this subsheaf.
\end{exercise*}
\begin{proof}[Solution]
We have a commutative diagram
\[\begin{tikzcd}[row sep=1cm, column sep=1cm]
	\mathcal{F'}\ar[d, "\hat{\imath}"]\ar[r, "i"] & \mathcal{F}\\
	\im' i\ar[r, "\theta"]\ar[ur] & \im i,\ar[u, "\iota"]
\end{tikzcd}\]
where the diagonal arrow is the inclusion, the bottom arrow is the natural map
into the associated sheaf, and the right arrow is induced by the diagonal arrow.
By Exercise 5, $\im i$ can be regarded as a subsheaf of $\mathcal{F}$.
Since $i$ is injective, for every $p \in X$, $i_p$ is injective, so by commutativity,
$\hat{\imath}_p$ is injective. Furthermore, for every $p \in X$, $\hat{\imath}$
is surjective, because it is surjective on open sets. Hence, the composite
$\theta \circ \hat{\imath}$ is an isomorphism on stalks. Since it is a map between
sheaves, this means that is is an isomorphism. Hence, $\mathcal{F}$ is isomorphic
to the subsheaf $\im i$.

Next, consider the diagram
\[\begin{tikzcd}[row sep=1cm, column sep=1cm]
	&\coker'\iota\ar[r, "\eta"]\ar[dr, "\hat{p}"]& \coker\iota\ar[d, "\hat{p}^+"]\\
	\im i\ar[r, "\iota"] &\mathcal{F}\ar[r, "p"]\ar[u, "\pi"] &\mathcal{F}'',
\end{tikzcd}\]
where the map $\hat{p}$ is defined on open sets using the fact that
$p \circ \iota = 0$, hence $p_U \circ \iota_U = 0$, hence $(\im i)(U) \subseteq (\ker p)(U)$.
The map, $\eta$ is the natural map into the associated sheaf and
$\hat{p}^+$ is obtained from the universal property. Since $p$ is surjective,
it is surjective on stalks, hence by commutatity $\hat{p}^+$ must also be surjective
on stalks.

I do not have a proof that $\hat{p}^+$ is injective on stalks.
\end{proof}


\subsection*{Exercise 9}
\label{Ex9}
\begin{exercise*}
\label{SectionsLeftExact}
If $U \subseteq X$ is an open subset and
\[\begin{tikzcd}
	0 \arrow[r] & \mathcal{F}' \arrow[r, "i"] & \mathcal{F} \arrow[r, "p"] & \mathcal{F}''
\end{tikzcd}\]
is exact, then
\[\begin{tikzcd}
	0 \arrow[r] & \Gamma(U, \mathcal{F}') \arrow[r, "i_U"] & \Gamma(U, \mathcal{F}) \arrow[r, "p_U"] & \Gamma(U, \mathcal{F}'')
\end{tikzcd}\]
is exact.
\end{exercise*}
\begin{proof}[Solution]
Since $(0, i)$ is exact, $(0, i_x)$ is exact for every $x \in U$, hence
$i_x$ is injective for every $x \in U$, hence $i$ is injective, hence
$\ker i = 0$, hence $0 = (\ker i)(U) = \ker i_U$, hence $i_U$ is injective,
so the sequence is exact at $\Gamma(U, \mathcal{F}')$.

It remains to show exactness at $\Gamma(U, \mathcal{F})$. Since
$p_U \circ i_U = (p \circ i)_U = 0_U = 0$, we have
$\im i_U \subseteq \ker i_U$.

Conversely, let $s \in \ker p_U$, i.e., $p_U(s) = 0$. By Exercise 6 we know
that $(i_x, p_x)$ is exact for every $x \in U$. Since $p_U(s) = 0$, we have
$p_x(U, s) = 0$ for every $x \in U$, hence $(U, s) \in \im i_x$ for all $x \in U$,
i.e., we find $(V_x, t_x) \in \mathcal{F}'_x$ such that $i_x(V_x, t_x) = (U, s)$.
If necessary, shrink $V_x$ such that $i_{V_x}(t_x) = s|_{V_x}$.

For $x, y \in U$, we have
\[ i_{V_x\cap V_y}(t_x|_{V_x\cap V_y} - t_y|_{V_x\cap V_y}) = s|_{V_x\cap V_y} - s|_{V_x\cap V_y} = 0. \]
Since $i$ is injective, we conclude $t_x|_{V_x\cap V_y} = t_y|_{V_x\cap V_y}$, hence
we can glue the $t_x$ to a $t \in \mathcal{F}'U$. For any $x \in U$ we have
\[ \mathcal{F}_x\ni (U, i_U(t)) = (V_x, i_{V_x}(t|_{V_x})) = (V_x, i_{V_x}(t_x))
	= (V_x, s|_{V_x}) = (U, s), \]
and since stalks characterize sections, this implies that $i_U(t) = s$, hence
$s \in \im i_U$ as required.
\end{proof}




\end{document}