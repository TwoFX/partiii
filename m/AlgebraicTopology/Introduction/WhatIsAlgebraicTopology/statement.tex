Algebraic Topology is the study of topological spaces up to homotopy equivalence.

Idea: homeomorphism is too delicate a relation; homotopy equivalence keeps track of
most \enquote{essential} topologyical information.

More precisely, we try to find functors from the category of topological spaces
into the category of groups. Therefore, the algebraic invariants we study are defined
for all spaces, but frequently have more structure for spaces which themselves have
more structure (e.g., manifolds).

Classically, the first attempt to do this was homotopy theory. The basic observation
is that given two loops at the same base point, we can concatenate them.
Quotienting out loops which are related by basepoint-preserving homotopies, we obtain
the fundamental group $\pi_1(X, x_0)$.

Similarly, there is a map $c\colon (S^n, p) \to (S^n, p) \vee (S^n, p)$ by collapsing the
equator. Using this map, we can combine $f, g\colon (S^n, p)\to (X, x_0)$ into
$f\vee g\colon (S^n, p) \to (X, x_0)$ by setting $f\vee g\coloneqq (f, g)\circ c$.
Again, after quotienting out homotopic maps, this defines the operation of a group
$\pi_n(X, x_0)$, the $n$-th homotopy group of $X$.

The issue with homotopy groups is that they are very hard to compute. For example,
not all hoptopy groups of $S^2$ are known at the moment. In fact,
there is no simply connected manifold of dimension greater than zero such that
all homotopy groups of $X$ are known.

Instead, we will focus on something else: (co)homology. These functors are
slightly more difficult to construct, but turn out to be vastly easier to
compute. Note though, computing the cohomology of complicated spaces is still
very hard.

Some general remarks:
\begin{itemize}
	\item The whole point of algebraic topology is being able to compute. Examples
		play a central role.
	\item There will be a particular focus on manifolds and smooth manifolds.
		There will be overlap with the Differential Geometry course.
\end{itemize}
