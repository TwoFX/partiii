We will first show that if $f\colon C\to D$ is any morphism and $c : C \to C$ and
$d : D \to D$ are idempotents, then $dfe = f \iff df = f = fe$.

Indeed, if $df=f=fe$, then $dfe = fe = f$. Conversely, if $dfe = f$, then
$f = dfe = ddfe = df$ and $f = dfe = dfee = fe$.

To show that $\mathcal{C}[\check{\mathcal{E}}]$ is a category, we need to show
that the composition of two morphisms is indeed a morphism and that there are
identity morphism.

Assume that $c\colon C \to C$, $d\colon D\to D$, $e\colon E \to E$ are idempotents
and that $f\colon C\to D$ and $g\colon D\to E$ satisfy $dfc = f$ and $egd = g$.
We need to show that $egfc = gf$. Using the lemma, we have $egf = (eg)f = gf$ and
$gfc = g(fc) = gf$, so, again by the lemma, the claim follows.

If $e\colon E \to E$ is an idempotent, define $1_e\coloneqq e \stackrel{e}{\longrightarrow} e$.
By idempotency of $e$, this is indeed a morphism. If $f\colon d \to e$ is a morphism,
then the morphism $f1_d$ is the morphism $fd = f$ (here we use the lemma again)
in $\mathcal{C}$, so $f1_d = f$ as required. Similarly, $1_ef = f$. This completes part (i).

Next, assume that $\mathcal{E}$ contains all identity morphisms of $\mathcal{C}$.
Define the functor $I$ via
\begin{align*}
	I\colon \mathcal{C}&\to \mathcal{C}[\check{\mathcal{E}}]\\
	A &\mapsto 1_A\\
	(f\colon A\to B)&\mapsto (f\colon 1_A\to 1_B)
\end{align*}
This is indeed a functor and since the data of a morphism $A\to B$ in $\mathcal{C}$ is
precisely the same as the data of a morphism $1_A \to 1_B$ in $\mathcal{C}[\check{\mathcal{E}}]$,
$I$ is fully faithful.

Now let $T\colon \mathcal{C}\to \mathcal{D}$ be any functor.

First, assume that there is some functor $\widehat{T}\colon \mathcal{C}[\check{\mathcal{E}}]\to \mathcal{D}$
such that $T = \widehat{T}I$. Let $e : A\to A \in \mathcal{E}$ be an idempotent. Then we have
\begin{align*}
	Te &= \widehat{T}(1_A \stackrel{e}{\longrightarrow}1_A)\\
	&= \widehat{T}(1_A \stackrel{e}{\longrightarrow}e\stackrel{e}{\longrightarrow}1_A)\\
	&= \widehat{T}(e\stackrel{e}{\longrightarrow}1_A)\circ \widehat{T}(1_A\stackrel{e}{\longrightarrow}e),
\end{align*}
and we also have
\begin{align*}
	\widehat{T}(1_A\stackrel{e}{\longrightarrow} e)\circ\widehat{T}(e\stackrel{e}{\longrightarrow}1_A)
	&= \widehat{T}(e\stackrel{e}{\longrightarrow}1_A\stackrel{e}{\longrightarrow}e)\\
	&= \widehat{T}(e\stackrel{ee}{\longrightarrow}e)\\
	&= \widehat{T}(e\stackrel{e}{\longrightarrow}e)\\
	&= \widehat{T}(1_e)\\
	&= 1_{\widehat{T}e},
\end{align*}
which shows that $Te$ is split.

Next, assume that $Te$ is split for any $e \in \mathcal{E}$. For any $e \in \mathcal{E}$,
choose a splitting
\[\begin{tikzcd}
	TA \ar[r, "g_e", shift left] & B_e\ar[l, "f_e", shift left],
\end{tikzcd}\]
i.e., $f_e \circ g_e = Te$, $g_e \circ f_e = 1_{B_e}$.
For identity morphisms $1_A$ (A an object of $\mathcal{C}$), choose the specific splitting
given by $B_{1_A} \coloneqq TA$, $f_{1_A} \coloneqq 1_{TA}$, $g_{1_A}\coloneqq 1_{TA}$.

Now define the functor $\widehat{T}$ via
\begin{align*}
	\widehat{T}\colon \mathcal{C}[\check{\mathcal{E}}] &\to \mathcal{D}\\
	(e\colon A\to A) &\mapsto B_e\\
	(f\colon d\to e) &\mapsto g_e \circ Tf \circ f_d.
\end{align*}

If $e \in \mathcal{E}$, then we have
\begin{align*}
	\widehat{T}(1_e) &= g_e \circ Te \circ f_e\\
	&= g_e \circ f_e \circ g_e \circ f_e\\
	&= 1_{B_e} \circ 1_{B_e} = 1_{B_e}
\end{align*}

Furthermore, if $f\colon c\to d$ and $g\colon d\to e$, then we have
\begin{align*}
	\widehat{T}(g \circ f) &= g_e \circ T(g \circ f) = f_c\\
	&= g_e \circ Tg \circ Tf = f_c\\
	&= g_e \circ Tg \circ T(d \circ f) \circ f_c\\
	&= g_e \circ Tg \circ Td \circ Tf \circ f_c\\
	&= g_e \circ Tg \circ f_d \circ g_d \circ Tf \circ f_c\\
	&= \widehat{T}g \circ \widehat{T}f.
\end{align*}

So $\widehat{T}$ is indeed a functor. If $A$ is an object of $\mathcal{C}$, then
\[ \widehat{T}IA = \widehat{T}1_A = B_{1_A} = TA \]
and if $f\colon C\to D$ is a morphism in $\mathcal{C}$, then
\[ \widehat{T}If = \widehat{T}(1_C\stackrel{f}{\longrightarrow}1_D) = g_{1_D} \circ Tf \circ f_{1_C} = 1_{TD} \circ Tf \circ 1_{TC} = Tf, \]
so $\widehat{T}$ is the required factorisation, completing part (ii).

Define a functor $\Phi\colon [\widehat{\mathcal{C}}, \mathcal{D}]\to [\mathcal{C}, D]$ via
$F\mapsto F \circ I$, $\eta\mapsto I\eta$, where $I\eta$ is defined cia
$I\eta_C \coloneqq \eta_{IC} = \eta_{1_C}$. Naturality of $I\eta$ immediately follows
from naturality of $\eta$. Functoriality is also clear.

We will show that this functor is full, faithful and essentially surjective.

Indeed, let $F\colon \mathcal{C}\to \mathcal{D}$ be a functor. Then $\widehat{F}$ as
defined in the previous part satisfies $\Phi\widehat{F} = F$, so $\Phi$ is essentially
surjective.

Next, let $F, G\colon \widehat{\mathcal{C}}\to \mathcal{D}$ be functors and $\eta\colon F\circ I\to G \circ I$ a
natural transformation. For an idempotent $e\colon A\to A$ in $\mathcal{C}$, define $\hat{\eta}_e$ to be the composite
\[\begin{tikzcd}[column sep=huge]
	Fe \ar[r, "F(e\stackrel{e}{\longrightarrow}1_A)"] & F1_A = (F\circ I)A \ar[r, "\eta_A"] & (G \circ I)A = G1_A \ar[r, "G(1_A\stackrel{e}{\longrightarrow}e)"] & Ge.
\end{tikzcd}\]

We claim that this defines a natural transformation $\hat{\eta}\colon F \to G$. Indeed,
if $f\colon d\to e$ is a morphism, then
\begin{align*}
	\hat{\eta}_e \circ Ff &= G(1_A\stackrel{e}{\longrightarrow}e) \circ\eta_E \circ F(e\stackrel{e}{\longrightarrow}1_E) \circ F(d\stackrel{f}{\longrightarrow} e)\\
	&= G(1_E\stackrel{e}{\longrightarrow}e) \circ\eta_E \circ F(d\stackrel{f}{\longrightarrow}e\stackrel{e}{\longrightarrow}1_E)\\
	&= G(1_E\stackrel{e}{\longrightarrow}e) \circ\eta_E \circ F(d\stackrel{d}{\longrightarrow}1_D\stackrel{d}{\longrightarrow}d\stackrel{f}{\longrightarrow}e\stackrel{e}{\longrightarrow}1_E)\\
	&= G(1_E\stackrel{e}{\longrightarrow}e) \circ\eta_E \circ F(d\stackrel{d}{\longrightarrow}1_D\stackrel{d}{\longrightarrow}d\stackrel{f}{\longrightarrow}e\stackrel{e}{\longrightarrow}1_E)\\
	&= G(1_E\stackrel{e}{\longrightarrow}e) \circ\eta_E \circ F(1_D\stackrel{efd}{\longrightarrow}1_E)\circ F(d\stackrel{d}{\longrightarrow}1_D)\\
	&= G(1_E\stackrel{e}{\longrightarrow}e) \circ\eta_E \circ F(efd)\circ F(d\stackrel{d}{\longrightarrow}1_D)\\
	&= G(1_E\stackrel{e}{\longrightarrow}e) \circ G(efd) \circ \eta_D\circ F(d\stackrel{d}{\longrightarrow}1_D),
\end{align*}
and doing the whole thing backwards we conclude that $\hat{\eta}_e \circ Ff = Gf \circ \hat{\eta}_d$,
so $\hat{\eta}$ is indeed a natural transformation.

For any $A \in \mathcal{C}$ we have
\begin{align*}
	(I\hat{\eta})_A &= \hat{\eta}_{IA} = \hat{\eta}_{1_A} = G(1_A \stackrel{1_A}{\longrightarrow}1_A) \circ \eta_A \circ F(1_A\stackrel{1_A}{\longrightarrow}1_A)\\
	&= G(1_{1_A}) \circ \eta_A \circ F(1_{1_A}) = \eta_A,
\end{align*}
which means that $\Phi(\hat{\eta}) = \eta$, so $\Phi$ is full.

Finally, let $F, G\colon \widehat{\mathcal{C}}\to \mathcal{D}$ be functors and
$\eta, \eta'\colon F\to G$ be natural transformations such that $\Phi(\eta) = \Phi(\eta')$.
To show that $\Phi$ is faithful, we need to prove that $\eta = \eta'$.
The assumption $\Phi(\eta) = \Phi(\eta')$ means that for all $A \in \mathcal{C}$
we have $\eta_{IA} = \eta'_{IA}$, so $\eta_{1_A} = \eta'_{1_A}$.

Let $e\colon A\to A$ be any idempotent in $\mathcal{C}$. We need to show that
$\eta_e = \eta'_e$. Indeed, we have
\begin{align*}
	\eta_e &= G(1_e) \circ \eta_e\\
	&= G(e\stackrel{e}{\longrightarrow}e) \circ \eta_e\\
	&= G(e\stackrel{ee}{\longrightarrow}e)\circ \eta_e\\
	&= G(e\stackrel{e}{\longrightarrow}1_A\stackrel{e}{\longrightarrow}e)\circ\eta_e\\
	&= G(1_A\stackrel{e}{\longrightarrow}e)\circ G(e\stackrel{e}{\longrightarrow}1_A)\circ \eta_e\\
	&= G(1_A\stackrel{e}{\longrightarrow}e)\circ\eta_{1_A}\circ F(e\stackrel{e}{\longrightarrow}1_A)\\
	&= G(1_A\stackrel{e}{\longrightarrow}e)\circ\eta'_{1_A}\circ F(e\stackrel{e}{\longrightarrow}1_A),
\end{align*}
and the same argument in backwards direction shows that $\eta_e = \eta'_e$,
completing the proof.
