Let $X$ be a topological space. Then for any $x \in X$, the constant map
$c_x\colon X \to X$ sending $y \in X$ to $x$ is continuous. Furthermore, the
identity on $X$ is continuous. This, if $X$ is infinite, then $X$ has infinitely
many endomorphisms, and if $X$ is finite, then $X$ has at least $\abs{X} + 1$
endomorphisms.

Now assume that $X$
has precisely three endomorphisms. Then $X$ is finite
and has at most two points. Clearly, if $X$ has zero or one point, then there is only
one endomorphism. So $X$ has two points, say  $X = \set{a, b}$. There are four
set-functions $\set{a, b}\to \set{a, b}$, three of which (the identity and the two
constant maps) are continuous regardless of the topology. The final map interchanges
$a$ and $b$ and is not continuous.

The empty set and all of $X$ are open. If $X$ had the trivial or the discrete topology,
then the interchange would be continuous, a contradiction. Hence, precisely one of the
sets $\set{a}$ and $\set{b}$ is open. Sending the member of that set to $1$ and the
other element to $0$ describes a homeomorphism with $S$.
