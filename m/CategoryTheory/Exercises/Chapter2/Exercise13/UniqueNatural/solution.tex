Of course, the terminal object in the category of sets is just the one-element set
$1 = \set{\star}$.
Since $F1$ is also terminal, it is in bijection with $1$. We write
$F1 = \set{\star_{F1}}$.

By the Yoneda lemma, the set of natural transformations
$\mathsf{Set}(1, {-}) \to F$
is in bijection with $F1$, so there is a unique natural transformation
$\eta\colon \mathsf{Set}(1, {-}) \to F$.  Examining the proof, we see that the
components of this natural transformation are given by
\begin{align*}
	\eta_A \colon \mathsf{Set}(1, A)&\to FA\\
	f&\mapsto Ff(\star_{F1})
\end{align*}
for any object $A$ of $\mathcal{C}$. Let $A$ be an object of $\mathcal{C}$.
We will show that $\eta_A$ is an isomorphism, i.e., a bijection.

First, let $x \in FA$. Then
$\eta_A(F^{-1}(\star_{F1}\mapsto x)) = x$, so $\eta_A $ is surjective.

Additionally, let $f, g \colon 1\to A$ such that $\eta_A(f) = \eta_A(g)$. Since
a map $F1\to FA$ is completely determined by its value at $\star_{F1}$, we must
have $Ff = Fg$. But then $f = F^{-1}F(f) = F^{-1}F(g) = g$.

This means that $\eta_A$ is an isomorphism, so $\eta$ is in fact a natural
isomorphism.

We define a natural transformation $\alpha\colon 1_{\mathsf{Set}}\to \mathsf{Set}(1,{-})$
by setting \[ \alpha_A(a)(\star)\coloneqq a. \]
The naturality square for $f\colon A\to B$ is
\[\begin{tikzcd}
	A\ar[r, "\alpha_A"]\ar[d, "f"] & \mathsf{Set}(1, A)\ar[d, "g\mapsto f \circ g"]\\
	B\ar[r, "\alpha_B"] &\mathsf{Set}(1, B)
\end{tikzcd}\]
Both paths are just $a \mapsto (\star \mapsto f(a))$, so $\alpha$ is natural.
It is also clear that $\alpha_A$ is bijective, so $\alpha$ is a natural isomorphism.
In other words, $\star$ is a universal element of the identity functor.

In particular, this tells is that composition with $\alpha$ and its inverse
exhibits a bijection between
the collection of natural transformations
\[ 1_{\mathsf{Set}} \to F \]
and the collection of natural transformations
\[ \mathsf{Set}(1, {-}) \to F. \]
This means that there is a unique natural transformation $1_{\mathsf{Set}}\to F$,
and it is given by $\alpha \circ \eta$, and since $\alpha$ and $\eta$ are both
natural isomorphisms, so is $\alpha \circ \eta$, completing the proof.
