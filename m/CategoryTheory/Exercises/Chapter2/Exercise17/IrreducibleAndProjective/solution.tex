If $F$ is irreducible and projective, then by part (i) we find an epimorphism
$e\colon \mathcal{C}(A, {-})\to F$ for some $A$. Applying the projectivity of
 $F$ to the diagram
\[\begin{tikzcd}[row sep=1cm, column sep=1cm]
	& F\ar[d, "1_F"]\ar[dl, densely dotted, "s"]\\
	\mathcal{C}(A, {-})\ar[r, two heads, "e"] & F
\end{tikzcd}\]
yields $s\colon F\to \mathcal{C}(A, {-})$ such that $es = 1$, so $e$ is split.

Conversely, if $e\colon \mathcal{C}(A, {-}) \to F$ admits a section
$s\colon F\to \mathcal{C}(A, {-})$ such that $es = 1$, then $F$ is irreducible by
part (i). Suppose we have a morphism  $f\colon F\to R$ and an epimorphism
$g\colon Q\to R$.
\[\begin{tikzcd}[row sep=1cm, column sep=1cm]
	\mathcal{C}(A, {-})\ar[r, shift left, "e", two heads]\ar[d, densely dotted, "h"] & F\ar[l, shift left, "s"]\ar[d, "f"]\\
	Q\ar[r, two heads, "g"] & R
\end{tikzcd}\]
Since $\mathcal{C}(A, {-})$ is projective by 2.11, we find some $h\colon \mathcal{C}(A, {-}) \to Q$ such
that $fe = gh$. But then $ghs = fes = f$, hence $hs\colon F\to Q$ solves the lifting
problem, and $F$ is projective.
