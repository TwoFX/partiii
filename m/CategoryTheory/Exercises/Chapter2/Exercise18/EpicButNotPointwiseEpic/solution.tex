Let $G\colon \mathsf{2}\to D$ be the functor that sends the morphism
$0\to 1$ to $h$. Consider any functor $H\colon \mathsf{2}\to D$ and
a natural transformation $\eta\colon G\to H$.
\[\begin{tikzcd}[row sep=1cm, column sep=1cm]
	B\ar[r, "\eta_0"]\ar[d, "h"] & H0\ar[d, "H(0\to 1)"]\\
	D\ar[r, "\eta_1"]& H1.
\end{tikzcd}\]
Clearly, $H_1 = D$, $\eta_1 = 1_D$.
$H_0$ is either $B$ or $D$. If $H_0 = B$, then $\eta_0 = 1_B$ and $H(0\to 1) = h$.
If $H_0 = D$, then $\eta_0 = h$ and $H(0\to 1) = 1_D$. In both cases,
there is only one natural transformation $G\to H$. Hence, any natural transformation
$\alpha\colon F\to G$ is automatically epic. Choose $F$ to be the functor that
sends $0\to 1$ to $f$ and set $\alpha_0 \coloneqq f$, $\alpha_1\coloneqq h$. Then
$\alpha$ is a natural transformation. By what we have just seen, it is epic,
but $\alpha_0 = f$ is not an epimorphism, hence $\alpha$ is not pointwise epic.
