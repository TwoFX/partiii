Suppose that $K$ has characteristic zero. Then we can identify $\mathcal{Q}$ with
a subfield of $K$. Since $K$ is finitely generated as a $\mathbb{Z}$-algebra (this
is just a different way of saying that $K$ is finitely generated as a ring), it
is certainly finitely generated as a $\mathbb{Q}$-algebra. By Zariski's lemma,
$K$ is a finite-dimensional $\mathbb{Q}$-vector space.

Hence, all assumptions
for the Artin-Tate lemma for the chain $\mathbb{Z} \subseteq \mathbb{Q} \subseteq K$
are satisfied, so we find that $\mathbb{Q}$ is a finitely generated $\mathbb{Z}$-algebra.
This is of course nonsense: if we had finitely many generators, then only finitely
many primes could appear as divisors of denominators in $\mathbb{Q}$.

Therefore, $K$ has characteristic $p > 0$ and is a finitely generated $\mathbb{Z}$-algebra,
so $K$ is also a finitely generated $\mathbb{Z}/p\mathbb{Z}$-algebra. Hence, by
Zariski's lemma, $K$ is a finite-dimensional $\mathbb{Z}/p\mathbb{Z}$-vector space,
hence $K$ is finite.
