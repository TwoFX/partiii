For a zero divisor $a \in A$ denote by  $\Sigma_a$ the set of ideals containing
$a$ in which every element is a zero divisor (notice that $\Sigma = \Sigma_0$). Since $(a) \in \Sigma_a$, we know
that $\Sigma_a$ is nonempty. Furthermore, the union of a chain of ideals in
$\Sigma_a$ is once again an element of $\Sigma_a$, hence $\Sigma_a$ admits
a maximal element $I_a$ by Zorn's lemma.

Let $x, y \in A$ such that $xy \in I_a$, $x\notin I_a$ and $y\notin I_a$. Then
The ideals $I_a + Ax$ and $I_a + Ay$ contain non-zero-divsors $u$ and $v$. Write
$u = i + u_1x$ and $v = j + u_2y$ with $i, j \in I_a$, $u_1, u_2 \in A$. Then
$uv = ij + iu_2y + ju_1x + u_1u_2xy \in I_a$, hence $uv$ is a zero divisor, but
then $u$ and $v$ are also zero divisors, a contradiction.

Hence $I_a$ is prime and if $Z$ is the set of zero divisors, then we find that
\[ Z = \bigcup_{a \in Z} I_a \]
as required.
