If $V$ has a finite basis $v_1, \ldots, v_n$, then
defining  $V_i\coloneqq \langle v_1, \ldots, v_{n-i}\rangle$ gives a composition
series, hence (1) implies (2).

(2) implies (3) and (2) implies (4) by part (i) of the exercise.

If $V$ is not finite-dimensional, then choose a basis $B$ and let
$v_1, v_2, \ldots \in B$ pairwise distinct. Then
\[ \langle v_1 \rangle \subsetneq \langle v_1, v_2 \rangle \subsetneq \dots \] is
an infinite strictly ascending chain and
\[ \langle B \rangle \supsetneq \langle B\setminus\set{v_1}\rangle \supsetneq \dots \]
is an infinite strictly descending chain. Hence (3) implies (1) and (4) implies (1).
