First assume that $f$ is nilpotent. Write $f = \sum_{i = 0}^\infty a_iX^i$ for
some $a_i \in A$. We will argue by induction. Since $f$ is nilpotent,
there is some $k \in \mathbb{N}$ such that $f^k = 0$. The constant term of
$f^k$ is $a_0^k$, hence $a_0$ is nilpotent.

Next, assume that $a_0, \ldots, a_n$ are nilpotent for some $n \in \mathbb{N}$.
Then they are also nilpotent as elements of $A[[X]]$. Since the set of nilpotent
elements forms an ideal, we have that $g\coloneqq\sum_{i = 0}^na_iX^i$ is nilpotent, so
$f - g = \sum_{i = n+1}^\infty a_iX^i$ is nilpotent, i.e., there is some $k$ such
that $(f - g)^k = 0$. But the $X^{k(n+1)}$-coefficient of $(f - g)^k$ is just
$a_{n+1}^k$, hence $a_{n+1}$ is nilpotent.

Next, assume that $f = \sum_{i = 0}^\infty a_iX^i$ and every $a_i$ is nilpotent.
Denote by $I$ the ideal of $A$ generated by all $a_i$. Then $I \subseteq N(A)$.
Since $N(A) = \sqrt{0}$, by Lemma 1.21, there is some natural number $n$ such
that $N(A)^n \subseteq (0)$. Since $I^n \subseteq N(A)^n$, this implies that
$I^n = (0)$. Since the coefficients of $f^n$ are elements of $I^n$, we conclude
that $f^n = 0$, so $f$ is nilpotent.
