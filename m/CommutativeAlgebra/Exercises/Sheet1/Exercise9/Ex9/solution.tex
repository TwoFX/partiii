For the second part, let $P$ be a prime ideal in $A$. $P[X]$ is obviously
an ideal of $A[X]$. Let $f, g \in A[X]$ such that $fg \in P[X]$. We can write
\[ f = \sum_{i = 0}^n a_iX_i,\quad g = \sum_{i = 0}^m b_iX_i, \]
and also define $a_i = 0$ for $i > n$ and $b_i = 0$ for $i > m$. Suppose
that $f\notin P[X]$ and $g\notin P[X]$. Then we find $i$ and $j$ such that
$a_i \notin P$, $b_j \notin P$. Choose $i$ and $j$ to be minimal among the possible
$i$ and $j$. Then the coefficient of $fg$ for $X^{i+j}$
is given by
\[ \left(\sum_{k = 0}^{i - 1}a_kb_{i + j - k}\right) + a_ib_j + \left(\sum_{k = i + 1}^{i + j}a_k b_{i+j-k}\right). \]
The coefficient is in $P$, and so are the sums on the left and the right, by minimality
of $i$ and $j$. But then $a_ib_j \in P$, so $a_i \in P$ or $b_j \in P$, a contradiction.
Hence $f \in P[X]$ or $g \in P[X]$.

Let $Q$ be a maximal ideal of $A$. Then $1 \notin Q$, hence $1\notin Q[X]$ and $X\notin Q[X]$,
hence $1 \notin (Q[X], X)$, but $X \in (Q[X], X)$. We conclude that
$Q[X] \subsetneq (Q[X], X) \subsetneq A[X]$, so $Q[X]$ is not a maximal ideal.

The proof of the third part is almost identical to the proof of Hilbert's basis
theorem. We will show that every submodule of $M[X]$ is finitely generated.
Let $N$ be an $A[X]$-submodule of $M[X]$ and define
$N_n\coloneqq\set{f \in N\given \deg f\leq n}$. We have $0 \in N_n$ and
$N_0 \subseteq N_1 \subseteq \cdots$ form an ascending chain.

Define $M_n$ to be the set of coefficients of $X^n$ appearing in elements
of $N_n$. If $m+n \in M_n$ and $a \in A$, then $m+n \in M_n$ and $am \in M_n$.
Therefore $M_n$ is an $A$-submodule of $M$.

Furthermore, if $m \in M_n$, then $m \in M_{n+1}$ by multiplying the corresponding
polynomial by $X$.

Since $M$ is noetherian, the chain $M_0 \subseteq M_1 \subseteq \cdots$ terminates,
so we have $k$ such that $\forall n\geq k\colon M_n = M_k$. Each of
$M_0, \ldots, M_k$ is a finited generated submodule of $M$, say
$M_j$ is generated by $m_{j1}, \ldots, m_{j\ell_j}$. There are polynomials
$f_{j1}, \ldots, f_{j\ell_j} \in N$ such that $\deg f_{ji} = j$ and the
leading coefficient of $f_{ji}$ is $m_{ji}$.

We will show that the finite set $\set{f_{ji}\given 0\leq j\leq N, 1\leq i\leq \ell_j}$
generated $N$.

We will use induction on $\deg f$, where $f \in N$. If $\deg f = 0$, then $f = m$
for some $m \in M$. By definition of $M_0$, $m \in M_0$, and $m$ is in the
submodule generated by the $f_{0i}$.

Assume next that $0 < \deg f\leq k$ and that the claim is true for smaller degrees.
Let $m$ be the leading coefficient of $f$. Then $m \in M_n$ so we may write
\[ m = \sum_j a_{nj}m_{nj}. \]
Then
\[ f - \sum_j a_{nj}f_{nj} \]
is in $N$ and of smaller degree, so is expressible as a linear combination of
the $f_{ij}$, so $f$ is expessible as a linear combination as well.

Finally, assume that $N < \deg f$ and that the claim is true for smaller degrees.
If $m$ is the leading coefficient of $f$, then $m \in M_n = M_k$ so we may write
\[ m = \sum_j a_{kj}m_{kj}. \]
Then
\[ f - X^{n-k}\sum_ja_{kj}f_{kj} \]
is in $N$ and of smaller degree, so is expressible as a linear comination of the
$f_{ij}$, so $f$ is expressible as a linear combination as well.
