The claim is trivial if $0 \in S$. Hence, in the remainder, we will
assume that $0\notin S$.

Let $m/s \in T(S^{-1}M)$, i.e., we find $0\neq r/t \in S^{-1}R$ such that $rm/st = 0/1$, i.e.,
there is some $u \in S$ such that $urm  = 0$. Observe that $r\neq 0$, otherwise we would
have $r/t = 0 \in S^{-1}R$, and $u\neq 0$, since $u\in S$. Since $A$ is an integral
domain, this implies that $ru\neq 0$, hence $m \in T(M)$, so
$m/s \in S^{-1}T(M)$.

Conversely, let $m/s \in S^{-1}(T(M))$. This means that we find $0\neq r \in A$ such
that $rm = 0$. Then $r/1 \cdot m/s = 0/s = 0 \in S^{-1}M$, i.e.,
$m/s \in T(S^{-1}M)$, completing the proof.
