To show that (a) implies (b), notice that from the previous result
we have $T(M_{\mathfrak{p}}) = T(M)_{\mathfrak{p}}$ as $S^{-1}R$-submodules
of $M_{\mathfrak{p}}$. But the right hand side is trivial as $T(M) = 0$.

The implication from (b) to (c) is trivial.

Finally, assume that $T(M_{\mathfrak{m}})$ is trivial for all maximal ideals
$\mathfrak{m}$. Let $m \in T(M)$. Consider the annihilator
$\ann(m) = \set{r \in R\given rm = 0}$. Suppose $\ann(m)$ is a proper
ideal. Then $\ann(m) \subseteq \mathfrak{m}$ for some maximal ideal
$\mathfrak{m}$. By Exercise 14(ii) and our assumption, $m$ is in the kernel
of the map $M\to M_{\mathfrak{m}}$, so we find $s \in R\setminus \mathfrak{m}$
such that $sm = 0$. But then $s \in \ann(m)\cap R\setminus \mathfrak{m} = \varnothing$,
a contradiction. We conclude $\ann(m) = R$, so in particular $m = 1\cdot m = 0$,
i.e., $M$ is torsion-free.
