The union of a chain in $\Sigma$ is again an element of $\Sigma$, and the singleton
set  $\set{1}$ is an element of $\Sigma$. Hence, $\Sigma$ admits maximal elements
by Zorn's lemma.

If $S \in \Sigma$ is maximal, we claim that $I\coloneqq R\setminus S$ is a prime
ideal. If $r, s \in I$,
then $SM_r$ and $SM_s$, where $M_r\coloneqq \set{1, r, r^2, \ldots}$,
are multiplicatively closed subsets. Since  $r, s \notin S$, these are strictly
larger than $S$, hence must contain $0$, i.e., we find natural numbers
$n, m$ and $x, y \in S$ such that $xr^n = 0 = ys^m$. Then $xy(r+s)^{n+m} = 0$
by the binomial theorem, so we must have $r + s \in I$, since otherwise
we would have $0 \in S$, a contradiction.

If $r \in R$, $t \in I$, then again we find $n \in \mathbb{N}$ and $x \in S$ such
that $xt^n = 0$. Then $r^nxt^n = 0$, so if we have $rt \in S$, then $0 \in S$,
hence $rt \in I$. This makes $I$ into an ideal.

Next, let $r, s \in R$ such that $rs \in I$. Again, this means that we find $n \in \mathbb{N}$
and $t \in S$ such that $(rs)^nt = 0$. If $r$ and $s$ were both in $S$, this would again
lead to the contradiction $0 \in S$, hence $r \in I \vee s \in I$, making $I$ into
a prime ideal.

It remains to show that $I$ is minimal. If $\mathfrak{p} \subseteq I$ is a prime ideal,
then $R \setminus \mathfrak{p}$ is multiplicative, does not contain $0$, and
satisfies $S \subseteq R\setminus \mathfrak{p}$. By maximality of $S$, we find
$S = R\setminus \mathfrak{p}$, so $\mathfrak{p} = I$.

Conversely, assume that $R\setminus S$ is a minimal prime ideal. Then $S$ is
multiplicative, because $R\setminus S$ is prime. Suppose $S$ is not maximal.
Then we have $S \subsetneq S'$ for some maximal element $S'$ of $\Sigma$. Then
by what we have just shown,
$R\setminus S'$ is a minimal prime ideal, but then $R\setminus S$ cannot be a
minimal prime, since it is a strict superset of $R\setminus S'$.
