We perform induction on $n$. The case $n = 0$ is trivial, and the case
$n = 1$ is just the Hauptidealsatz. Hence, suppose $n > 1$.

As in the proof of the Hauptidealsatz, by passing to $R_{\mathfrak{p}}$, we may
assume that $R$ is local and that $\mathfrak{p}$ is the unique maximal ideal
$\mathfrak{p}$.

Pick any prime such that $\mathfrak{q}$ is maximal and satisfies
$\mathfrak{q}\subsetneq \mathfrak{p}$ (this is possible since $R$ is noetherian).
In particular, $\mathfrak{p}$ is the only prime strictly containing $\mathfrak{q}$.
We will show that $\hgt(\mathfrak{q}) \leq n - 1$. Then, if we have any maximal
chain of prime ideals ending in $\mathfrak{p}$, its second-to-last term will be
such a  $\mathfrak{q}$ (otherwise it would be possible to extend the chain), so
it follows that $\hgt(\mathfrak{p})\leq n$ as required.

Since $\mathfrak{p}$ is a minimal prime over $I$, we must have $I \nsubseteq \mathfrak{q}$.
By assumption, we find generators $a_1, \ldots, a_n$, and at least one of these
is not contained in $\mathfrak{q}$. Reorder them such that $a_n\notin \mathfrak{q}$.

Now $\mathfrak{q} + (a_n)$ is strictly larger than $\mathfrak{q}$, so if
$\mathfrak{q}+(a_n)$ is contained in any prime, this prime must be $\mathfrak{p}$
by how we chose $\mathfrak{q}$. In particular, the image of $\mathfrak{p}$ is the only
prime ideal in $R/(\mathfrak{q}+(a_n))$.
Replaying a part of the proof of the Hauptidealsatz
with $\mathfrak{q} + (a_n)$ in place of $(a)$, this means that $R/(\mathfrak{q} + (a_n))$
is artinian, hence noetherian. Now 1.21 tells us that the image of $\mathfrak{p}$ is
nilpotent, i.e., $\mathfrak{p}^m \subseteq \mathfrak{q} + (a_n)$ for some $m$.

In particular, since $I \subseteq \mathfrak{p}$, for all $1\leq i\leq n-1$, we have
have $a_i^m \in \mathfrak{q} + (a_i)$, i.e., we may choose $x_i \in \mathfrak{q}$
and $r_i \in R$ such that $a_i^m = x_i + r_ia_n$. Note that this means that any
prime ideal in $R$ that contains $x_1, \ldots, x_{n-1}$ and $a_n$ must contain
$a_1, \ldots, a_n$, so it must be equal to $\mathfrak{p}$. Also, since $x_i \in \mathfrak{q}$, we have
$J\coloneqq \sum_{i=1}^{n-1} Rx_i \subseteq \mathfrak{q}$. Now if we can show that
$\mathfrak{q}$ is minimal over $J$, we are done, since $J$ is generated by
$n - 1$ elements, so $\hgt(q)\leq n-1$ by the inductive hypothesis.

Indeed, let $\pi\colon R\to R/J$ be the projection. The ring $R/J$ is local
with unique maximal ideal $\pi(\mathfrak{p})$. Now any prime over the ideal
$\pi((a_n))$ lifts to a prime ideal of $R$ containing $x_1, \ldots, x_{n-1}$ and
$a_n$ so it is equal to $\mathfrak{p}$. Thus, $\pi(\mathfrak{p})$ is a minimal
prime over $\pi((a_n))$. The Hauptidealsatz now tells us $\hgt(\pi(\mathfrak{p}))\leq 1$,
but since $\pi(\mathfrak{q}) \subsetneq \pi(\mathfrak{p})$, this means that
$\hgt(\pi(\mathfrak{q})) = 0$, i.e., $\pi(\mathfrak{q})$ is a minimal prime in
$R/J$, i.e., $\mathfrak{q}$ is a minimal prime over $J$ as required, and we are
done.
