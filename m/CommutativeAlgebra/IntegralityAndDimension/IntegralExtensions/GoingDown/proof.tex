By induction, it will be sufficient to consider the case $m = 1$, $n = 2$, i.e.,
we are given prime ideals $\mathfrak{p}_1 \supsetneq \mathfrak{p}_2$ of $R$ and
a prime ideal $\mathfrak{q}_1$ of $T$ such that $\mathfrak{q}_1\cap R = \mathfrak{p}_1$.
We need to produce a prime ideal $\mathfrak{q}_2 \subseteq \mathfrak{q}_1$ such that
$\mathfrak{q}_2\cap R = \mathfrak{p}_2$.

Let $S_1 = T\setminus \mathfrak{q}_1$ and $S_2 = R\setminus \mathfrak{p}_2$.
$S\coloneqq S_1S_2$ is a multiplicatively closed subset of $T$ satisfying
$S_1 \subseteq S$ and $S_2 \subseteq S$.

Assume for now that $T \mathfrak{p}_2\cap S = \varnothing$ (we will prove this later).
$T \mathfrak{p}_2$ is an ideal of $T$, so we have $S^{-1}(T \mathfrak{p}_2) \subseteq S^{-1} T$
is an ideal, and it is a proper ideal, since otherwise we would have
$x \in T \mathfrak{p}_2$ and $s, y \in S$ such that $s(x-y) = 0$, but then
$sx = sy \in T \mathfrak{p}_2 \cap S = \varnothing$ (since $T \mathfrak{p}_2$ is
an ideal and $S$ is multiplicatively closed), a contradiction.

Hence $S^{-1} T \mathfrak{p}_2$ is contained in some maximal ideal $S^{-1}T$, which
by 2.5, is of the form $S^{-1} \mathfrak{q}_2$ for some prime ideal $\mathfrak{q}_2$
of $T$ satisfying $\mathfrak{q}_2\cap S = \varnothing$. Furthermore, if
$x \in T \mathfrak{p}_2$, then $x/1 \in S^{-1}(T \mathfrak{p}_2)$, i.e., we find
$y \in \mathfrak{q}_2$, $s_1, s_2 \in S$ such that $s_2(y - xs_1) = 0$. Hence,
$x(s_1s_2) \in \mathfrak{q}_2$, but $\mathfrak{q}_2$ is prime, $s_1s_2 \in S$ and
$\mathfrak{q}_2\cap S = \varnothing$, so $x \in \mathfrak{q}_2$. We conclude that
$T \mathfrak{p}_2 \subseteq \mathfrak{q}_2$ and hence
$\mathfrak{p}_2 \subseteq T \mathfrak{p}_2 \cap R \subseteq \mathfrak{q}_2 \cap R$.
On the other hand, if $\mathfrak{q}_2 \cap R \nsubseteq \mathfrak{p}_2$, then
we find $x \in \mathfrak{q}_2\cap R$ such that $x\notin \mathfrak{p}_2$. The
latter means that $x \in R\setminus \mathfrak{p}_2 = S_2 \subseteq S$, but then
$x \in \mathfrak{q}_2 \cap S = \varnothing$, a contradiction. We conclude
$\mathfrak{p}_2 = \mathfrak{q}_2\cap R$.

Similarly, if $\mathfrak{q_2} \nsubseteq \mathfrak{q}_1$, then we find
$x \in \mathfrak{q}_2$ such that $x\notin \mathfrak{q}_1$. Then
$x \in T\setminus \mathfrak{q}_1 = S_1 \subseteq S$, so $x \in \mathfrak{q}_2 \cap S = \varnothing$.
This shows, $\mathfrak{q}_2 \subseteq \mathfrak{q}_1$, so $\mathfrak{q}_2$ has
the desired properties.

It remains to show that $T \mathfrak{p}_2 \cap S = \varnothing$. Suppose
$x \in T \mathfrak{p}_2 \cap S$. Using the definition of $S$ and the fact that
$T$ is an integral domain, we see that $x\neq 0$. Lemma 4.22 with $I = \mathfrak{p}_2$
tells us that the integral closure of  $\mathfrak{p}_2$ in $T$ is $\sqrt{T \mathfrak{p}_2}$.
In particular, $x$ is in the integral closure of $\mathfrak{p}_2$. Lemma
4.23 then tells us that it is algebraic over $K$, the field of fractions of $R$,
and the coefficients of the minimal polynomial $f$ of $x$ over $K$ are contained
in $\sqrt{\mathfrak{p}_2} = \mathfrak{p}_2$.

Since $x \in S$, it is of the form $x=rt$ with $t \in S_1$, $r \in S_2$. If
$r_i$ are the coefficients of $f$, then the minimal polynomial of $t = x/r$ over $K$ has
coefficients $r_i' \coloneqq r_i/r^{n-i} \in K$. Since $t$ is integral over $R$ by assumption, we may apply
4.23 with $I = R$ and find that $r_i'$ is in fact contained in $R$.

Now $r_i = r_i'r^{n-i} \in \mathfrak{p}_2$ and $r \notin \mathfrak{p}_2$ since
$r \in S_2$, so we must have $r_i' \in \mathfrak{p}_2$. Since these coefficients
belong to  a monic polynomial killing $t$, this just means that $t$ is integral
over $\mathfrak{p}_2$. Hence, 4.22 tells us that $t \in \sqrt{T \mathfrak{p}_2}$.
We have
$\mathfrak{p}_2 \subseteq \mathfrak{p}_1 \subseteq \mathfrak{q}_1$, which
implies $T \mathfrak{p}_2 \subseteq T \mathfrak{q}_1 = \mathfrak{q}_1$. Since
$\mathfrak{q}_1$ is prime, this implies $\sqrt{T \mathfrak{p}_2} \subseteq \mathfrak{q}_1$,
so we conclude $t \in \mathfrak{q}_1$.
On the other hand, $t \in S_1 = T\setminus \mathfrak{q}_1$, so we have
arrived at the desired contradiction.
