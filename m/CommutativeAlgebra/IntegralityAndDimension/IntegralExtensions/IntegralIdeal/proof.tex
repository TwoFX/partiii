For (i), if $x \in T$, then we have an expression
$x^n = r_{n-1}x^{n-1}+\cdots + r_0 = 0$ for some $r_i \in R$.
Projecting onto $T/J$, this yields an equation
 $\overline{x}^n + \overline{r_{n-1}}\overline{x}^{n-1} + \cdots + \overline{r_0} = \overline{0}$
 in $T/J$, such that $\overline{r_i} \in (R+J)/J$,
 hence $\overline{x}$ is interal over $(R+J)/J$ as
 required.

 For (ii), let $x/s \in S^{-1}T$. Again we have
$x^n = r_{n-1}x^{n-1}+\cdots + r_0 = 0$ for some $r_i \in R$.
In particular, this implies
\[ \left(\frac{x}{s}\right)^n + \frac{r_{n-1}}{s}\left(\frac{x}{s}\right)^{n_1}+\cdots+\frac{r_0}{s_n} = 0 \]
in $S^{-1}T$, so $x/s$ is integral over $S^{-1}R$.
