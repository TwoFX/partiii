We will make use of the trace function. There are many ways to define it
(cf. Galois theory). We will use it as a black box with the following
property (cf. Reid 8.13): if $K \subseteq L$ is separable, then the map
$L\times L\to K$ given by $(x, y)\mapsto \Tr(xy)$ is a non-degenerate
symmetric $K$-bilinear form. Furthermore, the trace of and element $x$ is an
integer multiple of a coefficient of the minimal polynomial of $x$.

Pick a $K$-basis $z_1, \ldots, z_n$ of $L$. Each $z_i$ is algebraic over $K$,
so it satisfies a monic polynomial with coefficients in $K$, say
\[ z_i^n + \frac{r_{n-1}}{s_{n-1}}z_i^{n-1} + \cdots + \frac{r_0}{s_0} = 0. \]
By multiplying with the product of the denominators, we find $t_i \in R$
such that we have an equation of the form
\[ t_nz_i^n + t_{n-1}z_i^{n-1} + \cdots + t_0 = 0. \]
Finally, define $y_i\coloneqq t_nz_i$. Then we have
\begin{multline*}
y_i^n + t_{n-1}y_i^{n-1} + t_nt_{n-2}y_i^{n-2} + t_n^2t_{n-3}y_i^{n-3} + \cdots + t_n^{n-1}t_0\\
	= t_n^{n-1}(t_nz_n + t_{n-1}z_{n-1}+\cdots t_0) = 0.
\end{multline*}
Therefore, $y_i \in T_1$, and the $y_i$ still form a $K$-basis of $L$.
By non-degeneracy, we find a $K$-basis $x_1, \ldots, x_n$ such that $\Tr(x_iy_j) = \delta_{ij}$.

Let $x \in T_1$. Then may write $x = \sum \lambda_i x_i$ for some $\lambda_i \in K$.
Then we may calculate
\[ \Tr(xy_j) = \sum_i\lambda_i \Tr(x_iy_j) = \sum\lambda_i\delta_{ij} = \lambda_j. \]
On the other hand, $xy_j \in T_1$, since the integral closure is a subalgebra.
By (4.23) with $I = R$, the minimal polynomial of $xy_j$ has coefficients in $R$,
so in particular $\lambda_j = \Tr(xy_j) \in R$. Therefore, $x \in \sum Rx_i$, so
we have an inclusion of $R$-submodules $T_1 \subseteq \sum Rx_i$. But
$\sum Rx_i$ is a finitely generated module over a Noetherian ring, so it is
Noetherian itself. Hence, $T_1$ is finitely generated.
