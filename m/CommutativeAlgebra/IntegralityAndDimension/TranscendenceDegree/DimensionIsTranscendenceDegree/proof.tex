We proceed by induction over $r\coloneqq \trdeg_k K$. For $r = 0$, there
is nothing to do.

By Noether Normalisation, we find algebraically independent elements
 $x_1, \ldots, x_r \in T$ such that $T$ is integral over $R = k[x_1, \ldots, x_r]$,
This implies that $K$ is algebraic over $k(x_1, \ldots, x_r)$, which in turn
implies that $\trdeg_k(K) = \trdeg_kk(x_1, \ldots, x_r) = r$.
By 4.19, $\dim T = \dim R = \dim k[X_1, \ldots, X_r]$.

Thus, it remains to show that $\dim k[X_1, \ldots, X_r] = r$. We have already
seen that $\dim k[X_1, \ldots, X_r] \geq r$.
Take
a chain $\mathfrak{p}_0 \subsetneq \cdots\subsetneq \mathfrak{p}_s$ of primes in $R$.

Since $R$ is an integral domain, we may assume that $\mathfrak{p}_0 = 0$. Furthermore,
we may assume using Lemma 4.5 that $\mathfrak{p}_1 = (f)$.

As shown in an example below, we have $\trdeg_k(K_f) = r-1$, where $K_f$ is the
field of fractions of $R/(f)$. By induction, we find $\dim R/(f) = r-1$.

Next, consider the chain
\[ \mathfrak{p}_1/\mathfrak{p}_1 \subsetneq \mathfrak{p}_2/\mathfrak{p}_1 \subsetneq\cdots\subseteq \mathfrak{p}_s/\mathfrak{p}_1. \]
This is again a chain of prime ideals (since $\mathfrak{p}_1 \subseteq \mathfrak{p}_i$
and the quotient map is surjective) in $R/(f)$ of length $s-1$. Hence
$s - 1 \leq r-1$ and so $s\leq r$, so we conclude $\dim R = r$ as required.
