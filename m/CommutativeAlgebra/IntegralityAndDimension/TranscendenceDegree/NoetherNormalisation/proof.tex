Since $T$ is affine, we have $T = k[a_1, \ldots, a_n]$ for some $n \in \mathbb{N}$,
$a_i \in T$. We proceed by induction on $n$. Let $r$ denote the maximal number
of algebraically independent elements of $\set{a_i}$. Without loss of generality, $r\geq 1$,
since otherwise all elements of $T$ are integral over $k$ (the kernel of the map
$k[X] \to k[a]$ contains a monic polynomial for every $a)$, so $R = k$ will do the
trick.

If $r = n$, there is nothing to do.
Reorder the $a_i$  in such a way that
$a_1, \ldots, a_r$ are algebraically indepdent
and $a_{r+1}, \ldots, a_n$ are algebraically dependent on $a_1, \ldots, a_r$ over
$k$.

In particular, we find $0\neq f \in k[X_1, \ldots, X_r, X_n]$ such that
$f(a_1, \ldots, a_r, a_n)= 0$ (this exists because $a_1, \ldots, a_r, a_n$ are
algebraically dependent). Then $f$ is a sum of terms of the form
$\lambda_\ell X_1^{\ell_1}\cdots X_r^{\ell_r}X_n^{\ell_n}$, where
$\ell = (\ell_1, \ldots, \ell_r, \ell_n)$ and $\ell_i \in \mathbb{N}_0$.

We claim that there are positive integers $m_1, \ldots, m_r$  such that
$\varphi\colon \ell\mapsto m_1\ell_1 + \cdots + m_r\ell_r + \ell_n$ is injective
for those $\ell$ with $\lambda_\ell \neq 0$.

Since there are only finitely many $\ell$ such that $\lambda_\ell = 0$, there
are only finitely many $d = \ell - \ell'$ with $\lambda_\ell \neq 0$ and
$\lambda_{\ell'}\neq 0$. Writing $d = (d_1, \ldots, d_r, d_n)$, consider the
finitely many $d = (d_1, \ldots, d_r)\neq 0$ obtained in this way (observe that
we have dropped the final component).
Vectors in $\mathbb{Q}^n$
that are orthogonal to one of these $r$-tuples lie in finitely many $(r-1)$-dimensional
subspaces. Hence it is possible to pick $(q_1, \ldots, q_r)$ such that each
$q_i$ satisfies $q_i > 0$ and $\sum q_id_i\neq 0$ for all of the finitely many
$(d_1, \ldots, d_r)\neq 0$. Multiplying by a sufficiently large positive integer,
we obtain
$(m_1, \ldots, m_r) \in \mathbb{Z}_{\geq 0}^r$ satisfying
$\abs{\sum m_id_i} > \abs{d_n}$ for all of the $(d_1, \ldots, d_r) \neq 0$

Now if  $\ell$ and $\ell'$ are such that $\varphi(\ell) = \varphi(\ell')$, define
$d = \ell - \ell'$ such that $\varphi(d) = 0$. Notice that this implies
$d_1 = \ldots = d_r = 0$, since otherwise we would have $\varphi(d) \neq 0$ by
the inequality of absolute values. But then $0 = \varphi(d) = d_n$, hence
$\ell = \ell'$. This completes the proof of the claim.

Pick $m_1, \ldots, m_r$ as in the claim and set
\[ g(X_1, \ldots, X_n)\coloneqq f(X_1 + X_n^{m_1}, \ldots, X_r+X_n^{m_r}, X_n). \]
Thus, $g$ is a sum of the form

\[ \sum_{l\colon \lambda_\ell\neq 0} \lambda_\ell(X_1+X_n^{m_1})^{\ell_1}\cdots(X_r+X_n^{m_r})^{\ell_r}X_n^{\ell_n}. \]
By our choice of $m_i$, different terms will have different powers of $x_n$. Hence,
there will be a single term with the highest power of $X_n$. Viewing $g$ as a
polynomial in $X_n$, the leading coefficient is one of the $\lambda_{\ell}$, so
in particular it is an element of $k$.

Next, define $b_i\coloneqq a_i - a_n^{m_i}$ ($1\leq i\leq r$) and
$h(X_n) \coloneqq g(b_1, \ldots, b_r, X_n)$.

Then the leading coefficient of $h$ is once again in $k$ and all coefficients are
elements of $k[b_1, \ldots, b_r]$. Moreover, we may calculate
\[ h(a_n) = g(b_1, \ldots, b_r, a_n) = f(a_1, \ldots, a_r, a_n) = 0, \]
using the definition of $g$ and the defining property of $f$.

Now divide $h$ by its leading coefficient to find that $a_n$ is integral over
$k[b_1, \ldots, b_r]$. Additionally, for each $i\leq r$, $a_i = b_i + a_n^{m_i}$ is also
integral over $k[b_1, \ldots, b_r]$ (recall that sums and products of integral
elements are integral).

This means that $T$ is integral over $S\coloneqq k[b_1, \ldots, b_r, a_{r+1}, \ldots, a_{n-1}]$.
Since that is one generator less than before, we find that $S$ is integral over
some polynomial algebra $R$, so $T$ is also integral over $R$.
