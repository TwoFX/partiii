We say that $x_1, \ldots, x_n$ are algebraically indepdendent over $k$ if
the ring map $k[X_1, \ldots, X_n]\to k[x_1, \ldots, x_n]$ which sends
$X_i\mapsto x_i$ is an isomorphism. In this situation, $k[x_1, \ldots, x_n]$ may
be regarded as a polynomial algebra.

As in linear algebra, we consider maximal
algebraically indepdendent sets: they all have the same size (we will not prove
this here). Such a set is called a transcendence basis over $k$ and the
transcendence degree is the cardinality.

There are some concepts that carry over
from linear algebra: an algebraically independent set can be thought of like a
linearly independent set, the algebraic closure of a set $S$ is like the span
of $S$ and transcendence degree is like dimension.
