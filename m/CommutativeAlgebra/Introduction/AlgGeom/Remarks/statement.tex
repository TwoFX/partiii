\begin{enumerate}[label=(\arabic*)]
	\item One can replace $I$ by the ideal generated by $I$ and get the same
		algebraic set. Replacing an ideal by a generating set of the ideal leaves
		the algebraic set unchanged. Hilbert's basis theorem asserts that any
		algebraic set is the set of common zeros of a finite set of polynomials.

	\item
		\begin{align*}
			\bigcap_j Z(I_j) &= Z(\bigcup_j I_j),\\
			\bigcup_{j = 1}^nZ(I_j) &= Z(\prod_{j = 1}^nI_j)
		\end{align*}
		for ideals $I_j$. Define a topology of $\mathbb{C}^n$ with closed sets
		being the algebraic sets. This is the Zariski topology; it is coarser than
		the normal topology on $\mathbb{C}^n$.

	\item For $S \subseteq \mathbb{C}^n$ define
		\[ I(S) \coloneqq \set{f  \in \mathbb{C}[X_1, \ldots, X_n]\given \forall (a_1, \ldots, a_n) \in S\colon f(a_1, \ldots, a_n) = 0}. \]
		This is an ideal of $\mathbb{C}[X_1, \ldots, X_n]$ and it is radical, i.e.,
		if $f^r \in I(S)$ for some $r\geq 1$, then $f \in I(S)$.

		The Nullstellensatz is a family of results asserting that the corrspondence
		\begin{align*}
			I &\mapsto Z(I)\\
			I(S) &\mapsfrom S
		\end{align*}
		gives a bijection between the radical ideals of $\mathbb{C}[X_1, \ldots, X_n]$ and the
		algebraic subsets of $\mathbb{C}^n$. In particular, the maximal ideals of
		$\mathbb{C}[X_1, \ldots, X_n]$ correspond to points in $\mathbb{C}^n$.
\end{enumerate}
