A large section of the course treats dimension of rings:
\begin{itemize}
	\item the maximal length of chains of prime ideals;
	\item in geometric context in terms of growth rates (uses Hilbert function);
	\item the transcendence degree of the field of fractions (of an integral domain).
\end{itemize}
Over commutative rings these all give the same answer. A fourth way uses homological
algebra and gives the same answer at least for nice noetherian rings.

Most of the theory dates between 1920 and 1950.

Rings of dimension $0$ are called artinian rings. In dimension $1$, special things
happen which are important in number theory; this is crucial in the study of algebraic
curves.
