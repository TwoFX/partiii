Denote by $\Sigma_n$ the symmetric group on $\set{1, \ldots, n}$. $\Sigma_n$ acts
on $k[X_1, \ldots, X_n]$ by permuting variables: given $\sigma \in \Sigma_n$,
$f \in k[X_1, \ldots, X_n]$, we set
\[ (\sigma f)(X_1, \ldots, X_n)\coloneqq f(X_{\sigma^{-1}(1)}, \ldots, X_{\sigma^{-1}(n)}). \]
The action of $\Sigma_n$ is via ring automorphisms so it makes sense to define
the \emph{ring of invariants}
\[ S\coloneqq \set{f \in k[X_1, \ldots, X_n]\given \forall \sigma \in \Sigma_n\colon \sigma f = f}. \]
$S$ is a ring, called the \emph{ring of symmetric polynomials}. Consider the following
elementary symmetric functions:
\begin{align*}
	e_1(X_1, \ldots, X_n) &= X_1 + \cdots + X_n),\\
	e_2(X_1, \ldots, X_n) &= \sum_{i < j} X_iX_j,\\
	&\ \,\vdots\\
	e_n(X_1, \ldots, X_n)&= X_1\cdot\cdots\cdot X_n.
\end{align*}

It turns out that $S$ is generated as a ring by these $e_i$ and the canonical
map $k[Y_1, \ldots, Y_n]\to S$ given by $Y_i\mapsto e_i$ is an isomorphism of rings.

Hilbert showed that $S$ is finitely generated for many other groups. Among the
way he proved a few very deep results.
\begin{itemize}
	\item the basis theorem,
	\item the Nullstellensatz,
	\item the polynomial nature of the Hilbert function (and beginnings of dimension theory),
	\item the syzygy theorem (and beginnings of the homological theory of polynomial rings).
\end{itemize}
