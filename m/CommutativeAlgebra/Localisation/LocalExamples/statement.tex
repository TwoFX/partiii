\begin{enumerate}
	\item Let $R = \mathbb{Z}$, and $p$ prime number. Then $(p)$ is a prime ideal,
		and we have $R_{(p)} = \set{\frac{m}{n}\given p\nmid n} \subseteq \mathbb{Q}$.

		The maximal ideal is given by $\set{\frac{m}{n}\given p\mid m, p\nmid n}$.
	\item Let $R = k[X_1, \ldots, X_n]$, $p = (X_1-\alpha_1, \ldots, X_n-\alpha_n)$.
		Then we can interpret $R_p$ as a subring of $k(X_1, \ldots, X_n)$ consisting
		of those rational functions that are defined at
		$(\alpha_1, \ldots, \alpha_n) \in k^n$, and the unique maximal ideal consists
		of those rational functions which are zero at $(\alpha_1, \ldots, \alpha_n)$.
\end{enumerate}
