\begin{enumerate}
	\item If $R$ is an integral domain and $S = R\setminus\set{0}$, then $S^{-1}R$
		is just the field of fractions of $R$.
	\item We have that $S^{-1}R$ is the zero ring if and only if $0 \in S$.
	\item If $I$ is an ideal of $R$, then $S = 1 + I$ is multiplicatively closed.
	\item Let $p$ be a prime ideal. Then $S = R\setminus p$ is multiplicatively closed
		(indeed, if $x, y \in S$, then if  $xy \in R \setminus S = p$, then
		 $x \in p = R \setminus S$ or $y \in p = R\setminus S$, which is not
		 possible). We write $R_p$ for $S^{-1}R$, and the process of passing
		 from $R$ to $R_p$ is called localisation at $p$. The elements $\frac{r}{s}$
		 with $r \in p$ form an ideal of $R_p$. This is a unique maximal ideal
		 in $R_p$: if $\frac{r}{s}$ satisfies $r\notin p$, then $r \in S$, so
		 $\frac{r}{s}$ has an inverse in $R_p$ and is not part of any maximal
		 ideal.
\end{enumerate}
