It is obvious that (i) implies (ii) and (ii) implies (iii), so it will suffice
to show that (iii) implies (i). Indeed, suppose that $M_q = 0$ for every maximal
ideal $q$, but $M \neq 0$.

Let $0\neq m \in M$. The annihilator $\set{r \in R\given rm = 0}$ of $m$ is
a proper ideal of $R$, hence it is contained in a maximal ideal $q$ of $R$.
Since $M_q$ is trivial, we have $\frac{m}{1} = 0$ in $M_q$, so there is some
$s \in R\setminus q$ such that $sm = 0$ in $R$. But since $s \notin q$,
we have $s\notin \ann(m)$, i.e., $sm \neq 0$, a contradiction.
