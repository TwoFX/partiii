Let $p$ be any prime ideal of $R$ and let $s \in R\setminus p$. The
set $S\coloneqq\set{1, s, s^2, \ldots}$ is multiplicative, so we get
a localization $S^{-1}R$ and a map $\theta\colon R\to S^{-1}R$.
$R$ is a finitely generated $k$-algebra and $S^{-1}R$ generated as a
$k$-algebra by $\theta(R)$ and $1/s$. Hence $S^{-1}R$ is a finitely generated
$k$-algebra. Let $q$ be a maximal ideal of $S^{-1}R$ containing $S^{-1}p$. By
the weak Nullstellensatz,  $S^{-1}R/q$ is a finite field extension of $k$.

The ideal  $p_1\coloneqq \theta^{-1}(q)$ is a prime ideal containing $p$, and
by the correspondence of prime dieals we know that  $p_1$ does not meet $S$.
Hence $\theta$ induces an injective $k$-vector space homomorphism
$R/p_1\to S^{-1}R/q$. Since $S^{-1}R/q$ is finite-dimensional, this implies
that $R/p_1$ is also finite-dimensinoal.

By the remark, this implies that $R/p_1$ (which is an integral domain since
 $p_1$ is prime) is a field, hence $p_1$ is a maximal ideal. Hence, for any
 $s \notin p$, we find a maximal ideal containing $p$ but not containing $s$,
i.e.,
\[ R\setminus p \subseteq \bigcup \set{\text{complemenents of maximal ideals containing $p$}}. \]
By elementary set theory, this means that
\[ \bigcap\set{\text{maximal ideals containing $p$}} \subseteq p. \]
Since the converse inclusion is trivial, we have that $p$ is the intersection
of maximal ideals containing $p$. Hence the intersection of all primes is
the same as the intersection of all maximals, which is what we wanted to show.
