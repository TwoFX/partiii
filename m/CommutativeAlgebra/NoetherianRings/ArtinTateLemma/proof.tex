Suppose $T$ is generated as an $R$-algebra by $t_1 = 1, \ldots, t_n \in T$.
By assumption, we have $x_1 = 1\ldots, x_m \in T$ such that $T = Sx_1 + \cdots + Sx_m$.
Therefore, if $1\leq i\leq n$, we may write
\begin{equation}\tag{1}
t_i = \sum_{j=1}^m s_{ij}x_j
\end{equation}
for some $s_{ij} \in S$. Furthermore,  $1\leq i, j\leq m$, we find $s_{ijk} \in S$
satisfying
\begin{equation}\tag{2}
x_ix_j = \sum_{k = 1}^m s_{ijk}x_k.
\end{equation}
Define $S_0$ as the $R$-subalgebra of $S$ generated by the $s_{ij}$ and the
$s_{ijk}$. We have $R \subseteq S_0 \subseteq S$. If $t \in T$, we may write
$t$ as a polynomial in the $t_i$. Since $t_1 = 1$, we may assume that this
polynomial does not have a constant term. Substituting (1) and then repeatedly
substituting (2), we find that $T$ is finitely generated by the $x_i$ as
a $S_0$ module.

Next, we note that $S_0$ is a noetherian ring. Since $S_0$ is finitely
generated as an $R$-algebra, we have a surjective homomorphism of rings
$\varphi\colon R[X_1, \ldots, X_k]\to S_0$. Then $S_0$ is isomorphic to a quotient
of $R[X_1, \ldots, X_k]$, which is noetherian by the Basissatz. Quotients
of noetherian rings are noetherian rings: indeed, $R[X_1, \ldots, X_n]/\ker\varphi$
is a noetherian $R[X_1, \ldots, X_n]$-module, which implies that it is a
$R[X_1, \ldots, X_n]/\ker\varphi$-module.

As a finitely generated module over a noetherian ring, we find that $T$ is
a noetherian $S_0$-module. Since $S$ is an $S_0$-submodule of $T$, we find that
$S$ is finitely generated as a $S_0$-module.

This allows us to write every element of $S$ as a polynomial in the generators
of $S$ as an $S_0$-module and the $s_{ij}$ and $s_{ijk}$, so $S$ is a finitely
generated $R$-algebra.
