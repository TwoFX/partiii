If all submodules of $M$ are finitely generated and $N_1 \subseteq N_2 \subseteq \ldots$ is
an increasing chain of submodules of $M$, define $N\coloneqq \bigcup_{i = 1}^\infty N_i$.
This is a submodule of $M$, so it is finitely generated with generators
$m_1, \ldots, m_k$. Each $m_i$ lies in some $N_{n_i}$. If $n$ is the maximum of all
$n_i$, we have $N_n = N$ and the chain is stationary.

If the ACC holds and $\mathcal{S}$ is nonempty, let $M_0 \coloneqq \set{0}$.
Proceed inductively. If $M_i$ is maximal, we are done. Otherwise, there is some
$M_{i+1}$ such that $M_i\subsetneq M_{i+1}$. By the ACC, this process must terminate
after a finite number of steps.

If the maximum condition holds and $N$ is any submodule of $M$, define $\mathcal{S}$
to be the collection of finitely generated submodules of $N$. $\mathcal{S}$ is nonempty
as it contains the zero module. Let $L$ be a maximal member of $\mathcal{S}$. Let
$x \in N$. Then $L + Rx$ is finitely generated and $L \subseteq L + Rx$, hence, $x \in L$
and therefore $N = L$.
