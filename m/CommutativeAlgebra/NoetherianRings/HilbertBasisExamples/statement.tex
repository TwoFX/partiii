\begin{itemize}
	\item Fields are notherian.
	\item PIDs are noetherian.
	\item Let $p$ be a prime number. $\set{\frac{m}{n}\given m, n \in \mathbb{Z}, p \nmid n}$ is an example
		of a localization of $\mathbb{Z}$ (at $p$). All localizations of noetherian rings are noetherian.
	\item $k[X_1, X_2, \ldots]$ is not noetherian, as there are is an infinite chain $(X_1) \subsetneq (X_1, X_2) \subsetneq \cdots$.
	\item $k[X_1, \ldots, X_n]$ is noetherian, by Hilbert's basis theorem and induction.
	\item $\mathbb{Z}[X_1, \ldots, X_n]$ is noetherian: every finitely generated commutative ring is noetherian,
		since if $R$ is generated by $r_1, \ldots, r_n$, we have a surjective map
		$\mathbb{Z}[X_1, \ldots, X_n]\to R$ given by $X_i\mapsto r_i$.
	\item Group algebras of free abelian groups of finite rank: if $A$ is an abelian group,
		the group algebra of $A$ is the free $\mathbb{Z}$-module with basis  $A$. It is
		an $A$-algebra with the multplication defined as the $\mathbb{Z}$-bilinear
		continutation of $(a, b)\mapsto ab$. If $A$ is generated by
		$g_1, \ldots, g_n$, then $\mathbb{Z}A$ is generated as a ring by
		$g_1, g_1^{-1}, \ldots, g_n, g_n^{-1}$.

	\item The ring of formal power series $k[[X]]$ is noetherian if $k$ is noetherian, see below.
\end{itemize}

Here are some non-commutative rings which are left and right noetherian:
\begin{itemize}
	\item The enveloping algebra of a finite dimensional Lie agebra.
	\item The Iwasawa algebras of compact $p$-adic groups.
\end{itemize}
