Recall that the Nullstellensatz gives a bijection between
radical ideals $\mathbb{C}[X_1, \ldots, X_n]$ and algebraic subsets of
$\mathbb{C}^n$.

If $I$ is a radical ideal of $\mathbb{C}[X_1, \ldots, X_n]$, then
$(a_1, \ldots, a_n)$ is a common zero of all $f \in I$ if and only if
$I \subseteq (X_1 - a_1, \ldots, X_n-a_n)$\footnote{Indeed, if
$\set{(a_1, \ldots, a_n)} \subseteq V(I)$, then
$I = \sqrt{I} = I(V(I)) \subseteq I(\set{(a_1, \ldots, a_n)}) =
(X_1 - a_1,\ldots, X_n-a_n)$. Conversely, if $I \subseteq (X_1 - a_1, \ldots
X_n - a_n)$, then $\set{(a_1, \ldots, a_n)} \subseteq V(I)$. To see that
$I(\set{(a_1, \ldots, a_n)}) = (X_1 - a_1, \ldots, X_n - a_n)$, note that
\enquote{$\supseteq$} is clear, but the latter is maximal as we have seen
before.}. Consider the ideal
\[J\coloneqq \bigcap_{(a_1, \ldots, a_n) \in V(I)} (X_1-a_1, \ldots, X_n-a_n), \]

This is a radical ideal (TODO: why?). The bijection in the Nullstellensatz
tells us that $I=J$. Therefore, we may write any radical ideal as the
intersection of maximal ideals it is contained in, which are all of the form
$(X_1-a_1, \ldots, X_n-a_n)$ (as we already know).

Furthermore, Hilbert's Nullstellensatz tells us that if $J \subseteq \mathbb{C}[X_1, \ldots, X_n]$ is an ideal, then $N(\mathbb{C}[X_1, \ldots, X_n]/J) = J(\mathbb{C}[X_1, \ldots, X_n]/J)$.
