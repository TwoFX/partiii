Suppose that $M\neq 0$. Define $\mathcal{S}$ to be the collection of proper
submodules of $M$. Then $(0) \in \mathcal{S}$, and if we have an ascending chain
of proper submodules, then the union is also a proper submodule (otherwise all
generators would already lie in one of the proper submodules). So by Zorn,
there is a maximal proper submodule $M_1$.

The the quotient $M/M_1$ is a simple module, as we can pullback any submodule of
$M/M_1$ to a submodule of $M$ lying between $M_1$ and $M$. If $0\neq m \in M/M_1$,
the submodule generated by $m$ is all of $M/M_1$.

The homomorphism $R \to M/M_1$ of $R$-modules given by $r\mapsto rm + M_1$ is
surjective. If $I$ is the kernel of this map, then there is an isomorphism of
 $R$-modules $M/M_1 \cong R/I$, but since the former is a simple $R$-module, so is
the latter. Now if $J$ is an ideal of $R/I$, then it is also an $R$-submodule of
$R/I$, which shows that $R/I$ has only two ideals, so it is a field. This means
that $I$ is a maximal ideal.

Let $n \in M$. Since $m$ generates $M/M_1$, we can write $n = rm + m'$ for some
$r \in R$, $m' \in M_1$. If $i \in I$, then $in = rim + im' \in M'$, since
$im \in M'$ by definition of $I$. This means that $IM \subseteq M_1$.

Since $I$ is maximal, we have $J(R) \subseteq I$, and so
\[ J(R)M \subseteq IM \subseteq M_1 \subsetneq M, \]
contrary to our assumption.
