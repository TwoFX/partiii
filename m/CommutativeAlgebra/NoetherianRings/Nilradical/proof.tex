If $x \in N(R)$, then there is $m \in \mathbb{N}$ such that $x^m = 0$,
which implies $(rx)^m = 0$, so $rx \in N(R)$. If $x, y \in N(R)$, there
are $n, m \in \mathbb{N}$, $x^n = y^m = 0$. Then $(x + y)^{m + n - 1}$ is
a linear combination of terms $\lambda x^sy^t$ with $s + y = m + n -  1$. In
particular, $s\geq n \vee t\leq m$, and so  $(x + y)^{m + n - 1} = 0$ and
$x + y \in N(R)$.

Furthermore, if $s \in R/N(R)$, then $s = x + N(R)$. If $s$ is nilpotent, i.e.,
$s^n = 0$, then $0 = s^n = (x + N(R))^n = x^n + N(R)$, i.e., $x^n \in N(R)$.
That means that for some $m$ we have $x^{nm} = 0$, so $x \in N(R)$, so
$s = 0$.
