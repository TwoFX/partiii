In a commutative ring, $N(R)\leq J(R)$. They are in general not equal,
take for example $R_p = \set{\frac{m}{n} \in \mathbb{Q}\given p\nmid n}$ for
some prime $p$. This has a unique maximal ideal $p = \set{\frac{m}{n} \in \mathbb{Q}\given p\mid n, p \nmid n}$,
but it is an integral domain, so $N(R) = (0)$ while $J(R) = p$

On the other hand, for $R = k[X_1, \ldots, X_n]/I$, where $k$ is algebraically
closed and $I$ is any ideal, then we do indeed have $N(R) = J(R)$. This is
Hilbert's Nullstellensatz.
