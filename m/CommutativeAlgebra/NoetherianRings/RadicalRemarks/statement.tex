It is unsubstantial whether $0$ is allowed as an exponent or not: if
$r^0 = 1 \in I$, then $I = R$, so $r^1 \in I$.

We have an equality  $\sqrt{I} + I = N(R/I)$ of ideals of $R/I$.

$\sqrt{I}$ is the intersection of all prime ideals that contain $I$:
$\sqrt{I}/I$ is the intersection of all prime ideals of $R/I$, then use the
correspondence between prime ideals of $R/I$ and prime ideals of $R$ that
contain $I$.
