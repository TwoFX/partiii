Let $m$ be a maximal ideal of $T$. Define $R\coloneqq T/m$. This is a field.
By Zariski's lemma, $k \subseteq T/m$ is a finite algebraic extension. If $k$ is
algebraically closed and $T = k[X_1, \ldots, X_n]$, then this means that the map
natural map
$\Phi\colon k\to k[X_1, \ldots, X_n]\to k[X_1, \ldots, X_n]/m$ is an
isomorphism. Let $a_i\coloneqq \Phi^{-1}(X_i)$. Then we have that
$I\coloneqq (X_1-a_1, \ldots, X_n-a_n) \subseteq \ker\Phi = m$.

On the other hand the natural map  $k \to k[X_1, \ldots, X_n]/I$ is injective,
because the kernel is not trivial and $k$ is a field, and it is surjective,
because every polynomial in the quotient by $I$ \enquote{reduces} to an element
of $k$, so $I$ is maximal, so $I = m$ since $m\supseteq I$ is a proper ideal.
