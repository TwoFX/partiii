Given a ring homomorphism $\varphi_1\colon R\to T_1$ (which in particular makes
$T_1$ into an $R$-module), we say that $T_1$ together with $\varphi_1$ is
an $R$-algebra. Given another ring homomorphism $\varphi_2\colon R\to T_2$,
we can take the tensor product of the $R$-moodules $T_1$ and $T_2$ to
give $T_1 \tensor_R T_2$. We can declare a product on $T_1\times T_2$ by
\begin{align*}
	(T_1\tensor T_2)\times(T_1\tensor T_2)&\to T_1\tensor T_2\\
	(t_1\tensor t_2, t_1'\tensor t_2') &\mapsto t_1t_1'\tensor t_2t_2'.
\end{align*}
As usual, it needs to be checked that this map actually exists:
first, we notice that multiplication is bilinear, hence it induces a
map $T_1\times T_i \to T_i$. The composite
\[ (T_1\tensor T_1)\times (T_2\tensor T_2) \to T_1\times T_2 \to T_1\tensor T_2 \]
is again bilinear, hence it induces a map
\begin{align*}
	(T_1\tensor T_1)\otimes (T_2\tensor T_2)&\to T_1\tensor T_2\\
	(t_1\tensor t_1')\tensor (t_2\tensor t_2') &\mapsto t_1t_1' \tensor t_2t2'.
\end{align*}
By (3.4), we can reassociate and permute this to a map
\[ (T_1\tensor T_2)\tensor (T_1\tensor T_2) \to T_1\tensor T_2, \]
and we see that composition with the tensoring map gives exactly the map
we postulated above. Hence the product exists, and $1\tensor 1$ is the
multiplicative identity. This makes $T_1\tensor T_2$ into a ring and we have
an $R$-algebra structure via $r\mapsto \varphi_1(r)\tensor 1 = 1\tensor\varphi_2(r)$.
