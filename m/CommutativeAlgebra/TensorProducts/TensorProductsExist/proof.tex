Let $F$ be the free $R$-module on the generators $e_{(m, n)}$ indexed by pairs
$(m, n) \in M\times N$. Let $X$ be the $R$-submodule generated by all elements of
the forms
\[ e_{(r_1m_1 + r_2m_2, n)} - r_1e_{(m_1, n)}-r_2e_{(m_2, n)},\quad e_{(m, r_1n_1 + r_2n_2)} - r_1e_{(m, n_1)} - r_2e_{(m, n_2)}. \]
Define $T\coloneqq F/X$ and write $m\tensor n$ for the image of the basis element
$e_{(m, n)}$ in $T$. $T$ is generated by elements of the form $m\tensor n$, and we
have the relations
\begin{align*}
	(r_1m_1 + r_2m_2)\tensor n &= r_1(m_1\tensor n) + t_2(m_2\tensor n)\\
	m\tensor (r_1n_1 + r_2n_2) &= r_1(m\tensor n_1) + r_2(m\tensor n_2).
\end{align*}
Define $\varphi M\times N\to T$ via $(m, n)\mapsto m\tensor n$ and note that
$\varphi$ is bilinear. Any map $\alpha\colon M\times N\to L$ extends to a map
of $R$-modules $\overline{\alpha}\colon F\to L$ by sending
$e_{(m, n)} \mapsto \alpha(m, n)$. If $\alpha$ is bilinear then $\overline{\alpha}$
vanishes on the generators of $X$, hence it induces a map of $R$-modules
$\alpha'\colon T\to L$ such that $\alpha'(m\tensor n) = \alpha(m, n)$, and
$\alpha'$ is uniquely determined by these relations. Hence  $\varphi$ is
universal.

The proof of uniqueness is just the usual dance with universal properties.
