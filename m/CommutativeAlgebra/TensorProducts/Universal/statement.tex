The idea is to reduce the study of bilinear maps to the of linear (i.e, $R$-module)
maps.

If $\varphi\colon M\times N\to T$ is bilinear and $\theta\colon T\to L$ is
linear, then $\theta \circ \varphi$ is bilinear. Composition with $\varphi$
gives a well defined function $\varphi^*$ from  $R$-linear maps $T\to L$ to
bilinear maps $M\times N\to L$.
\[\begin{tikzcd}[row sep=1cm, column sep=1cm]
	M\times N\ar[r, "\varphi"]\ar[dr, "\varphi^*(\theta)" below left] & T\ar[d, "\theta"]\\
	& L
\end{tikzcd}\]
We say that $\varphi$ is universal if $\varphi^*$ is a bijection for every $L$.
If this happens that study of bilinear maps $M\times N\to L$ is reduced to
the study if linear maps $T\to L$.
