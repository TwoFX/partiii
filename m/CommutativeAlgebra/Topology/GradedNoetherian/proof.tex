First suppose $T$ is a noetherian ring.
We have an isomorphism of $T$-modules $T_0\cong T/T_+$. This means that $T_0$ is
a noetherian $T$-module. Observe that elements from $T_i$ for $i\neq 0$ act  trivially
on $T_0$, so $T_0$ is actually a Noetherian $T_0$-module. Hence, $T_0$ is a noetherian
ring.

Since $T$ is noetherian, $T_+$ is a finitely generated ideal. Splitting generators
into components, we may assume that  $T_+$ is generated by homogeneous elements
$x_1, \ldots, x_s$ of degrees $m_1, \ldots, m_s$, and $m_i > 0$ for all $i$. Define
$T' \coloneqq T_0[x_1, \ldots, x_s]$. It will suffice to show that $T_i \subseteq T'$
for every $i\geq 0$. We will prove this by induction. The case $i = 0$ is trivial.

If $i > 0$, let $y \in Y_i$. Then $y \in T_+$, so we may write $y = \sum_{j=1}^s t_jx_j$,
where $t_j \in T_{i - m_j}$ (after possibly removing other summands which cancel
each other out in different degrees). Then the inductive hypothesis says that
$t_j \in T'$ for each $j$, and so $y \in T'$ as claimed.

Conversely, if $T_0$ is noetherian and $T$ is a finitely generated $T_0$-algebra,
then $T$ is a surjective image of a polynomial ring over  $T_0$ in finitely many
variables, so the claim follows from Hilbert's basis theorem.
