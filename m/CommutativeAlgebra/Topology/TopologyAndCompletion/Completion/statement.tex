The $I$-adic completion $\hat{M}$ of $M$ is the set of equivalence classes of
Cauchy sequences. If $(x_n)$, $(y_n)$ are Cauchy sequences, so is $(x_n + y_n)$ and
its class only depends on the classes of $(x_n)$ and $(y_n)$. The $I$-adic completion
$\hat{R}$ of $R$ is the set of equivalence classes of Cauchy sequences with the
$I$-adic topology. We get a well-defined product via $[r_n][s_n] = [r_ns_n]$ and
addition as aobve.

Hence (after checking a whole bunch of stuff), we have made $\hat{R}$ into a ring
and we make $\hat{M}$ into an $R$-module in the obvious way. We get a homomorphism
of rings $R\to \hat{R}$ by sending $r$ to the constant sequence $[r]$. The
kernel of this map is easily seen to be $\bigcap_{n \in \mathbb{N}} I^n$. Similarly,
we have a homomorphism of abelian groups (and also of $R$-modules?)
$M\to \hat{M}$ sending $x\mapsto [x]$. The kernel is $\bigcap I^nM$.

Krull's intersection theorem can be interpreted as giving sufficient conditions for
these maps to be injective. For example, $R\to \hat{R}$ is injective if $R$ is
an integral domain, and $M\to \hat{M}$ is injective if $I \subseteq J(R)$.
