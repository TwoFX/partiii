Let $\phi\colon S\to S$ be an $R$-linear map. Then either
$\phi(S) = 0$ and hence $\phi = 0$ or $0\neq \phi(S) = S$ since $S$ is simple.
Furthermore, $\ker \phi$ is a submodule of $S$, so it is either $0$ or $S$. Hence,
if $\phi\neq 0$, it is an isomorphism (so it has a two-sided inverse).
Thus, $\End_R(S)$ is a division ring.

If $S_1$ and $S_2$ are non-isomorphic simple right $R$-modules and
$0\neq\phi\colon S_1\to S_2$ is  $R$-linear, then $\im\phi = S_2$ and
$\ker\phi = 0$, hence $\phi$ is an isomorphism, which is a contradiction.
