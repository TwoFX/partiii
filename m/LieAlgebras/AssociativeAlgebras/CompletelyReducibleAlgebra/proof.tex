By semisimplicity, the interaction of all maximal right ideals is trivial.
Consider
$R \supsetneq I_1 \supsetneq I_1\cap I_2\supsetneq\ldots$,
where $I_i$ are maximal right ideals. This must terminate since we are dealing
with finite-dimensional $k$-vector spaces, i.e., there is some $n$ such that
$0 = I_1\cap\ldots \cap I_n$. Choose $n$ minimal. Consider the homomorphism
of right $R$-modules $\theta\colon R\to \bigoplus R/I_i$ defined via
$r\mapsto (r+I_1, \ldots, r+I_n)$. Note that $\cap_{j\neq i} I_j\neq 0$ by
minimality and if the restriction of the quotient map $R \to R/I_i$ to
$\bigcap_{j\neq i} I_j$ is injective (since the kernel is just the intersection
of all $I_i$). Hence, the image is non-zero in $R/I_i$, but the quotient is
simple, so the map is actually an isomorphism $\bigcap_{j\neq i} I_j\cong R/I_i$.
In particular, $I_2\cap\cdots\cap I_n$ gets mapped to
$(R/I_1, 0, \ldots, 0)$ under $\theta$, and so we see that $\theta$ is
surjective, so it is an isomorphism as the kernel of $\theta$ is again
the intersection of the $I_i$.
