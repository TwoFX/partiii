An $R$-$R$-bimodule $M$ is an abelian group which is both a left $R$-module and
a right $R$-module with the obvious associativity property: $(rm)s = r(ms)$.

A right $R$-module can be thought of a left $R^\op$-module, where $R^\op$ is
the opposite ring.

Thus an $R$-$R$-bimodule may be viewed as a left $R^\op\tensor_k R$-module.
Similarly it is a right $R^\op\tensor_k R$-module.

For example, $(kG)^\op\cong kG$ via the map $g\mapsto g^{-1}$.

If $R$ is finite-dimensional, then $R^\op\tensor R$ is finite-dimensional, so the
unique decomposition property holds by Krull-Schmidt.

Hence we have a decomposition  $R\cong \bigoplus B_j$ into indecomposable ideals
(i.e., sub-bimodules, i.e., $R^\op\tensor R$-submodules)
that are unique up to reordering.
