Let $M = \bigoplus_{i=1}^m M_i = \bigoplus_{i = 1}^n M_i'$. We will proceed by
induction on $m$. If $m = 1$, then $M$ is indecomposable, so the claim follows.

Next, assume that $m>1$. Let
\[ \alpha_i\colon M_i'\to M\to M_1\qquad \beta_i\colon M_1\to M\to M_i' \]
be the obvious maps. Then $\sum (\alpha_i \circ \beta_i)$ is the identity on $M_1$.
Now since $\End_R(M_1)$ is local by assumption, some of the $\alpha_i \circ \beta_i$
must be invertible. Otherwise, they would all be in the maximal ideal, so
their sum would be in the ideal and thus not equal to the identity.

This means that we find a left inverse $f\colon M_1 \to M_1$ of $\alpha_i \circ \beta_i$.
But then the short exact sequence
\[\begin{tikzcd}
	0\ar[r] & \im \beta_i\ar[r, hook, "\iota"] & M_i'\ar[r] & \coker\iota\ar[r] & 0
\end{tikzcd}\]
splits on the left via the map $\beta_i \circ f \circ \alpha_i\colon M_i'\to \im\beta_i$,
hence $M_i'\cong \im\beta_i\oplus \coker \iota$ by the splitting lemma, so we conclude $M_i' = \im\beta_i$
by indecomposability. Hence $\beta_i$ is surjective, so $\beta_i$ is an isomorphism,
so $\alpha_i = \alpha_i \circ \beta_i \circ \beta_i^{-1}$ is also an isomorphism.

Renumber the $M_i'$ such that $\alpha_i \circ \beta_i$ is invertible and
$M_1\cong M_1'$. Our next will be to find an $R$-linear automorphism of $M$ sending
$M_1\to M_1'$.

Consider the map $\mu = 1 - \theta$, where $\theta$ is the composite
\[\begin{tikzcd}
	M\ar[r, two heads] & M_1\ar[r, "\alpha_1^{-1}"] & M_1'\ar[r, hook] & M\ar[r, two heads] & \bigoplus_{i = 2}^m M_i\ar[r, hook] & M.
\end{tikzcd}\]
Observe that $\mu(M_1') = M_1$. Indeed, on $M_1'$, $\theta$ acts like
$M_1' \to M \to \bigoplus_{i=2}^m M_i$, so on $M_1'$ $\mu$ acts like,
$M_1'\to M\to M_1$, and we know that this is surjective.
Moreover, $\mu(\bigoplus_{i=2}^m M_i) = \bigoplus_{i=2}^m M_i$, since $\theta$
vanishes on this submodule (this is obvious: it is precisely the kernel of
the first map).
This shows that $\mu$ is surjective.

Next, if $\mu(x) = 0$, then $x = \theta(x)$, so $x \in \bigoplus_{i=2}^m M_i$ (since
that is the image of the final map). But then $x = \theta(x) = 0$ as seen above.
Hence $\mu$ is an automorphism of $M$ satisfying $\mu(M_1') = M_1$. Hence
\[\bigoplus_{i = 2}^n M_i' \cong M/M_1' \cong M/M_1 \cong \bigoplus_{i=2}^m M_i, \]
using $\mu$ in the second step. Hence, we are done using the inductive step.
