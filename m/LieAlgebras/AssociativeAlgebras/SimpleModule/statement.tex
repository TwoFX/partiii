An $R$-module $M$ is called simple if its only submodules are $0$ and $M$.

Note that $I$ is a maximal right ideal if and only if $R/I$ is a simple right
$R$-module.

Let $M$ be a simple right $R$-module and $m \in M$. Then
$\Ann_R(m) = \set{r \in R\given mr = 0}$ is a right ideal, but not necessarily
a two-sided ideal.

However, $\Ann_R(M) = \bigcap_{m \in M} \Ann(m)$, the annihilator of the module,
is a two-sided ideal: if $r \in \Ann_R(M)$ and $x \in R$, then $m(xr) = (mx)r = 0$,
since $r \in \Ann_R(mx)$. Hence $xr \in \Ann_R(M)$, so $\Ann_R(M)$ is a left
ideal.

If $M$ is simple, then the $\Ann_R(m)$ for $m\neq 0$ are maximal right ideals,
because $mR = M$ by simplicity of $M$, so $R/\Ann_R(m)\cong M$
(via the map $r\mapsto mr$) is simple, so $\Ann_R(m)$ is maximal. So we can see
that $J(R) = \bigcap_{M~\text{simple}} \Ann_R(M)$ is a two-sided ideal (all
ideals of the for  $\Ann_R(M)$ are maximal, and if $I$ is a maximal right ideal,
then $I = \Ann_R(R/I)$).
