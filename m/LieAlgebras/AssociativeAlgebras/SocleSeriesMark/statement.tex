\begin{enumerate}
	\item The series is strict until $\soc_i(M) = M$.
	\item By the previous lemma, $\soc_i(M) = \set{m \in M\given mJ(R)^i = 0}$.
		Indeed, the case $i = 0$ is trivial, and if we know that the claim is
		true for $i$ and if $\pi\colon M\to M/\soc_i(M)$ is the quotient map,
		then by definition and the inductive claim we have
		\begin{align*}
			\soc_{i+1}(M) &= \pi^{-1}(\soc(M/\soc_i(M)))\\
			&= \set{m \in M\given (m + \soc_i(M))J(R) = 0}\\
			&= \set{m \in M\given mJ(R) \in \soc_i(M)}\\
			&= \set{m \in M\given mJ(R)^{i+1}}
		\end{align*}
		as required.
\end{enumerate}
