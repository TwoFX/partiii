For (a), take $u, v, x$ as in the previous proof, and let $A$ be the Lie
subalgebra generated by $u$ and $v$ and $N$ the vector space span of
$v$, $H$ and $\sum_{r>0} L_{r\alpha}$.

We can calculate $[u, N] \subseteq H \oplus \sum L_{r\alpha} \subseteq N$,
noting that $[u, v] \in [L_\alpha, L_{-\alpha}] \subseteq L_0 = H$.

Similarly, $[v, N] \subseteq [v, H] + \sum_{r>0} [v, L_{r\alpha}] \subseteq N$,
again using the addition formula for the second term. TODO: why is $[v, H] \subseteq N$?

So $[A, N] \subseteq N$. Then $x = [u, v] \in A^{(1)}$. Consider
$\ad(x)|_N\colon N\to N$. We have $0 = \tr\ad(x)|_N$ as $x$ is in the derived
subalgebra (TODO: but it's the derived subalgebra of $A$ and not $N$. Why does
that not matter?). Hence
\[ 0 = -\alpha(x) + \sum m_{r\alpha}r\alpha(x) = \left(-1 + \sum rm_\alpha\right)\alpha(x). \]
But $\alpha(x)\neq 0$ by part (b) of the previous lemma. Hence
$\sum rm_{\alpha} = 1$ for all $\alpha \in \Phi$. Thus for $\alpha \in \Phi$
we have $m_\alpha = 1$ and if $n\alpha$ is a root, then $n = \pm 1$ (the negative
case comes from considering $-\alpha$ in the argument above.

Part (b) is obtained by combining (a) with 4.15(a).
