For (a), define $M = \sum_{r=-q}^p L_{\beta + r\alpha}$. Observe that
$[L_{\pm\alpha}, M] \subseteq M$ by maximality of $p$ and $q$.
Let $U$ be the Lie subalgebra generated by $L_{\alpha}$ and $L_{-\alpha}$.
Then $\ad(U)(M) \subseteq M$.

Now take $x \in [L_\alpha, L_{-\alpha}]$. We have that $x \in M^{(1)}$, so
$\ad(x)|_M\colon M\to M$ (exists by the above) has zero trace,
since it is an element of the derived subalgebra of $\ad(M)$. But then
\[ 0 = \tr \ad(x)|_M = \sum_{r=-q}^p m_{\beta + r\alpha}(\beta + r\alpha)(x). \]
Rearranging gives part (a). Note that $\sum_{-q}^p m_{\beta+r\alpha}$ is nonzero
since multiplicities are positive and $\beta$ is a root, hence its multiplicity
is nonzero.

For part (b), let $0\neq x \in [L_\alpha, L_{-\alpha}]$ and suppose that
$\alpha(x) = 0$. Then we deduce from (a) that $\beta(x) = 0$ for all
roots $\beta$. This contradicts 4.13(f). Hence $\alpha(x)\neq 0$.

Finally, for (c) let $v \in L_{-\alpha}$. By definition, we have
$[h, v] = -\alpha(h)v$ for $h \in H$.

Choose $u \in L_\alpha$ and $v \in L_{\alpha}$ such that $\vspan{u, v}_{\ad} \neq 0$.
This is possible by 4.13(e). Furthermore, choose $h \in H$ such that $\alpha(h)\neq 0$.
Define $x\coloneqq[u, v] \in [L_\alpha, L_{-\alpha}]$. Then
$\vspan{x, h}_{\ad} = \vspan{u, [v, h]}_{\ad} = \alpha(h)\vspan{u, v}_{\ad} \neq 0$.

In particular, $x\neq 0$ as required.
