For (a) choose a basis for each weight space and take the union to obtain
a basis of $L$. Then $\ad(x)$ and $\ad(y)$ are represented by diagonal matrices,
and $\tr(\ad(x) \circ \ad(y))$ is precisely the right hand side of the formula.

The proof of the first part of (b) is similar to the proof of 4.2(i) (TODO). By
an argument similar to that used in 4.10(d) we have that
$\tr(\ad(x) \circ \ad(y)) = 0$ if $x \in L_{\alpha}$ and $y \in L_{\beta}$ and
$\alpha+\beta\neq 0$ (TODO). So $\vspan{L_\alpha, L_{\beta}}_{\ad} = 0$.

For (c), take $\alpha \in \Phi$ and suppose $-\alpha \in \Phi$. Then using (b) we
have that for all weights $\beta$, $\vspan{L_\alpha, L_\beta}_{\ad} = 0$.

Hence, $\vspan{L_{\alpha}, L}_{\ad} = 0$. But by Cartan-Killing
$\vspan{\ ,\ }_{\ad}$ is non-degenerate on $L$, so $L_{\alpha} = 0$, which is
a contradiction by definition of a root.

Part (d) is the same as 4.10(d).

Take $x \in L_\alpha \cap L_{-\alpha}^\perp$. Then
$\vspan{x, L_\beta}_{\ad} = 0$ for all weights $\beta$, since if $\beta \neq -\alpha$,
this is true by (b), and  if $\beta = -\alpha$, then it is true by
choice of $x$. Hence, $\vspan{x, L}_{\ad} = 0$, so $x = 0$ by non-degeneracy.

Suppose $h \in H$ is such that $\alpha(h) = 0$ for all $\alpha \in \Phi$. Let
$x \in H$. Then $\vspan{h, x}_{\ad} = \sum_{\alpha \in \Phi}m_\alpha\alpha(h)\alpha(x) = 0$.
By (d), we have $h = 0$.

Thus, if $h\neq 0$, there is some $\alpha \in\Phi$ such that $\alpha(h)\neq 0$.
TODO: why does this imply that $\Phi$ spans $H^*$?
