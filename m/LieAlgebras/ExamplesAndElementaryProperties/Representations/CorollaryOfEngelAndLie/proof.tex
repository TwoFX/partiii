We will prove both parts at the same time by induction on $\dim V$.

By (2.12) and (2.13) we can pick a common eigenvector $v_1$ of $\mathfrak{L}$.

Then  $\mathfrak{L}(\vspan{v_1}) \subseteq \vspan{v_1}$. Define
$V_1\coloneqq \vspan{v_1}$.
Define $\overline{\mathfrak{L}}\coloneqq \set{\overline{x}\given x \in \mathfrak{L}} \subseteq \End (V/V_1)$
where $\overline{x}(v + V_1) = x(v) + V$ for  $x \in \mathfrak{L}, v \in V$.
This definition makes sense because $V_1$ is invariant under the action of $\mathfrak{L}$.

$\overline{L}$ inherits the properties of $\mathfrak{L}$. By the inductive hypothesis,
$\overline{\mathfrak{L}}$ is represented by (strictly) upper triangular matrices
with regard to the basis $v_2 + V_1, \ldots, v_n + V_1$ of $V/V_1$. Then
$v_1, \ldots, v_n$ is a basis of $V$ with respect to which $\mathfrak{L}$ is represented
by (strictly) upper triangular matrices .
