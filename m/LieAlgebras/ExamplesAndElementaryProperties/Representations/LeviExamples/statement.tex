\begin{enumerate}
	\item $\mathfrak{L} = \mathfrak{gl}_2$.
		Then $R(\mathfrak{L}) = Z(L)$, where $Z(L)$ are the matrices of the form
		$\lambda I$. Indeed, $\mathfrak{L}/R(\mathfrak{L})\cong \mathfrak{sl}_2$
		is semisimple (TODO: why?)

		By Levi's theorem, we find that $\mathfrak{L} = \mathfrak{sl}_2 + Z(\mathfrak{L})$,
		and $\mathfrak{sl}_2$ is the Levi subalgebra of $\mathfrak{gl}_2$.

	\item
		Let $\mathfrak{L}$ be the subalgebra of $\mathfrak{gl}_4$ consisting of
		matrices of the form
		\[
			\begin{pmatrix}
				\mathfrak{sl}_2 & \star\\0&\mathfrak{sl}_2
			\end{pmatrix}.
			\]

		Then $R(\mathfrak{L})$ consists of matrices of the form
		\[
			\begin{pmatrix}
				0 & \star\\0 & 0
			\end{pmatrix}.
			\]

		This is soluble, and in fact nilpotent. The Levi subalgebra consists
		of matrices of the form
		\[
			\begin{pmatrix}
				\mathfrak{sl}_2&0\\0&\mathfrak{sl}_2
			\end{pmatrix}.
			\]
		So $\mathfrak{L}_1\cong \mathfrak{sl}_2\times \mathfrak{sl}_2$.

	\item
		Let $\mathfrak{L}$ be the subalgebra of $\mathfrak{gl}_4$ consisting of
		matrices of the form
		\[
			\begin{pmatrix}
				\mathfrak{gl}_2 & \star\\0&\mathfrak{gl}_2
			\end{pmatrix}.
			\]

		Then $R(\mathfrak{L})$ consists of matrices of the form
		\[
			\begin{pmatrix}
				\lambda I & \star\\0 & \mu I
			\end{pmatrix}, \]
		which is soluble but not nilpotent.

		Now we have $\mathfrak{L}/R(\mathfrak{L}) \cong \mathfrak{gl}_2/\set{\lambda I}\times \mathfrak{gl}_2/\set{\mu I} \cong \mathfrak{sl}_2\times \mathfrak{sl}_2$.
		So the Levi subablegra is the same as in the previous example.
\end{enumerate}
