Let $\mathfrak{L}$ be a $2$-dimensional Lie algebra. Either $\mathfrak{L}$ is
abelian or there are $x, y$ such that $[x, y] \neq 0$, so $\mathfrak{L}^{(1)} \neq 0$.

However, $x$ and $y$ form a basis of $\mathfrak{L}$, $\mathfrak{L}^{(1)}$ is
equal to the span of $[x, y]$. Therefore, the derived series of $\mathfrak{L}$ looks
like \[ \mathfrak{L} \supsetneq \mathfrak{L}^{(1)} \supsetneq 0. \]

So in the first case, where $\mathfrak{L}$ is abelian, the derived length is $1$,
and otherwise the derived length is $2$.

Annoying exercise: classify three-dimensional Lie algebras. It is done in
Jacobson's book.
