If $L/Z(L)$ is nilpotency class $n$, then all expressions of the form
$[[\cdots[[x_0, x_1], x_2]\cdots], x_n]$ are contained in $Z(L)$. Hence
all expressions of the form $[[\cdots[[x_0, x_1], x_2]\cdots], x_{n+1}]$ vanish
in $L$, i.e., $L$ is nilpotent of nilpotency class at most $n+1$.
