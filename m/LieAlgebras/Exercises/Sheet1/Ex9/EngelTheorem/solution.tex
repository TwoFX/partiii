If $L$ is nilpotent, then $\ad(x)$ is obviously nilpotent for all $x \in L$.

Conversely, if $\ad(x)$ is nilpotent for all $x \in L$, then $\ad(L) \subseteq \End L$
satisfies the condition of Engel, hence by (2.14) $\ad(L)$ is isomorphic to a subalgebra of
$\mathfrak{n}_n$ for some $n$. In particular $\ad(L)$ is nilpotent. But we
have $L/Z(L)\cong \ad(L)$, so by the previous lemma $L$ is nilpotent.
