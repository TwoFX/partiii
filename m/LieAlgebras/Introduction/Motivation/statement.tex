Groups describe symmetries. Lie algebras describe infinitesimal symmetries.

For example, let $G = \GL_n(\mathbb{R})$. This is an example of a Lie group,
i.e., an analytic manifold with continuous group operations. The associated
Lie algebra is the tangent space $T_1G$ at the identity.

The matrix exponential diffeomorphically (with inverse $\log$) takes a
neighborhood of $0$, which is the same as $T_1G$, to a neighborhood of $1$. 

$\exp A\exp B = \exp(\mu(A, B))$ for sufficiently small $A$ and $B$.

The Taylor series for $\mu$ is
\[ \mu(A, B) = A + B + \frac{1}{2}[A, B] + \text{higher degree terms}, \]
where $[A, B] = AB - BA$ (matrix multiplication).

This is an example of a Lie bracket. Note that $T_1G\times T_1G \to T_1G$,
$(A, B)\mapsto [A, B]$ is bilinar, skew-symmetric.

The Lie algebra corresponding to $G$ is often called $\mathfrak{g}$.

Note that
\begin{enumerate}[label=(\arabic*)]
	\item The first approximation to the group product is addition in the Lie algebra
		$T_1G$.
	\item If $[g_1, g_2] = g_1g_2g_1^{-1}g_2^{-1}$ is the group commutator, then
		the Lie bracket is the first approximation of the commutator
		$[\exp A, \exp B]$ in $G$.
	\item The Jacobi identity arises from the associativity in $G$. Note that
		Lie algebras in general are non-associative.
\end{enumerate}

As a further example, let $G = \GL_n(\mathbb{C})$. This is an example of an
algebraic group, i.e., a complex algebraic variety with continuous group
operations. We have $T_1G \cong M_n(\mathbb{C})$ as the tangent space at the
identity. Similarly to before, we define a Lie bracket and end up with a complex
Lie algebra.
