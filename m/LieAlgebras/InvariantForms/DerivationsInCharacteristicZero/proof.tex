Let $D$ be a derivation of $\mathfrak{L}$ and $x \in \mathfrak{L}$. Then
for every $y \in \mathfrak{L}$ we have
\begin{align*}
	[D, \ad_{\mathfrak{L}}(x)](y) &= (D \circ \ad_{\mathfrak{L}}(x) - \ad_{\mathfrak{L}}(x) \circ D)(y)\\
	&= D([x, y]) - [x, D(y)]\\
	&= [D(x), y] + [x, D(y)] - [x, D(y)]\\
	&= [D(x), y]\\
	&= \ad_{\mathfrak{L}}(D(x))(y),
\end{align*}
so we conclude that
\begin{equation}\tag{$\star$}
[D, \ad(x)] = \ad(D(x)).
\end{equation}
The centre $Z(\mathfrak{L})$ of $\mathfrak{L}$ is an abelian ideal, hence zero
(since $\mathfrak{L}$ is semisimple)

Since $\mathfrak{L}$ is semisimple
and the kernel of the map $\mathfrak{L} \to \ad_{\mathfrak{L}}$ is an abelian ideal,
it must be zero (since it is trivially soluble), hence
$\mathfrak{L}\cong \ad(L)$.

Let $\vspan{\ ,\ }$ denote the Killing form on $\Der \mathfrak{L}$. By question
13 from the example sheet, the restriction of $\vspan{\ ,\ }$ to $\ad(\mathfrak{L})$
is the Killing form on $\ad(\mathfrak{L})$.

Let $J$ be the orthogonal space to $\ad(\mathfrak{L})$ inside $\Der(\mathfrak{L})$
with respect to $\vspan{\ ,\ }$. By 3.3(ii) $J$ is an ideal of $\Der \mathfrak{L}$.
Now, since $\mathfrak{L}$ is semisimple, so is $\ad(\mathfrak{L})$, and by the
Cartan-Killing criterion,  $\vspan{\ ,\ }$ restricted to $\ad(\mathfrak{L})$ is
non-degenerate. Hence $\ad(\mathfrak{L})\cap J = 0$ and
$[\ad(\mathfrak{L}), J] \subseteq \ad(\mathfrak{L})\cap J = 0$,
since both are ideals.

Thus if $D \in J$, then for all $x \in \mathfrak{L}$ we have $\ad(D(x)) = 0$ by
($\star$). Thus, $D(x) \in Z(\mathfrak{L}) = 0$, since $\mathfrak{L}$ is semisimple,
so $D$ is the zero derivation, and we conclude $J = 0$. This can only happen
if $\Der(\mathfrak{L}) = \ad(\mathfrak{L})$ (by linear algebra) and so we are done.
