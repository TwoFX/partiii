Symmetry follows from $\tr x \circ y = \tr y \circ x$. Bilinearity is
immediate.

For $x, y, z \in \mathfrak{L}$, we have
\begin{align*}
	\vspan{[x, y], z} &= \tr(\rho([x, y]) \circ \rho(z))\\
	&= \tr([\rho(x), \rho(y)] \circ \rho(z))\\
	&= \tr(\rho(x)\circ\rho(y)\circ\rho(z)) - \tr(\rho(y)\circ\rho(x)\circ\rho(z))\\
	&= \tr(\rho(x)\circ\rho(y)\circ\rho(z)) - \tr(\rho(x)\circ\rho(z)\circ\rho(y))\\
	&= \tr(\rho(x) \circ [\rho(y), \rho(z)])\\
	&= \tr(\rho(x) \circ \rho([y, z]))\\
	&= \vspan{x, [y, z]},
\end{align*}
so the trace form is invariant\footnote{Note that we even have $\vspan{[x, y], z} = 0 = \vspan{x, [y, z]}$.}.
This completes the proof of (i).

Next, let $J$ be a Lie ideal. Let $x \in J^\perp$, $y \in \mathfrak{L}$.
We will show that $[x, y] \in J^\perp$. Indeed, let $z \in J$.
Then  $[y, z] = -[z, y] \in J$ since $J$ is a Lie ideal. But then
$\vspan{[x, y], z} = \vspan{x, [y, z]} = 0$ since $x \in J^\perp$ and we are
done.
