\begin{enumerate}
	\item Let $R = k[X]/(X^p)$. Then there is only one isomorphism class of
		simple modules, the trivial module $S = k$ with $X$ acting like
		$0$\footnote{Let $S$ be a simple $R$-module. Suppose there is some
		$s \in S$ such that $Xs \neq 0$. Then $S = RXs$ by simplicity, so
		we find $r \in R$ such that $s = rXs = r^pX^ps = 0$, a contradiction.
		Hence $X$ acts like $0$ on $S$ and may define a map $k \to S$ via
		$\lambda\mapsto \lambda x$, where $x$ is any non-zero element of $S$.
		This map is $R$-linear. The kernel is an $R$-submodule, so it is a
		$k$-submodule, and it is not $k$, so it is zero. Since $S = Rx$ by
		simplicity, the map is surjective. Hence $S\cong k$ as an $R$-module
		as required.}.

		There is the indecomposable $X_1\coloneqq k[X]/(X^2)$ and a short exact sequence
		\[\begin{tikzcd}
			0\ar[r] & (X)/(X^2)\ar[r] & X_1\ar[r] & k[X]/(X)\ar[r] & 0,
		\end{tikzcd}\]
		where the left and right terms are both isomorphic to $S$. Hence, the
		Ext quiver has at least one arrow $S\to S$.

		By the structure theorem of modules over the PID $k[X]$, the
		finite-dimensional $k[X]/(X^p)$-modules are of the form
		$\bigoplus_i k[X]/(X^{r_i})$ for some $r_i\leq p$. Hence, for $X_i$ to
		be indecomposable, it must be of the form $k[X]/(X^r)$ for some $r$,
		but there is no injective map $k^n\to k[X]/(X^r)$ for any $r$ and any
		$n \geq 2$, since the fact that $X$ acts like $0$ on the left hand
		side forces the entire map to have image in $(X^{r-1})/(X^r)$.

		Hence, the Ext-quiver consists of one vertex $S$ with one arrow $x$.

		Thus, $kQ\cong k[X]$ as a $k$-algebra, and the $kQ$-modules are just
		representations $M$ with a linear map $\theta_x\colon M\to M$.

	\item Let $R = kS_3$, where $\chr k = 2$. There are two isomorphism classes of
		simple $R$-modules: the trivial $S_1$ and the two-dimensional $S_3$.

		We saw that $kS_3= kC_2\oplus M_2(k)$ was the block decomposition.
		In characteristic $2$, we have $kC_2 = k[X]/(X-1)^2$. We have an exact
		sequence
		\[\begin{tikzcd}
			0\ar[r] & \displaystyle\frac{(X-1)}{(X-1)^2}\ar[r] & \displaystyle\frac{k[X]}{(X-1)^2}\ar[r] & \displaystyle\frac{k[X]}{(X-1)}\ar[r]& 0,
		\end{tikzcd}\]
		and $kC_2\cong k[X]/(X-1)^2$ is indecomposable. Hence there is a loop $x\colon S_1\to S_1$
		in the Ext quiver. In fact, the Ext quiver is
		given by the loop $x$ and an isolated vertex for $S_3$ (TODO: work this out).
		Observe that we had two blocks and have two components in the Ext quiver.

		As an additional exercise, work out the case where $\chr k = 3$. In this
		case there are two isomorphism classes of simple modules, both of dimension
		$1$, the trivial module $S_1$ and the signature $S_2$. One should show that there
		is an indecomposables $X_1$ that fits in a sequence
		\[\begin{tikzcd}
			0\ar[r] & k\ar[r] & X_1\ar[r] & k\ar[r] & 0,
		\end{tikzcd}\]
		where the left $k$ is the signature and the right $k$ is trivial. Furthermore,
		one should show that there is $X_2$ that fits into
		\[\begin{tikzcd}
			0\ar[r] & k\ar[r] & X_2\ar[r] & k\ar[r] & 0,
		\end{tikzcd}\]
		where this time the left $k$ is trivial and the right $k$ is the signature.
		Hence, the Ext quiver contains two vertices and at least one arrow in either
		direction.
		As an exercise (TODO) show that this is already the full Ext quiver.

		There is one block and one component.

		In both characteristics, there is a directed cycle, so the path algebra
		is not finite-dimensional in either case.
\end{enumerate}
