First observe that if $Q$ has finite representation type, then any subquiver
obtained by removing a subset of the vertices is also of finite representation
type (because any representation of the subquiver can be promoted to a representation
on $Q$ by making everything else zero. Of course, this representation is still
indecomposable).

We have already seen that if we have a directed cycle, then $Q$ does not have
finite representation type. We have also seen that if any two vertices are
connected by more than one edge, then the result does not have finite
representation type. We deduce that any subquiver obtained as above leaving
two vertices has at most one edge.

From this, we deduce that the underlying graph of $Q$ must be a tree.
Recall that we
defined a symmetric bilinear form on the real span of the vertices via
\[ q(v_i, v_j) =
\begin{cases}
	2, & i = j,\\
	-1, & \text{there is an edge between $i$ and $j$},\\
	0, & \text{otherwise}.
\end{cases}\]
Suppose that this symmetric bilinear form is not positive definite. Then
we find non-negative integers $k_i$ such that $q(v, v)\leq 0$ with
$v = \sum k_iv_i \neq 0$\footnote{To see that we may assume the $k_i$ to be positive
integers, approximate by a rational and multiply with a large number to get
to integers. Then write $v = v_+ + v_-$, where  $v_+$ is the sum of those
$k_iv_i$ where $k_i > 0$ and $v_-$ is defined analogously. Now
$q(v, v) = q(v_+, v_+) + 2q(v_+, v_-) + q(v_-, v_-)$. But notice that the middle
term is nonnegative, since $q(v_i, v_j)$ is zero or negative, and $k_ik_j$ is
negative. Hence, either $q(v_+, v_+)$ or $q(v_-, v_-)$ is negative, so replace
$v$ either by $v_+$ or by $-v_-$.}.

Thus $2\sum k_i^2\leq 2\sum_{\text{$i, j$ connected}} k_ik_j$, and hence
\begin{equation}\tag{$\star$}
	\sum k_i^2 \leq \sum_{\text{$i, j$ connected}} k_ik_j.
\end{equation}
Let $M_i$ be a vector space of dimension $k_i$. We will show that there are
infinitely many isomorphism classes of representations with this dimension
vector $\sum k_iv_i$.

We need to assigne linear maps $\theta_x\colon M_i\to M_j$ for each edge
$x\colon i\to j$.

Two such representations are isomorphic if and only if there are automorphisms
$\prod_i \GL(M_i)$ such that the diagram
\[\begin{tikzcd}
	M_i\ar[r, "\theta_x"]\ar[d, "f_i"] & M_j\ar[d, "f_j"]\\
	M_i'\ar[r, "\theta_x'"] = M_i & M_j' = M_j
\end{tikzcd}\]
commutes.

We will consider oribts of $\prod_i \GL(M_i)$ on $\prod_{x\colon i\to j} \Hom(M_i, M_j)$.
Two representations are isomorphic if and only if the homomorphisms
representing the arrows yield elements of  $\prod_i \Hom(M_i, M_j)$ in the same
orbit. But $\prod \GL(M_i')$ is an algebraic variety of dimension
$\sum k_i^2$. The dimension of $\prod \Hom(M_i, M_j)$ is $\sum_{x\colon i\to j} k_ik_j$.

Notice that the scalar multiplication in $\prod \GL(M_i)$ act trivially
on $\prod \Hom(M_i, M_j)$ and so we have that
\[ V\coloneqq \frac{\prod \GL(M_i)}{k^\times} \]
operates on the homs. We have $\dim V = \left(\sum k_i^2\right) - 1$. By ($\star$),
we find $\dim V < \sum k_ik_j = \dim \prod \Hom(k_ik_j)$, and so the orbits
have dimension strictly less than $\dim \prod\Hom(M_i, M_j)$. This implies
that we have infinitely many orbits, which means that there are infinitely
many isomorphism classes of representations with dimension vector
$\sum k_iv_i$.

But if $Q$ has finite represenation type, then there are only finitely
many indecomposable representations. By Krull-Schmidt, every representation
decomposes as a direct sum of indecomposables in an essentially unique way.
Hence, there are only finitely many representations of a given dimension
up to isomorphism. In particular, there can only be finitely many isomorphism
classes of representations of a given dimension vector, so we have arrived
at a contradiction.

Hence, the underlying graph is positive definite.
