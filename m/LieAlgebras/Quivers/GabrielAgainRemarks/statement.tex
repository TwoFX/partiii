\begin{enumerate}
	\item There is an alternative proof analogous to the strategy used in
		the classification of positive definite Coxeter graphs. We find
		certain quivers that cannot appear as full subquivers.
	\item Suppose that $k$ is not algebraically closed.  In this situation, we should
		modify the definition of a basic algebra to say that $R/J(R)$ is a direct
		sum of division algebras and so has simple modules corresponding to each
		division algebra (which were the endomorphism rings of the simple $R$-modules).

		So when we look at the Ext quiver, it is a good idea to include information
		in the quiver about the endomorphism algebras of the simple modules.

		This additional information is of a similar sort to stipulating that
		vertices have different lengths with respect to the symmetric
		bilinear form. In fact, we get a positive definite quiver (with this
		additional information). We get a positive definite Coxeter graph (not
		necessarily simply laced).
\end{enumerate}
