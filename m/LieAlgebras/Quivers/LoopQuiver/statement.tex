Consider now that quiver $Q$ which has a single vertex $1$ and a loop $x\colon 1\to 1$.
Consider the representation given by $M_1 = k$ and $\theta_x\colon \lambda\mapsto \lambda\mu$
for some fixed $\mu \in k$.

The corresponding $kQ$-module is indecomposable, and for distinct $\mu_1, \mu_2$
these representations are not isomorphic. Indeed, if
$\sigma\colon M_1^{\mu_1}\to M_1^{\mu_2}$ is a morphism of representations, then
the commutativity condition in the definition of representation says that
$\mu_1\sigma(1) = \mu_2\sigma(2)$. Hence, if $\mu_1\neq \mu_2$, then $\sigma(1) = 0$,
hence $\sigma$ is not an isomorphism.

In particular, if $k$ is infinite, then there are infinitely many
indecomposables.

This generalises to the case where $Q$ has a directed cycle: take a directed
cycle, put a copy of $k$ at each vertex of the cycle, and make each arrow of the
path act like the identity, except for one, which acts as multiplication by some
non-zero element. Make
everything else zero. Then we get a representation, and it is indecomposable, since
it is generated as a $kQ$-module by any non-zero element of any of the copies of
$k$, and so it is generated by any non-zero element (projecting to a component using
$e_i$ if necessary). Again, different choices of the multiplicative constant lead to non-isomorphic
representations.

Exercise (TODO): what if $k$ is finite?

%We can even do it if $k$ is finite and has at least three elements: instead of
%putting a copy of $k$ at each
%vertex, we put $k^{\oplus \mathbb{N}}$ at each vertex. Again, all but one of the
%edges act like the identity, while the final edge sends to $i$-th component to
%the $i+1$-th component and multiplies by a factor (that depends on $i$). Again,
%this representation is generated as a $kG$-module by a single element: the element
%of the $k^{\oplus \mathbb{N}}$ at the destination of the special edge that is
%$1$ in the first component and $0$ in all other components.

Next, let $Q$ be the quiver considered at the beginning of the chapter. For
every $\mu \in k$, we get a representation by setting $M_1 = M_2 = k$,
$\theta_x = \id$ and $\theta_y\colon \lambda\mapsto \lambda\mu$. These representations
are indecomposable for the same reasons as seen before, and they are pairwise
non-isomorpic, since if $\mu_1, \mu_2 \in k$ and we have an isomorphism
$\sigma\colon M^{\mu_1}\to M^{\mu_2}$, then the compatibility condition for
$\theta_x$ says that $\sigma_1 = \sigma_2$ (as maps $k\to k$) and thus if we set
$\lambda\coloneqq \sigma_1(1) = \sigma_2(1)$, then the compatibility condition
for $\theta_y$ gives that $\lambda\mu_1 = \lambda\mu_2$. Since $\sigma$ is an
isomorphism, $\lambda\neq 0$, hence $\mu_1=\mu_2$.
