Let $Q$ be a finite quiver with no directed cycles (hence $kQ$ is finite dimensional).

Let $J$ be the $k$-span of paths of length $\ell\geq 1$. Then
$J^r$ is the $k$-span of paths of length $\ell\geq r$. Since $Q$ is finite and has
no directed cycles, we find $J^n = 0$ for some $n$. Thus $J \subseteq J(kQ)$
(for example using Nakayama's lemma and the standard telescoping trick).

Now $kQ/J\cong k\oplus\cdots\oplus k$, where we get one copy of $k$ for every
vertex.

Recall that $S_i$ was the simple $kQ$-module corresponding to the
representation that has $k$ at vertex $i$, $0$ everywhere else and all maps
are zero.

Now if an element of $J(kQ)$ has a non-zero component for a path of length $0$,
then by multiplying with the appropriate $e_i$ and scaling we find $e_i \in J(kQ)$
for some $i$. But then $e_iS_i \neq 0$, which is a contradiction since
$J(kQ)$ is the intersection of the annihilators of the simple modules, so in particular
$e_i$ should annihilate $S_i$. Hence we conclude $J(kQ) \subseteq J$, i.e.,
$J = J(kQ)$.

Now using Artin-Wedderburn and the explicit description of $kQ/J$ from above,
we conclude that the $S_i$ are actually all simple $kQ$-modules.
