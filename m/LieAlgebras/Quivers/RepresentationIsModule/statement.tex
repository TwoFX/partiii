Suppose $M = \bigoplus M_i$ is a representation of a quiver $Q$. If
$x\colon i\to j$ is an arrow, then $x$ acts on $M_i$ via $\theta_x$ and
on all other $M_\ell$ as zero.

Also, $e_i$ acts like the projection onto the component $M_i$.

Thus we can regard $\bigoplus M_i$ as a module over the path algebra $kQ$. In
fact, we have a correspondence between $kQ$-modules and quiver representations.

Note that there are simple $kQ$-modules $S_i$ for each vertex $i$ corresponding to
the representation which is $k$ at the vertex $i$, zero everywhere else and has
all maps trivial.
