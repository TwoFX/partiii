A subset $\Delta$ of a root system $\Phi$ is called a base
of $\Phi$ if
\begin{enumerate}[label=(\roman*)]
	\item $\Delta$ is a basis of the Euclidean space $E$,
	\item each $\gamma \in \Phi$ can be written as a linear
		combination $\gamma  = \sum_{\alpha \in \Delta} k_{\alpha}\alpha$,
		where the $k_{\alpha}$ are  either all positive integers or
		negative integers.
\end{enumerate}

Elements of $\Delta$ are called the simple roots and the $\gamma$ with all
$k_{\alpha} \geq 0$ are called positive roots with regard to $\Delta$, and
the other roots are called the negative roots.

We will show that every root system has a base in due course.
