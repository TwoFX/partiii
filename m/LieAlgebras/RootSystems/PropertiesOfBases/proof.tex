For (a), if  $\alpha - \beta \in \Phi$, then part (ii) of the definition of base
would be violated. We already saw $(\alpha, \beta) \leq 0$ in the previous lemma.

If $(\alpha, \beta)\leq 0 $ for all $\beta \in \Delta$, then $\Delta \cup \set{\alpha}$
would be linearly independent. Hence, we have $(\alpha, \beta) > 0$ for
some $\beta \in \Delta$, and using the same argument as in the previous
lemma, $\alpha - \beta$ is a root.

For (b), if $\alpha = \sum_{\gamma \in \Delta} k_{\gamma}\gamma$, then $k_\gamma$ for
at least two $\gamma \in \Delta$, and so for at least one $\gamma \neq \beta$. So
$\alpha - \beta \in \Phi^+$.

(c) follows from (b) via an induction on the sum of the coefficients.

For (d), if $\beta = \sum_{\gamma \in\Delta} k_{\gamma}\gamma \in \Phi^+\setminus\set{\alpha}$,
then there is some $k_\gamma > 0$ with $\gamma\neq \alpha$. But the coefficient
of $\gamma$ in $s_\alpha(\beta) = \beta - 2\frac{(\beta, \alpha)}{(\alpha, \alpha)}\alpha$
is $k_\gamma >0$. So $s_\alpha(\beta) \in \Phi^+$, so $s_\alpha(\beta) \in \Phi^+\setminus{\alpha}$.

For $\rho = \frac{1}{2}\sum_{\beta \in \Phi^+}\beta$ set $\rho' = \rho - \alpha/2$.
We have $s_\alpha(\rho') = \rho'$ since $s_\alpha$ permutes. Thus
$s_\alpha(\rho) = \rho - \alpha$.
