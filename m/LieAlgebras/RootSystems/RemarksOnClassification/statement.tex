We conclude with a few remarks on the proof of the classification of connected
Coxeter graphs arising from irreducible reduced root systems.

Given such a Coxeter graph, one can define a symmetric bilinear form on the
$\mathbb{R}$-span of the vertices $v_i$ represented with respect to the maxtrix
$v_1, \ldots, v_r$ by a matrix $(q_{ij})$ with $q_{ij} = 2$ for $i = j$ and otherwise
$q_{ij} = -\sqrt{t_ij}$, where $t_{ij}$ is the number of edges joining $v_i$ to
$v_j$.

If the graph is coming from the simple roots $\Delta$ if a root system $\Phi$ in
Euclidean space $E$, then $q_{ij} = 2\frac{(\alpha_i, \alpha_j)}{(\alpha_i, \alpha_i)}$,
where $(\ ,\ )$ is the inner product in $E$.

Note that the matrix is the same as $2$ times the matrix representing the inner product
$(\ ,\ )$ with respect to the basis $\set{\frac{\alpha_i}{\abs{\alpha_i}}\given \alpha_i \in \Delta}$,
the normalised simple roots.

The matrix is therefore positive definite. Our task is to classify positive definite
Coxeter graphs, i.e., Coxeter graphs for which the bilinear form as defined above
is positive definite. Some remarks:
\begin{enumerate}
	\item Positive semidefinite Coxeter graphs are also of interest for infinite
		Lie algebras.
	\item Positive definite Coxeter graphs also arise in the representation theory
		of quivers (directed graphs).
\end{enumerate}
