A subset $\Phi$ of a real Euclidean vector space $E$ is called a finite root system
if
\begin{enumerate}[label=(\alph*)]
	\item $\Phi$ is finite, spans $E$ and does not contain $0$,
	\item for each $\alpha \in \Phi$ there is a reflection $s_\alpha$
		preserving the inner product such that $s_\alpha(\alpha) = -\alpha$,
		the set of fixed points of $s_\alpha$ is a hyperplane of $E$ and
		$s_\alpha$ leaves $\Phi$ invariant,
	\item for each $\alpha, \beta \in \Phi$,
		$s_\alpha(\beta) - \beta$ is an integral multiple of $\alpha$,
	\item for $\alpha, \beta \in \Phi$, $2\frac{(\beta, \alpha)}{(\alpha, \alpha)} \in \mathbb{Z}$, and
	\item $s_\alpha(\beta) = \beta - 2\frac{(\beta, \alpha)}{(\alpha, \alpha)}\alpha$ for
		all $\beta \in E$.
\end{enumerate}
